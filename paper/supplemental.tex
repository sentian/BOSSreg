%!TEX root = boss.tex
\beginsupplement
\appendix
\pagenumbering{arabic}
\begin{center}
\textbf{\large Supplemental Material \\
On the use of information criteria for subset selection in least squares regression}

Sen Tian, Clifford M. Hurvich, Jeffrey S. Simonoff
\end{center}

\section{Proof of theorem \ref{thm:hdf_ydf_representation} and its corollary}
\label{sec:proof_hdf_ydf}
In this section, we assume an orthogonal $X$ and a null true model. This is the only scenario under which both df$_C(k)$ and hdf$(k)$ have analytical expressions. We will prove that the ratio of df$_C(k)$ and hdf$(k)$ goes to $1$ as $k,p\rightarrow \infty$ while $k=\left \lfloor{xp}\right \rfloor $, where $\left \lfloor{\cdot}\right \rfloor$ denotes the greatest integer function and $x\in(0,1)$. We start by laying out a few lemmas to be used in the proof of the main theorem.
\begin{lemma}
	\label{lemma:hdf_nulltrue}
	Assume the design matrix is orthogonal and the true model is null ($\mu=0$). Then
	\begin{equation}
	\text{hdf}(k) = df_L(\lambda_k^\star) = k - 2p\cdot \Phi^{-1} \left(\frac{k}{2p}\right) \cdot \phi\left[\Phi^{-1}\left(\frac{k}{2p}\right) \right].
	\label{eq:hdf_nulltrue}
	\end{equation}
\end{lemma}
\begin{proof}
	We follow the steps described in algorithm \ref{alg:hdf}. We first find $\lambda_k^\star$ from \eqref{eq:thdf_size_expression}, by using the fact that $\mu=0$, and we get $\displaystyle -\frac{\sqrt{2\lambda_k^\star}}{\sigma} = \displaystyle \Phi^{-1}\left(\frac{k}{2p}\right)$, which we then substituted into \eqref{eq:thdf_expression} to get \eqref{eq:hdf_nulltrue}.
\end{proof}

\iffalse
\begin{lemma}
	\label{lemma:hdf_limitends}
	Assume the design matrix is orthogonal and the true model is null ($\mu=0$), with a fixed $p$, and by treating $k$ as continuous, we have
	\begin{equation}
	\lim_{k\to 0} \text{hdf}(k) = 0 \quad \text{and} \quad \text{hdf}(p) = p.
	\end{equation}
\end{lemma}
\begin{proof}
	By \eqref{eq:hdf_nulltrue}, we have
	\begin{equation}
	\text{hdf}(p) = p - 2p\cdot \Phi^{-1} (\frac{1}{2}) \cdot \phi \left[\Phi^{-1}(\frac{1}{2}) \right]=p,
	\end{equation}
	since $\Phi^{-1} (\frac{1}{2})=0$. Meanwhile, 
	\begin{equation}
	\begin{aligned}
	\lim_{k\to 0} \text{hdf}(k) &= \lim_{k\to 0} - 2p\cdot \Phi^{-1} (\frac{k}{2p}) \cdot \phi\left[\Phi^{-1}(\frac{k}{2p}) \right],\\
	&= \lim_{x \to -\infty} -2p \cdot x \cdot \phi(x),\\
	&= \lim_{x \to -\infty} -2p \cdot x \cdot \frac{1}{\sqrt{2\pi}} \exp^{-x^2/2},\\
	&= 0,
	\end{aligned}
	\end{equation}
	where the last step is given by the L'Hopital rule. 
\end{proof}
\fi

\begin{lemma}
	\label{lemma:G(x)}
	Define $\tilde{G}(x)=  x-\Phi^{-1}(x)\cdot \phi \left[\Phi^{-1}(x)\right]$, where $x\in (0,1)$ is a continuous variable. We have
	\begin{equation*}
	\lim_{x\to 0} \tilde{G}(x) = 0,
	\end{equation*}
	and 
	\begin{equation*}
	\tilde{G}^\prime(x) = \left[\Phi^{-1}(x)\right]^2.
	\end{equation*}
	Therefore by the fundamental theorem of calculus,
	\begin{equation*}
	\tilde{G}(x) = \int_{0}^{x} \left[\Phi^{-1}(u)\right]^2 du.
	\end{equation*}
\end{lemma}
\begin{proof}
	First note that, since $\phi^\prime(v)= -v\cdot \phi(v)$ and $\lim_{v\to \pm \infty} \phi^\prime(v) =0$, we have
	\begin{equation*}
	\lim_{v\to \pm \infty} v \cdot \phi(v) = 0.
	\end{equation*}
	Let $v=\Phi^{-1}(x)$. Then
	\begin{equation*}
	\lim_{x\to 0} \tilde{G}(x)  = \lim_{v\to -\infty} -v \cdot \phi(v) = 0.
	\end{equation*}
	\iffalse
	and 
	\begin{equation}
	\lim_{x\to 1} \tilde{G}(x)  = 1- \lim_{v\to \infty} v \cdot \phi(v) = 1,
	\end{equation}
	\fi
	
	Next, we obtain the derivative of $\tilde{G}(x)$. Since $\Phi^\prime(x) = \phi(x)$, we have
	\begin{equation}
	\left[\Phi^{-1}(x)\right]^\prime = \frac{1}{\Phi^\prime \left[\Phi^{-1}(x)\right]}=\frac{1}{\phi \left[\Phi^{-1}(x)\right]}.
	\label{eq:G(x)_derivative_1}
	\end{equation}
	Also since $\phi^\prime(x) = -x\cdot \phi(x)$, we have
	\begin{equation}
	\phi^\prime\left[\Phi^{-1}(x)\right] = - \Phi^{-1}(x) \cdot \phi\left[\Phi^{-1}(x)\right] \cdot \left[\Phi^{-1}(x)\right]^\prime = -\Phi^{-1}(x).
	\label{eq:G(x)_derivative_2}
	\end{equation}
	By \eqref{eq:G(x)_derivative_1} and \eqref{eq:G(x)_derivative_2}, we have
	\begin{equation*}
	\tilde{G}^\prime(x) = 1 - \left[\Phi^{-1}(x)\right]^\prime \cdot  \phi\left[\Phi^{-1}(x)\right] - \left[\Phi^{-1}(x)\right] \cdot \phi^\prime\left[\Phi^{-1}(x)\right] =  \left[\Phi^{-1}(x)\right]^2.
	\end{equation*}
	
	Therefore, by the fundamental theorem of calculus, we have
	\begin{equation*}
	\tilde{G}(x) = \int_{0}^{x} \tilde{G}^\prime(u) du + \tilde{G}(0) = \int_{0}^{x} \left[\Phi^{-1}(u)\right]^2 du.
	\end{equation*}
\end{proof}

\begin{lemma}
	\label{lemma:sigmasq}
	Denote $\tilde{Q}$ as the quantile function of a $\chi_1^2$ distribution, and let $\tilde{H}(s) = -\tilde{Q}(1-s)$ where $s\in (0,1)$. For $0\le s \le t \le 1$, consider the truncated variance function
	\begin{equation}
	\tilde{\sigma}^2(s,t) = \int_{s}^{t} \int_{s}^{t} (u \wedge v -uv) d \tilde{H}(u) d \tilde{H}(v),
	\label{eq:sigmasq}
	\end{equation}
	where $u \wedge v =\min(u,v)$. We have
	\begin{equation*}
	0 \le \tilde{\sigma}^2(s,t) \le 1.
	\end{equation*}
\end{lemma}
\begin{proof}
	We first note three facts.
	\begin{align}
	\tilde{H}(s) &= -\left[\Phi^{-1}\left(1-\frac{s}{2}\right) \right]^2=-\left[\Phi^{-1}\left(\frac{s}{2}\right) \right]^2,\label{eq:Hs} \\
	d\tilde{H}(s) &= \frac{\Phi^{-1}(1-s/2)}{\phi\left[\Phi^{-1}(1-s/2)\right]}ds=-\frac{\Phi^{-1}(s/2)}{\phi\left[\Phi^{-1}(s/2)\right]}ds, \quad \text{by \eqref{eq:G(x)_derivative_1}},\label{eq:dH} \\
	\Phi^{-1}(w) &= -\sqrt{\log\frac{1}{w^2} - \log\log\frac{1}{w^2} - \log(2\pi)} + o(1), \quad \text{for small $w$, by \citetonline{Fung2017}}. \label{eq:Phiinv_order}
	\end{align}
	Hence for small $w$,
	\begin{equation}
	\label{eq:Phiinvsq_order}
	\left[\Phi^{-1}(w)\right]^2 = O\left(\log \frac{1}{w^2}\right).
	\end{equation}
	Then by \eqref{eq:Hs} and \eqref{eq:Phiinvsq_order}, we have
	\begin{equation}
	\label{eq:sHs_limit}
	\lim_{s \to 0} s\cdot \tilde{H}(s) = \lim_{s \to 0} -s\cdot \left[\Phi^{-1}\left(\frac{s}{2}\right)\right]^2 = 0.
	\end{equation}
	Also, by \eqref{eq:Hs} and Lemma \ref{lemma:G(x)},
	\begin{equation}
	\label{eq:Hs_integral}
	-\int_{0}^{x} \tilde{H}(s) ds =  2\cdot \tilde{G}\left(\frac{x}{2}\right).
	\end{equation}
	Since $u,v\in[0,1]$, we have $u \wedge v-uv \ge 0$. By \eqref{eq:dH}, we have $d\tilde{H}(s)/ds \ge 0$. Therefore, the integrand in \eqref{eq:sigmasq} is non-negative, so that
	\begin{equation*}
	\tilde{\sigma}^2(s,t) \ge 0,
	\end{equation*}
	and 
	\begin{equation*}
	\begin{aligned}
	\tilde{\sigma}^2(s,t) &\le \int_{0}^{1} \int_{0}^{1} (u \wedge v -uv) d \tilde{H}(u) d \tilde{H}(v),\\
	&= \int_{0}^{1} \left[\int_{0}^{v} u(1-v) d\tilde{H}(u) + \int_{v}^{1} v(1-u) d\tilde{H}(u)   \right]  d\tilde{H}(v),\\
	&= \int_{0}^{1} \left[\int_{0}^{v} u d\tilde{H}(u) + v\int_{v}^{1} d\tilde{H}(u) -v\int_{0}^{1}u d\tilde{H}(u)  \right]  d\tilde{H}(v).
	\end{aligned}
	\end{equation*}
	Denote 
	\begin{equation*}
	\tilde{M}(v) = \int_{0}^{v} u d\tilde{H}(u) + v\int_{v}^{1} d\tilde{H}(u) -v\int_{0}^{1}u d\tilde{H}(u).
	\end{equation*}
	Now, we consider the three integrals in $\tilde{M}(v)$. First note that
	\begin{equation*}
	\begin{aligned}
	\int_{0}^{x} u d\tilde{H}(u) &= u\cdot \tilde{H}(u) \Bigr|_{0}^x - \int_{0}^{x} \tilde{H}(u) du,\\
	&= x\cdot \tilde{H}(x) - \int_{0}^{x} \tilde{H}(u) du,\quad \text{by \eqref{eq:sHs_limit}}\\
	&=  x\cdot \tilde{H}(x) + 2\cdot \tilde{G}(x/2), \quad \text{by \eqref{eq:Hs_integral}}.
	\end{aligned}
	\end{equation*}
	It is easily verified that $\tilde{H}(1)=0$ and $\tilde{G}(1/2)=1/2$, we have
	\begin{equation*}
	\int_{0}^{1} u d\tilde{H}(u) = 2\cdot \tilde{G}(1/2)=1,
	\end{equation*}
	and
	\begin{equation*}
	v\int_{v}^{1} d\tilde{H}(u) = -v\cdot \tilde{H}(v).
	\end{equation*}
	Therefore,
	\begin{equation*}
	\begin{aligned}
	\tilde{M}(v) &= v\cdot \tilde{H}(v) +2 \cdot \tilde{G}(v/2)-v\cdot \tilde{H}(v)-2v\cdot \tilde{G}(1/2)\\
	&=2 \cdot \tilde{G}(v/2) - v.
	\end{aligned}
	\end{equation*}
	Finally,
	\begin{equation*}
	\begin{aligned}
	\int_{0}^{1} \tilde{M}(v) d\tilde{H}(v) &= \int_{0}^{1} 2 \cdot \tilde{G}(v/2)d\tilde{H}(v) - \int_{0}^{1}v d\tilde{H}(v),\\
	&= -\int_{0}^{1} \Phi^{-1}\left(\frac{v}{2}\right)\cdot \phi\left[\Phi^{-1}\left(\frac{v}{2}\right)\right]d\tilde{H}(v),\quad \text{by the definition of $\tilde{G}(x)$},\\
	&= 2\int_{0}^{1/2} \left[\Phi^{-1}(v)\right]^2 dv, \quad \text{by \eqref{eq:dH}},\\
	&=2\cdot \tilde{G}(1/2),\\
	&=1.
	\end{aligned}
	\end{equation*}
	Therefore,
	\begin{equation*}
	0 \le \tilde{\sigma}^2(s,t) \le 1.
	\end{equation*}
\end{proof}




\begin{theorem}
	\label{thm:ydf_representation}
	Assume the design matrix is orthogonal and the true model is null ($\mu=0$). Let $\tilde{X}_{(i)}$ be the $i$-th largest order statistic in an i.i.d sample of size $p$ from a $\chi^2_1$ distribution. Denote $\tilde{Y}_p = \tilde{\sigma}_p^{-1}(\sum_{i=1}^k \tilde{X}_{(i)} - \tilde{\mu}_p)$, where
	\begin{equation*}
	\tilde{\sigma}_p = \sqrt{p} \cdot \sigma(1/p,k/p),
	\end{equation*}
	and
	\begin{equation*}
	\tilde{\mu}_p = -p \int_{1/p}^{k/p} \tilde{H}(u) du - \tilde{H}\left(\frac{1}{p}\right),
	\end{equation*}
	where $\sigma(s,t)$ and $\tilde{H}(x)$ are defined in Lemma \ref{lemma:sigmasq}.
	
	As $k \to \infty$, $p \to \infty$ and $k=\left \lfloor{px}\right \rfloor$ with $x \in (0,1)$, we have
	\begin{equation}
	\frac{\text{df}_C(k)}{2p} = \frac{1}{2p} E\left[ \sum_{i=1}^k \tilde{X}_{(i)} \right]=  \frac{\tilde{\sigma}_p}{2p}E(\tilde{Y}_p) + \tilde{G}\left(\frac{k}{2p}\right) + O\left(\frac{\log(p)}{p}\right),
	\label{eq:ydf/2p_representation}
	\end{equation}
	where $\left \lfloor{\cdot}\right \rfloor$ denotes the greatest integer function, $\tilde{G}(x)$ is defined in Lemma \ref{lemma:G(x)}.
	
\end{theorem}
\begin{proof}
	We first apply a result in \citetonline{Csorgo1991}, to show that $\tilde{Y}_p=\tilde{\sigma}_p^{-1}(\sum_{i=1}^{k} \tilde{X}_{(i)}-\tilde{\mu}_p)$ converges in distribution to a standard normal. We then show how $\tilde{\mu}_p$ can be expressed in terms of function G plus a remainder term, which further leads to expression \eqref{eq:ydf/2p_representation}. 
	
	It follows from \citetonline{Csorgo1991} Corollary 2, that if there exist centering and normalizing constants $c_p$ and $d_p>0$, s.t.
	\begin{equation}
	d_p^{-1}(\tilde{X}_{(1)} - c_p) \xrightarrow{D} Y, \quad \text{where Y is the standard Gumbel distribution},
	\label{eq:scorgo_condition_gumbel} 
	\end{equation}
	then as $k \to \infty$, $p \to \infty$ and $k=\left \lfloor{px}\right \rfloor$ with $x \in (0,1)$,
	\begin{equation}
	\left(\sum_{i=1}^k \tilde{X}_{(i)}  - \tilde{\mu}_p\right) / \tilde{\sigma}_p \xrightarrow{D} Z, \quad \text{where Z is standard normal}.
	\label{eq:scorgo_result}
	\end{equation}
	
	% https://math.stackexchange.com/questions/450139/asymptotics-of-maxima-of-i-i-d-chi-square-random-variables
	First, it follows from \citetonline{Embrechts2013} that \eqref{eq:scorgo_condition_gumbel} holds, with $c_p=2\log(p)-\log\log(p)-\log(\pi)$ and $d_p=2$.
	\iffalse
	the maxima of $p$ gamma-distributed random variables, $\gamma(\alpha,\beta)$ with $\beta$ being the rate parameter, converges to a Gumbel distribution as $p \to \infty$, where the centering constant $c_p=\beta^{-1}(\log(p) +(\alpha-1)\log\log(p) - \log(\gamma(\alpha)))$, and the scaling constant $d_p=\beta^{-1}$. Meanwhile, we know that $\chi^2_1$ is a $\gamma(1/2,1/2)$ distribution. Therefore, \eqref{eq:scorgo_condition_gumbel} holds, where $c_p=2\log(p)-\log\log(p)-\log(\pi)$ and $d_p=2$.
	\fi
	
	Next, we have
	
	\begin{equation*}
	\begin{aligned}
	\tilde{\mu}_p &= -p \int_{1/p}^{k/p} \tilde{H}(u) du - \tilde{H}\left(\frac{1}{p}\right),\\
	&= -p \int_{0}^{k/p} \tilde{H}(u) du + p \int_{0}^{1/p} \tilde{H}(u) du - \tilde{H}\left(\frac{1}{p}\right),\\
	&= 2p\cdot \tilde{G}\left(\frac{k}{2p}\right) - 2p \cdot \tilde{G}\left(\frac{1}{2p}\right) + \left[\Phi^{-1}\left(\frac{1}{2p}\right)\right]^2, \quad \text{by \eqref{eq:Hs_integral}}.
	\end{aligned}		
	\end{equation*}
	Also, since
	\begin{equation*}
	\begin{aligned}
	\tilde{G}(\frac{1}{2p}) &= \frac{1}{2p} - \Phi^{-1}\left(\frac{1}{2p}\right) \cdot \phi\left[\Phi^{-1}\left(\frac{1}{2p}\right) \right],\quad \text{by definition of $\tilde{G}(x)$ in Lemma \ref{lemma:G(x)}},\\
	&= \frac{1}{2p} - \frac{1}{\sqrt{2\pi}}\Phi^{-1}\left(\frac{1}{2p}\right) \cdot \exp\left(-\frac{1}{2}\left[\Phi^{-1}\left(\frac{1}{2p}\right) \right]^2\right),\\
	&= \frac{1}{2p} + \frac{1}{\sqrt{2\pi}} \cdot \left(\sqrt{\log(4p^2)-\log\log(4p^2)-\log(2\pi)}+o(1)\right)\cdot \\
	& \qquad \exp\left[-\frac{1}{2} \left(\log(4p^2)-\log\log(4p^2)-\log(2\pi) +o(1)\right)\right], \quad \text{by \eqref{eq:Phiinv_order}},\\
	&= \frac{1}{2p} + \left(\sqrt{\log(4p^2)-\log\log(4p^2)-\log(2\pi)}+o(1)\right)\cdot \frac{\sqrt{\log(4p^2)}}{2p},\\
	&= O\left(\frac{\log(p)}{p}\right).
	\end{aligned}
	\end{equation*}
	Also
	\begin{equation*}
	\begin{aligned}
	\frac{1}{2p}\left[\Phi^{-1}\left(\frac{1}{2p}\right)\right]^2 &= O\left( \frac{\log(p)}{p}\right), \quad \text{by \eqref{eq:Phiinv_order}},
	\end{aligned}
	\end{equation*}
	and hence
	\begin{equation*}
	\begin{aligned}
	\frac{\tilde{\mu}_p}{2p} &= \tilde{G}\left(\frac{k}{2p}\right) - \tilde{G}\left(\frac{1}{2p}\right) + \frac{1}{2p}\left[\Phi^{-1}\left(\frac{1}{2p}\right)\right]^2,\\
	&= \tilde{G}\left(\frac{k}{2p}\right
	) + O\left( \frac{\log(p)}{p}\right).
	\end{aligned}
	\end{equation*}
	Therefore, \eqref{eq:ydf/2p_representation} holds, i.e.
	\begin{equation*}
	\frac{\text{df}_C(k)}{2p}=\frac{1}{2p} E\left(\sum_{i=1}^{k} \tilde{X}_{(i)}\right) = \frac{\tilde{\sigma}_p}{2p} E(\tilde{Y}_p) + \frac{\tilde{\mu}_p}{2p}=\frac{\tilde{\sigma}_p}{2p} E(\tilde{Y}_p) + \tilde{G}\left(\frac{k}{2p}\right) + O\left( \frac{\log(p)}{p}\right).
	\end{equation*}
\end{proof}

\begin{corollary}
	\label{corollary:Yp_order}
	If $\limsup |E(\tilde{Y_p})| < \infty$, we further have:
	\begin{equation}
	\frac{\text{df}_C(k)}{2p} = \tilde{G}\left(\frac{k}{2p}\right) + O\left(\frac{\log(p)}{p}\right) + O\left(\frac{1}{\sqrt{p}}\right).
	\label{eq:ydf/2p_representation_remark}
	\end{equation}
\end{corollary}
\begin{proof}
	By Lemma \ref{lemma:sigmasq} we have $0 \le \sigma(1/p,k/p) \le 1$, and hence $\tilde{\sigma}_p = O(\sqrt{p})$. Therefore by Theorem \ref{thm:ydf_representation}, we have
	\begin{equation*}
	\frac{\text{df}_C(k)}{2p}=  \tilde{G}\left(\frac{k}{2p}\right) + O\left( \frac{\log(p)}{p}\right) + O\left(\frac{1}{\sqrt{p}}\right).
	\end{equation*}	
\end{proof}

\dfasy*

\begin{proof}
	By Lemma \ref{lemma:hdf_nulltrue}, we have
	\begin{equation*}
	\text{hdf}(k) = df_L(\tilde{M}^{-1}(k)) = k - 2p\cdot \Phi^{-1} \left(\frac{k}{2p}\right) \cdot \phi\left[\Phi^{-1}\left(\frac{k}{2p}\right) \right].
	\end{equation*}
	Then by the definition of $\tilde{G}(x)$ in Lemma \ref{lemma:G(x)},
	\begin{equation*}
	\frac{1}{2p} \text{hdf}(k) = \tilde{G}\left(\frac{k}{2p}\right).
	\end{equation*}
	By Theorem \ref{thm:ydf_representation}, we also have
	\begin{equation*}
	\frac{1}{2p}\text{df}_C(k) = \frac{\sigma_p}{2p}E(\tilde{Y}_p) + \tilde{G}\left(\frac{k}{2p}\right) + O\left(\frac{\log(p)}{p} \right).
	\end{equation*}
	Therefore, \eqref{eq:hdf_ydf_yp_representation} holds, i.e.
	\begin{equation*}
	\frac{1}{2p} \text{hdf}(k) = \frac{1}{2p}\text{df}_C(k) - \frac{\tilde{\sigma}_p}{2p}E(\tilde{Y}_p) + O\left(\frac{\log(p)}{p} \right).
	\end{equation*}
\end{proof}

\dfasycorollary*
\begin{proof}
	By Theorem \ref{thm:hdf_ydf_representation} and Corollary \ref{corollary:Yp_order},
	\begin{equation*}
	\frac{1}{2p} \text{hdf}(k) = \frac{1}{2p}\text{df}_C(k) + O\left(\frac{1}{\sqrt{p}}\right) + O\left(\frac{\log(p)}{p} \right).
	\end{equation*}
	From Lemma \ref{lemma:G(x)}, $\tilde{G}(x)$ is a non-decreasing function with $\tilde{G}(0+)=0$ and $\tilde{G}(1/2)=1/2$. Thus, 
	\begin{equation*}
	\frac{2p}{\text{hdf}(k)} = \frac{1}{\tilde{G}\left(\frac{k}{2p}\right)} = O(1),
	\end{equation*}
	since $k=\left \lfloor{px}\right \rfloor$ and $x\in(0,1)$. Therefore, 
	\begin{equation*}
	\frac{\text{df}_C(k)}{\text{hdf}(k)} = 1 + O\left(\frac{1}{\sqrt{p}}\right) + O\left(\frac{\log(p)}{p} \right),
	\end{equation*}
	and hence
	\begin{equation*}
	\frac{\text{df}_C(k)}{\text{hdf}(k)} \to 1.
	\end{equation*}
\end{proof}

\section{Expected KL-based optimism, in the context of BS }
\label{sec:expectedkl_bs}
In this section, we obtain the expected Kullback-Leibler (KL) based optimism for BS with subset size $k$. Let's first consider fitting least squares regression on $k$ prefixed predictors. Recall that 
\begin{equation*}
y = \mu + \epsilon,
\end{equation*}
where $\epsilon \sim \mathcal{N}(0,\sigma^2 I)$. We use the deviance to measure the predictive error, that is 
\begin{equation*}
\Theta=-2 \log f(y|\mu,\sigma^2).
\end{equation*}
The training error is then 
$$\text{err}_{\text{KL}} = -2 \log f (y|\hat{\mu},\hat{\sigma}^2),$$
and the testing error (KL information) is
$$\text{Err}_{\text{KL}}  = -2 E_0 \left[ \log f(y^0|\hat{\mu},\hat{\sigma}^2)\right],$$
where $\hat{\mu}$ and $\hat{\sigma}^2$ are the maximum likelihood estimators (MLE) based on training data $(X,y)$, $y^0$ is independent and has the same distribution of $y$ and $E_0$ is the expectation over $y^0$. 

Due to the assumption of normality, the deviance can be expressed as
\begin{equation}
\Theta = n\log(2\pi \sigma^2) + \frac{\lVert y- \mu \rVert_2^2}{\sigma^2}.
\label{eq:deviance}
\end{equation}
Maximizing the likelihood, or minimizing the deviance \eqref{eq:deviance}, gives
\begin{equation}
\begin{aligned}
& \hat{\mu} = \argmin_\mu  \lVert y-\mu \rVert_2^2,\\
&\hat{\sigma}^2 = \frac{1}{n} \lVert y-\hat{\mu}\rVert_2^2.
\label{eq:appen_mle}
\end{aligned}
\end{equation}
Using these expressions, we then have
\begin{equation}
\text{err}_\text{KL} = n \log(2\pi \hat{\sigma}^2) +n,
\label{eq:err_kl}
\end{equation}
and
\begin{equation*}
\text{Err}_\text{KL} = n\log(2\pi \hat{\sigma}^2) + n\frac{\sigma^2}{\hat{\sigma}^2} +\frac{\lVert \mu- \hat{\mu} \rVert_2^2}{\hat{\sigma}^2}.
\end{equation*}
The expected optimism is then
\begin{equation}
\begin{aligned}
E(\text{op}_\text{KL})  &= E(\text{Err}_\text{KL}) - E(\text{err}_\text{KL}),\\
&= E\left(n\frac{\sigma^2}{\hat{\sigma}^2}\right) + E\left(\frac{\lVert \mu- \hat{\mu}) \rVert_2^2}{\hat{\sigma}^2}\right) -n.
\end{aligned}
\label{eq:eop}
\end{equation}

So far we've been considering a subset with $k$ fixed predictors. At subset size $k$, BS chooses the one with minimum residual sum of squares (RSS) from all $\binom{p}{k}$ possible subsets. In order for the above derivation to continue to hold for BS of subset size $k$, we need to show that the MLE from \eqref{eq:appen_mle} is also the BS fit. This can be easily obtained from the full likelihood (-2 times) \eqref{eq:err_kl}, which after substituting the expression of $\hat{\sigma}$ leads to
\begin{equation*}
n\log\left(\frac{2\pi}{n}\lVert y-\hat{\mu}\rVert_2^2\right) + n.
\end{equation*}
Therefore, for all $\binom{p}{k}$ models of size $k$, the one with largest log likelihood, is also the one with smallest RSS. Hence \eqref{eq:eop} holds for BS fit with subset size $k$ as well.

\section{Proof of Theorem \ref{thm:correspondence}}
\label{sec:correspondence}

\begin{proof}	
	Since $[X_1,X_2,\cdots,X_j]$ and $[Q_1,Q_2,\cdots,Q_j]$ span the same space, we have
	\begin{equation}
	\hat{\alpha}^{(j)} = \hat{\beta}^{(j)}.
	\label{eq:thmproof-correspondence-zrq-zrx-subset}
	\end{equation}
	We can express $\hat{\gamma}(k_Q)$ as
	\begin{equation}
		\hat{\gamma}(k_Q) = \sum_{j\in S_k} \hat{\gamma}^{(j)} - \hat{\gamma}^{(j-1)}.
		\label{eq:zs_expand}
	\end{equation}
	We multiply both sides by $R^{-1}$ ($X$ is assumed to have full column rank), and use \eqref{eq:thmproof-correspondence-zrq-zrx-subset} to get
	\begin{equation*}
		\hat{\beta}(k_Q) = \sum_{j\in S_k} \hat{\alpha}^{(j)} - \hat{\alpha}^{(j-1)}.
		%\label{eq:thmproof-correspondence-conclusion}
	\end{equation*}
	\iffalse
	 \eqref{eq:thmproof-correspondence-conclusion} tells us that when certain subset $Q_S$ is chosen, the coefficients projected from the $Q$ space, correspond to a linear combination of multiple regression coefficients of $y$ upon subsets in $X$, where these subsets are sequential. For example, in the simple $2$-predictor case, if $Q_2$ is the chosen predictor, by \eqref{eq:thmproof-correspondence-conclusion}, we get:
	\begin{equation*}
	\hat{\beta}^{(Q_2)} = \hat{\beta}^{(X_1,X_2)} - \hat{\beta}^{(X_1)}.
	\end{equation*}
	Hence, it corresponds to the difference between two regression coefficients, the coefficients of $y$ upon $X_1,X_2$, and the coefficients of $y$ upon just $X_1$. 
	\fi
\end{proof}


\clearpage
\topmargin= -0.4in
\textheight = +8.9in
\oddsidemargin = 0.05in
\evensidemargin = 0.05in
\textwidth = 6.5in

\spacingset{1.2} % DON'T change the spacing!

\section{Tables}
\begin{itemize}
	\item Orthogonal $X$, simulation setups are discussed in Section \ref{sec:simulation_setup_orthx}. 
	\begin{itemize}
		\item The performance of selection rules for BS. The selection rules include C$_p$, AICc, BIC, GCV and 10-fold CV. For each selection rule except CV, there are two columns in the table indicating the degrees of freedoms to use in calculating the information criterion. The `edf' (effective degrees of freedom) is estimated using definition \eqref{eq:edf} by assuming the knowledge of $\mu$ and $\sigma$, and hence it is an infeasible rule. The `ndf/hdf/bdf' (naive degrees of freedom /  heuristic degrees of freedom / degrees of freedom based on bootstrap) are feasible selection rules in practice. 
		\begin{itemize}
			\item Orth-Sparse-Ex1: tables S1-S2
			\item Orth-Sparse-Ex2: tables S3-S4
			\item Orth-Dense: tables S5-S6
		\end{itemize}
		\item The performance of BS and regularization methods. Note that for lasso, we use the number of non-zero coefficients $k(\lambda)$ in place of edf in the AICc formula \eqref{eq:aicc_edf}. \citetonline{Zou2007} showed that $k(\lambda)$ is an unbiased estimator of edf for lasso. For gamma lasso, \citetonline{Taddy2017} suggested a heuristic degrees of freedom to be plugged into \eqref{eq:aicc_edf} in order to use AICc as the selection rule.
		\begin{itemize}
			\item Orth-Sparse-Ex1: tables S7-S8
			\item Orth-Sparse-Ex2: tables S9-S10
			\item Orth-Dense: tables S11-S12
		\end{itemize}
	\end{itemize}
	\item General $X$, simulation setups are discussed in Section \ref{sec:simulation_setup_generalx}. 
	\begin{itemize}
		\item The performance of BOSS, BS, FS, lasso, gamma lasso, SparseNet and relaxed lasso (rlasso).
		\begin{itemize}
			\item Sparse-Ex1: tables S13-S18
			\item Sparse-Ex2: tables S19-S24
			\item Sparse-Ex3: tables S25-S30
			\item Sparse-Ex4: tables S31-S36
			\item Dense: tables S37-S42
		\end{itemize}
	\end{itemize}
\end{itemize}

\clearpage

% latex table generated in R 3.6.2 by xtable 1.8-4 package
% Sat Dec 21 23:17:11 2019
\begin{table}[ht]
\centering
\caption{The performance of BS compared to regularization methods, Orth-Sparse-Ex1, n=200} 
\scalebox{0.75}{
\begin{tabular}{|c|c|ccccc|}
  \toprule 
 \multicolumn{1}{|c}{} &       & BS    & LASSO & Gamma LASSO & SparseNet & \multicolumn{1}{c|}{rLASSO}  \\
 \multicolumn{1}{|c}{} &       & AICc-hdf & AICc/CV & AICc/CV & CV    & \multicolumn{1}{c|}{CV}       \\
 \midrule\multicolumn{1}{|c}{} &       & \multicolumn{5}{c|}{\% worse than the best possible BS} \\
 \midrule\multirow{4}[2]{*}{hsnr} & p=14 & 6 & 42/41 & 16/19 & 13 & 16 \\ 
   & p=30 & 2 & 70/68 & 28/22 & 14 & 18 \\ 
   & p=60 & 2 & 94/92 & 53/25 & 17 & 21 \\ 
   & p=180 & 1 & 128/132 & 129/28 & 20 & 22 \\ 
  \midrule\multirow{4}[2]{*}{msnr} & p=14 & 11 & 43/41 & 21/22 & 13 & 16 \\ 
   & p=30 & 8 & 70/68 & 44/26 & 16 & 19 \\ 
   & p=60 & 8 & 95/92 & 82/30 & 18 & 22 \\ 
   & p=180 & 7 & 127/132 & 228/32 & 19 & 22 \\ 
  \midrule\multirow{4}[2]{*}{lsnr} & p=14 & 24 & 11/11 & 15/17 & 20 & 17 \\ 
   & p=30 & 37 & 7/7 & 16/13 & 15 & 13 \\ 
   & p=60 & 31 & 3/3 & 27/9 & 9 & 7 \\ 
   & p=180 & 22 & 1/5 & 99/8 & 8 & 8 \\ 
   \midrule 
 \multicolumn{1}{|c}{} &       & \multicolumn{5}{c|}{Relative efficiency} \\
\midrule\multirow{4}[2]{*}{hsnr} & p=14 & 1 & 0.74/0.75 & 0.92/0.89 & 0.94 & 0.92 \\ 
   & p=30 & 1 & 0.6/0.61 & 0.8/0.84 & 0.9 & 0.87 \\ 
   & p=60 & 1 & 0.52/0.53 & 0.66/0.81 & 0.87 & 0.84 \\ 
   & p=180 & 1 & 0.44/0.43 & 0.44/0.79 & 0.84 & 0.83 \\ 
  \midrule\multirow{4}[2]{*}{msnr} & p=14 & 1 & 0.78/0.79 & 0.92/0.91 & 0.98 & 0.96 \\ 
   & p=30 & 1 & 0.64/0.64 & 0.75/0.86 & 0.94 & 0.91 \\ 
   & p=60 & 1 & 0.56/0.56 & 0.6/0.83 & 0.92 & 0.89 \\ 
   & p=180 & 1 & 0.47/0.46 & 0.33/0.81 & 0.9 & 0.88 \\ 
  \midrule\multirow{4}[2]{*}{lsnr} & p=14 & 0.89 & 1/1 & 0.97/0.95 & 0.92 & 0.95 \\ 
   & p=30 & 0.78 & 1/1 & 0.92/0.95 & 0.93 & 0.94 \\ 
   & p=60 & 0.78 & 1/1 & 0.81/0.95 & 0.94 & 0.96 \\ 
   & p=180 & 0.83 & 1/0.96 & 0.51/0.93 & 0.94 & 0.93 \\ 
   \midrule 
 \multicolumn{1}{|c}{} &       & \multicolumn{5}{c|}{Sparsistency (number of extra variables)} \\
\midrule\multirow{4}[2]{*}{hsnr} & p=14 & 6(0.2) & 6(3.8)/6(4.5) & 6(0.9)/6(1.3) & 6(0.6) & 6(0.6) \\ 
   & p=30 & 6(0.1) & 6(7.7)/6(8.6) & 6(2.1)/6(1.6) & 6(1.1) & 6(0.6) \\ 
   & p=60 & 6(0) & 6(11.1)/6(12.7) & 6(5.3)/6(2) & 6(1.7) & 6(0.9) \\ 
   & p=180 & 6(0) & 6(14.3)/6(22.9) & 6(20.5)/6(3.3) & 6(3.7) & 6(1.4) \\ 
  \midrule\multirow{4}[2]{*}{msnr} & p=14 & 6(0.5) & 6(3.8)/6(4.5) & 6(1.1)/6(1.3) & 6(0.5) & 6(0.6) \\ 
   & p=30 & 6(0.2) & 6(7.7)/6(8.6) & 6(2.8)/6(1.4) & 6(0.8) & 6(0.7) \\ 
   & p=60 & 6(0.1) & 6(11.1)/6(12.6) & 6(6.5)/6(1.8) & 6(1.2) & 6(0.9) \\ 
   & p=180 & 6(0.1) & 6(14.3)/6(23) & 6(33.6)/6(2.7) & 6(2.4) & 6(1.4) \\ 
  \midrule\multirow{4}[2]{*}{lsnr} & p=14 & 5.4(4.2) & 5.7(3.5)/5.7(4.1) & 5.2(1.4)/5.3(2.9) & 5.3(2.7) & 5.1(2.1) \\ 
   & p=30 & 3.3(2.2) & 5.3(6.7)/5.4(7.1) & 5.2(3.9)/4.9(4.8) & 4.9(5.1) & 4.6(3.5) \\ 
   & p=60 & 1.5(0.2) & 4.9(9.2)/4.9(9.9) & 5.1(8.6)/4.4(6.7) & 4.4(7.4) & 4.2(5) \\ 
   & p=180 & 0.5(0) & 3.9(9.6)/4(14.8) & 5.5(45.9)/3.5(10.5) & 3.6(11.3) & 3.3(8.1) \\ 
   \bottomrule 
\end{tabular}
}
\end{table}

% latex table generated in R 3.6.1 by xtable 1.8-4 package
% Sun Nov 10 01:26:57 2019
\begin{table}[ht]
\centering
\caption{The performance of BS using different selection rules, Orth-Sparse-Ex1, n=2000} 
\scalebox{0.65}{
\begin{tabular}{|c|c|cc|cc|cc|cc|c|}
  \toprule 
 \multicolumn{1}{|c}{} &       & \multicolumn{2}{c|}{C$_p$} & \multicolumn{2}{c|}{AICc} & \multicolumn{2}{c|}{BIC} & \multicolumn{2}{c|}{GCV} & \multirow{2}[2]{*}{CV} \\
 \multicolumn{1}{|c}{} &       & edf   & ndf/hdf/bdf & edf   & ndf/hdf/bdf & edf   & ndf/hdf/bdf & edf   & ndf/hdf/bdf &       \\
 \cmidrule{3-11}\multicolumn{1}{|c}{} &       & \multicolumn{9}{c|}{\% worse than the best possible BS} \\
 \midrule\multirow{4}[2]{*}{hsnr} & p=14 & 8 & 33/7/9 & 8 & 33/7/9 & 0 & 3/0/0 & 8 & 33/7/9 & 18 \\ 
   & p=30 & 3 & 85/3/6 & 3 & 85/3/6 & 0 & 9/0/0 & 3 & 86/3/6 & 23 \\ 
   & p=60 & 2 & 155/2/4 & 2 & 156/2/4 & 0 & 21/0/0 & 2 & 156/2/4 & - \\ 
   & p=180 & 0 & 334/1/3 & 1 & 337/1/3 & 0 & 60/0/0 & 1 & 340/1/3 & - \\ 
  \midrule\multirow{4}[2]{*}{msnr} & p=14 & 8 & 33/7/9 & 8 & 33/7/9 & 0 & 3/0/0 & 8 & 33/7/9 & 18 \\ 
   & p=30 & 3 & 85/3/6 & 3 & 85/3/6 & 0 & 9/0/0 & 3 & 86/3/6 & 23 \\ 
   & p=60 & 2 & 155/2/4 & 2 & 156/2/4 & 0 & 21/0/0 & 2 & 156/2/4 & - \\ 
   & p=180 & 0 & 334/1/3 & 1 & 337/1/3 & 0 & 60/0/0 & 1 & 340/1/3 & - \\ 
  \midrule\multirow{4}[2]{*}{lsnr} & p=14 & 8 & 33/9/9 & 8 & 33/9/9 & 0 & 3/0/0 & 8 & 33/9/9 & 18 \\ 
   & p=30 & 3 & 85/6/7 & 3 & 85/5/6 & 0 & 9/0/0 & 3 & 86/6/7 & 23 \\ 
   & p=60 & 2 & 155/5/5 & 2 & 156/5/5 & 0 & 21/0/0 & 2 & 156/5/5 & - \\ 
   & p=180 & 0 & 334/5/4 & 1 & 337/4/4 & 0 & 60/1/1 & 1 & 340/5/4 & - \\ 
   \midrule 
 \multicolumn{1}{|c}{} &       & \multicolumn{9}{c|}{Relative efficiency} \\
\midrule\multirow{4}[2]{*}{hsnr} & p=14 & 0.93 & 0.75/0.94/0.92 & 0.93 & 0.75/0.94/0.92 & 1 & 0.97/1/1 & 0.92 & 0.75/0.94/0.92 & 0.84 \\ 
   & p=30 & 0.97 & 0.54/0.97/0.94 & 0.97 & 0.54/0.97/0.94 & 1 & 0.92/1/1 & 0.97 & 0.54/0.97/0.94 & 0.81 \\ 
   & p=60 & 0.98 & 0.39/0.98/0.96 & 0.98 & 0.39/0.98/0.96 & 1 & 0.83/1/1 & 0.98 & 0.39/0.98/0.96 & - \\ 
   & p=180 & 1 & 0.23/0.99/0.97 & 0.99 & 0.23/0.99/0.97 & 1 & 0.62/1/1 & 0.99 & 0.23/0.99/0.97 & - \\ 
  \midrule\multirow{4}[2]{*}{msnr} & p=14 & 0.93 & 0.75/0.94/0.92 & 0.93 & 0.75/0.94/0.92 & 1 & 0.97/1/1 & 0.92 & 0.75/0.94/0.92 & 0.84 \\ 
   & p=30 & 0.97 & 0.54/0.97/0.94 & 0.97 & 0.54/0.97/0.94 & 1 & 0.92/1/1 & 0.97 & 0.54/0.97/0.94 & 0.81 \\ 
   & p=60 & 0.98 & 0.39/0.98/0.96 & 0.98 & 0.39/0.98/0.96 & 1 & 0.83/1/1 & 0.98 & 0.39/0.98/0.96 & - \\ 
   & p=180 & 1 & 0.23/0.99/0.97 & 0.99 & 0.23/0.99/0.97 & 1 & 0.62/1/1 & 0.99 & 0.23/0.99/0.97 & - \\ 
  \midrule\multirow{4}[2]{*}{lsnr} & p=14 & 0.93 & 0.75/0.92/0.92 & 0.93 & 0.75/0.92/0.92 & 1 & 0.97/1/1 & 0.92 & 0.75/0.92/0.92 & 0.84 \\ 
   & p=30 & 0.97 & 0.54/0.95/0.94 & 0.97 & 0.54/0.95/0.94 & 1 & 0.92/1/1 & 0.97 & 0.54/0.95/0.94 & 0.81 \\ 
   & p=60 & 0.98 & 0.39/0.95/0.95 & 0.98 & 0.39/0.95/0.95 & 1 & 0.83/1/1 & 0.98 & 0.39/0.95/0.95 & - \\ 
   & p=180 & 1 & 0.23/0.96/0.96 & 0.99 & 0.23/0.96/0.96 & 1 & 0.62/0.99/0.99 & 0.99 & 0.23/0.96/0.96 & - \\ 
   \midrule 
 \multicolumn{1}{|c}{} &       & \multicolumn{9}{c|}{Sparsistency (number of extra variables)} \\
\midrule\multirow{4}[2]{*}{hsnr} & p=14 & 6(0.3) & 6(1.2)/6(0.3)/6(0.3) & 6(0.3) & 6(1.2)/6(0.3)/6(0.3) & 6(0) & 6(0)/6(0)/6(0) & 6(0.3) & 6(1.2)/6(0.3)/6(0.3) & 6(0.6) \\ 
   & p=30 & 6(0.1) & 6(3.8)/6(0.1)/6(0.2) & 6(0.1) & 6(3.8)/6(0.1)/6(0.2) & 6(0) & 6(0.1)/6(0)/6(0) & 6(0.1) & 6(3.9)/6(0.1)/6(0.2) & 6(0.6) \\ 
   & p=60 & 6(0) & 6(8.6)/6(0)/6(0) & 6(0) & 6(8.6)/6(0)/6(0) & 6(0) & 6(0.3)/6(0)/6(0) & 6(0) & 6(8.7)/6(0)/6(0) & - \\ 
   & p=180 & 6(0) & 6(27.5)/6(0)/6(0) & 6(0) & 6(28.2)/6(0)/6(0) & 6(0) & 6(1.1)/6(0)/6(0) & 6(0) & 6(28.9)/6(0)/6(0) & - \\ 
  \midrule\multirow{4}[2]{*}{msnr} & p=14 & 6(0.3) & 6(1.2)/6(0.3)/6(0.3) & 6(0.3) & 6(1.2)/6(0.3)/6(0.3) & 6(0) & 6(0)/6(0)/6(0) & 6(0.3) & 6(1.2)/6(0.3)/6(0.3) & 6(0.6) \\ 
   & p=30 & 6(0.1) & 6(3.8)/6(0.1)/6(0.2) & 6(0.1) & 6(3.8)/6(0.1)/6(0.2) & 6(0) & 6(0.1)/6(0)/6(0) & 6(0.1) & 6(3.9)/6(0.1)/6(0.2) & 6(0.6) \\ 
   & p=60 & 6(0) & 6(8.6)/6(0)/6(0) & 6(0) & 6(8.6)/6(0)/6(0) & 6(0) & 6(0.3)/6(0)/6(0) & 6(0) & 6(8.7)/6(0)/6(0) & - \\ 
   & p=180 & 6(0) & 6(27.5)/6(0)/6(0) & 6(0) & 6(28.2)/6(0)/6(0) & 6(0) & 6(1.1)/6(0)/6(0) & 6(0) & 6(28.9)/6(0)/6(0) & - \\ 
  \midrule\multirow{4}[2]{*}{lsnr} & p=14 & 6(0.3) & 6(1.2)/6(0.4)/6(0.3) & 6(0.3) & 6(1.2)/6(0.4)/6(0.3) & 6(0) & 6(0)/6(0)/6(0) & 6(0.3) & 6(1.2)/6(0.4)/6(0.3) & 6(0.6) \\ 
   & p=30 & 6(0.1) & 6(3.8)/6(0.2)/6(0.2) & 6(0.1) & 6(3.8)/6(0.2)/6(0.2) & 6(0) & 6(0.1)/6(0)/6(0) & 6(0.1) & 6(3.9)/6(0.2)/6(0.2) & 6(0.6) \\ 
   & p=60 & 6(0) & 6(8.6)/6(0.1)/6(0.1) & 6(0) & 6(8.6)/6(0.1)/6(0.1) & 6(0) & 6(0.3)/6(0)/6(0) & 6(0) & 6(8.7)/6(0.1)/6(0.1) & - \\ 
   & p=180 & 6(0) & 6(27.5)/6(0.1)/6(0) & 6(0) & 6(28.2)/6(0.1)/6(0) & 6(0) & 6(1.1)/6(0)/6(0) & 6(0) & 6(28.9)/6(0.1)/6(0) & - \\ 
   \bottomrule 
\end{tabular}
}
\end{table}

% latex table generated in R 3.6.1 by xtable 1.8-4 package
% Sun Nov 10 01:45:47 2019
\begin{table}[ht]
\centering
\caption{The performance of BS compared to regularization methods, Orth-Sparse-Ex2, n=200} 
\scalebox{0.75}{
\begin{tabular}{|c|c|ccccc|}
  \toprule 
 \multicolumn{1}{|c}{} &       & BS    & lasso & gamma lasso & SparseNet & \multicolumn{1}{c|}{rlasso}  \\
 \multicolumn{1}{|c}{} &       & AICc-hdf & AICc/CV & AICc/CV & CV    & \multicolumn{1}{c|}{CV}       \\
 \midrule\multicolumn{1}{|c}{} &       & \multicolumn{5}{c|}{\% worse than the best possible BS} \\
 \midrule\multirow{4}[2]{*}{hsnr} & p=14 & 32 & 23/22 & 16/19 & 22 & 22 \\ 
   & p=30 & 25 & 32/30 & 32/20 & 19 & 23 \\ 
   & p=60 & 18 & 39/38 & 59/20 & 16 & 24 \\ 
   & p=180 & 11 & 51/55 & 189/19 & 14 & 23 \\ 
  \midrule\multirow{4}[2]{*}{msnr} & p=14 & 21 & 32/31 & 25/19 & 17 & 20 \\ 
   & p=30 & 15 & 52/51 & 43/23 & 18 & 26 \\ 
   & p=60 & 14 & 70/69 & 73/27 & 22 & 29 \\ 
   & p=180 & 15 & 93/99 & 293/28 & 26 & 32 \\ 
  \midrule\multirow{4}[2]{*}{lsnr} & p=14 & 34 & 19/18 & 18/24 & 24 & 23 \\ 
   & p=30 & 34 & 21/21 & 30/26 & 25 & 24 \\ 
   & p=60 & 33 & 23/24 & 57/26 & 25 & 25 \\ 
   & p=180 & 39 & 24/30 & 167/26 & 25 & 27 \\ 
   \midrule 
 \multicolumn{1}{|c}{} &       & \multicolumn{5}{c|}{Relative efficiency} \\
\midrule\multirow{4}[2]{*}{hsnr} & p=14 & 0.88 & 0.95/0.95 & 1/0.98 & 0.95 & 0.95 \\ 
   & p=30 & 0.95 & 0.9/0.91 & 0.9/0.99 & 1 & 0.96 \\ 
   & p=60 & 0.99 & 0.84/0.84 & 0.73/0.97 & 1 & 0.94 \\ 
   & p=180 & 1 & 0.74/0.72 & 0.38/0.94 & 0.97 & 0.9 \\ 
  \midrule\multirow{4}[2]{*}{msnr} & p=14 & 0.97 & 0.89/0.89 & 0.93/0.98 & 1 & 0.97 \\ 
   & p=30 & 1 & 0.76/0.76 & 0.81/0.94 & 0.98 & 0.92 \\ 
   & p=60 & 1 & 0.67/0.67 & 0.66/0.9 & 0.94 & 0.89 \\ 
   & p=180 & 1 & 0.59/0.58 & 0.29/0.9 & 0.91 & 0.87 \\ 
  \midrule\multirow{4}[2]{*}{lsnr} & p=14 & 0.88 & 1/1 & 1/0.96 & 0.95 & 0.96 \\ 
   & p=30 & 0.9 & 1/1 & 0.93/0.96 & 0.97 & 0.98 \\ 
   & p=60 & 0.92 & 1/1 & 0.78/0.98 & 0.99 & 0.99 \\ 
   & p=180 & 0.9 & 1/0.96 & 0.47/0.99 & 0.99 & 0.98 \\ 
   \midrule 
 \multicolumn{1}{|c}{} &       & \multicolumn{5}{c|}{Sparsistency (number of extra variables)} \\
\midrule\multirow{4}[2]{*}{hsnr} & p=14 & 5(1.3) & 5.9(3.7)/5.9(4.4) & 5.6(1.3)/5.4(1.5) & 5.3(1.2) & 5.3(1.1) \\ 
   & p=30 & 4.5(0.2) & 5.7(7.2)/5.8(8.2) & 5.5(3.4)/5.1(2) & 5.1(2) & 5.1(1.5) \\ 
   & p=60 & 4.2(0.1) & 5.6(10.5)/5.7(11.8) & 5.5(7.8)/4.9(2.6) & 5(3) & 4.8(1.8) \\ 
   & p=180 & 4.1(0) & 5.4(12.9)/5.4(20.3) & 5.8(45.9)/4.6(3.7) & 4.7(5.3) & 4.5(2.3) \\ 
  \midrule\multirow{4}[2]{*}{msnr} & p=14 & 4.4(1) & 5.2(3.2)/5.3(3.8) & 4.7(1.1)/4.5(0.9) & 4.4(0.8) & 4.5(0.8) \\ 
   & p=30 & 4.1(0.3) & 5(6.4)/5(7.1) & 4.6(2.9)/4.3(1.2) & 4.3(1.1) & 4.3(1.1) \\ 
   & p=60 & 4(0.2) & 4.8(9.4)/4.8(10.4) & 4.6(6.7)/4.2(1.4) & 4.2(1.9) & 4.2(1.3) \\ 
   & p=180 & 3.9(0.1) & 4.5(12)/4.5(18.2) & 5(46.3)/4.1(2.1) & 4.1(3.7) & 4.1(1.8) \\ 
  \midrule\multirow{4}[2]{*}{lsnr} & p=14 & 3.7(2.1) & 4.3(2.7)/4.4(3) & 3.7(1.3)/3.5(1.4) & 3.5(1.4) & 3.4(1.1) \\ 
   & p=30 & 2.5(0.9) & 3.9(5.1)/3.9(5.4) & 3.7(3.4)/3(2.3) & 3.1(2.7) & 3(1.9) \\ 
   & p=60 & 1.8(0.2) & 3.6(7.7)/3.6(8) & 3.7(8)/2.8(3.6) & 3(4.7) & 2.8(2.9) \\ 
   & p=180 & 1.1(0.1) & 3(9.6)/3.1(13.7) & 4.1(43.1)/2.4(6.1) & 2.6(7.7) & 2.4(4.6) \\ 
   \bottomrule 
\end{tabular}
}
\end{table}

% latex table generated in R 3.6.2 by xtable 1.8-4 package
% Sat Dec 21 23:17:24 2019
\begin{table}[ht]
\centering
\caption{The performance of BS compared to regularization methods, Orth-Sparse-Ex2, n=2000} 
\scalebox{0.75}{
\begin{tabular}{|c|c|ccccc|}
  \toprule 
 \multicolumn{1}{|c}{} &       & BS    & LASSO & Gamma LASSO & SparseNet & \multicolumn{1}{c|}{rLASSO}  \\
 \multicolumn{1}{|c}{} &       & AICc-hdf & AICc/CV & AICc/CV & CV    & \multicolumn{1}{c|}{CV}       \\
 \midrule\multicolumn{1}{|c}{} &       & \multicolumn{5}{c|}{\% worse than the best possible BS} \\
 \midrule\multirow{4}[2]{*}{hsnr} & p=14 & 9 & 40/40 & 20/16 & 14 & 15 \\ 
   & p=30 & 5 & 71/68 & 45/21 & 15 & 19 \\ 
   & p=60 & 4 & 95/91 & 83/24 & 14 & 18 \\ 
   & p=180 & 4 & 129/125 & 195/23 & 14 & 15 \\ 
  \midrule\multirow{4}[2]{*}{msnr} & p=14 & 32 & 34/33 & 20/21 & 23 & 23 \\ 
   & p=30 & 39 & 52/50 & 44/28 & 29 & 31 \\ 
   & p=60 & 38 & 60/57 & 74/28 & 29 & 32 \\ 
   & p=180 & 37 & 67/64 & 155/26 & 25 & 31 \\ 
  \midrule\multirow{4}[2]{*}{lsnr} & p=14 & 19 & 27/26 & 24/18 & 16 & 17 \\ 
   & p=30 & 13 & 46/44 & 57/20 & 15 & 18 \\ 
   & p=60 & 10 & 63/60 & 102/21 & 15 & 18 \\ 
   & p=180 & 8 & 88/84 & 230/19 & 15 & 16 \\ 
   \midrule 
 \multicolumn{1}{|c}{} &       & \multicolumn{5}{c|}{Relative efficiency} \\
\midrule\multirow{4}[2]{*}{hsnr} & p=14 & 1 & 0.78/0.78 & 0.91/0.94 & 0.96 & 0.94 \\ 
   & p=30 & 1 & 0.62/0.63 & 0.73/0.87 & 0.91 & 0.89 \\ 
   & p=60 & 1 & 0.54/0.55 & 0.57/0.84 & 0.91 & 0.89 \\ 
   & p=180 & 1 & 0.45/0.46 & 0.35/0.84 & 0.91 & 0.9 \\ 
  \midrule\multirow{4}[2]{*}{msnr} & p=14 & 0.91 & 0.9/0.9 & 1/0.99 & 0.98 & 0.98 \\ 
   & p=30 & 0.92 & 0.85/0.86 & 0.89/1 & 1 & 0.98 \\ 
   & p=60 & 0.93 & 0.8/0.82 & 0.74/1 & 1 & 0.97 \\ 
   & p=180 & 0.91 & 0.75/0.76 & 0.49/0.99 & 1 & 0.95 \\ 
  \midrule\multirow{4}[2]{*}{lsnr} & p=14 & 0.97 & 0.91/0.92 & 0.94/0.98 & 1 & 0.99 \\ 
   & p=30 & 1 & 0.77/0.78 & 0.72/0.94 & 0.98 & 0.95 \\ 
   & p=60 & 1 & 0.67/0.69 & 0.54/0.91 & 0.95 & 0.93 \\ 
   & p=180 & 1 & 0.58/0.59 & 0.33/0.91 & 0.95 & 0.93 \\ 
   \midrule 
 \multicolumn{1}{|c}{} &       & \multicolumn{5}{c|}{Sparsistency (number of extra variables)} \\
\midrule\multirow{4}[2]{*}{hsnr} & p=14 & 6(0.4) & 6(3.8)/6(4.4) & 6(1.1)/6(0.9) & 6(0.6) & 6(0.6) \\ 
   & p=30 & 6(0.2) & 6(8.5)/6(8.7) & 6(3)/6(1.3) & 6(0.9) & 6(0.7) \\ 
   & p=60 & 6(0.1) & 6(13.2)/6(12.4) & 6(6.8)/6(1.5) & 6(1) & 6(0.6) \\ 
   & p=180 & 6(0) & 6(21.5)/6(18.7) & 6(24)/6(1.6) & 6(1.4) & 6(0.5) \\ 
  \midrule\multirow{4}[2]{*}{msnr} & p=14 & 5.8(1.7) & 6(3.8)/6(4.4) & 6(1.2)/5.9(1.5) & 5.9(1.1) & 5.8(0.9) \\ 
   & p=30 & 5.5(0.9) & 6(8.5)/6(8.6) & 6(3.6)/5.8(2.3) & 5.8(2.2) & 5.7(1.4) \\ 
   & p=60 & 5.2(0.3) & 6(13.1)/6(12.2) & 5.9(7.9)/5.7(2.5) & 5.7(2.8) & 5.6(1.5) \\ 
   & p=180 & 4.8(0.1) & 5.9(21.2)/5.9(18.3) & 6(27.1)/5.5(3.2) & 5.5(4.3) & 5.3(1.6) \\ 
  \midrule\multirow{4}[2]{*}{lsnr} & p=14 & 4.5(0.9) & 5.4(3.4)/5.4(3.9) & 4.9(1.3)/4.6(1.1) & 4.5(0.7) & 4.5(0.7) \\ 
   & p=30 & 4.2(0.4) & 5.1(7.3)/5.2(7.3) & 4.8(3.8)/4.4(1.2) & 4.3(1.2) & 4.3(1) \\ 
   & p=60 & 4.1(0.1) & 5(11.1)/4.9(10) & 4.8(8.1)/4.2(1.2) & 4.2(1.3) & 4.2(0.9) \\ 
   & p=180 & 4(0.1) & 4.7(18.1)/4.6(15) & 4.9(27.3)/4.1(1) & 4.1(1.5) & 4.1(0.8) \\ 
   \bottomrule 
\end{tabular}
}
\end{table}

% latex table generated in R 3.6.1 by xtable 1.8-4 package
% Sun Nov 10 01:27:06 2019
\begin{table}[ht]
\centering
\caption{The performance of BS using different selection rules, Orth-Dense, n=200} 
\scalebox{0.75}{
\begin{tabular}{|c|c|cc|cc|cc|cc|c|}
  \toprule 
 \multicolumn{1}{|c}{} &       & \multicolumn{2}{c|}{C$_p$} & \multicolumn{2}{c|}{AICc} & \multicolumn{2}{c|}{BIC} & \multicolumn{2}{c|}{GCV} & \multirow{2}[2]{*}{CV} \\
 \multicolumn{1}{|c}{} &       & edf   & ndf/hdf/bdf & edf   & ndf/hdf/bdf & edf   & ndf/hdf/bdf & edf   & ndf/hdf/bdf &       \\
 \cmidrule{3-11}\multicolumn{1}{|c}{} &       & \multicolumn{9}{c|}{\% worse than the best possible BS} \\
 \midrule\multirow{4}[2]{*}{hsnr} & p=14 & 0 & 0/0/0 & 0 & 0/0/0 & 0 & 1/0/0 & 0 & 0/0/0 & 0 \\ 
   & p=30 & 1 & 11/1/2 & 1 & 13/1/2 & 1 & 28/3/5 & 1 & 11/1/2 & 7 \\ 
   & p=60 & 8 & 7/9/9 & 9 & 7/11/11 & 20 & 8/32/33 & 8 & 8/10/10 & - \\ 
   & p=180 & 7 & 45/21/20 & 9 & 52/18/19 & 18 & 26/39/42 & 7 & 64/13/13 & - \\ 
  \midrule\multirow{4}[2]{*}{msnr} & p=14 & 0 & 9/0/1 & 0 & 10/0/1 & 0 & 36/1/2 & 0 & 9/0/1 & 6 \\ 
   & p=30 & 3 & 10/3/4 & 3 & 11/4/5 & 21 & 27/19/25 & 3 & 10/4/4 & 11 \\ 
   & p=60 & 10 & 11/14/13 & 10 & 11/13/13 & 26 & 10/48/48 & 10 & 12/14/13 & - \\ 
   & p=180 & 8 & 52/23/23 & 10 & 62/18/19 & 21 & 25/61/56 & 8 & 74/14/14 & - \\ 
  \midrule\multirow{4}[2]{*}{lsnr} & p=14 & 5 & 22/6/8 & 7 & 23/8/10 & 73 & 50/73/72 & 6 & 22/7/8 & 19 \\ 
   & p=30 & 15 & 10/16/16 & 20 & 10/21/20 & 27 & 16/27/27 & 17 & 10/18/18 & 16 \\ 
   & p=60 & 13 & 25/17/16 & 13 & 25/13/13 & 13 & 11/13/13 & 13 & 26/14/14 & - \\ 
   & p=180 & 8 & 86/22/22 & 7 & 102/7/7 & 7 & 39/7/7 & 7 & 116/7/7 & - \\ 
   \midrule 
 \multicolumn{1}{|c}{} &       & \multicolumn{9}{c|}{Relative efficiency} \\
\midrule\multirow{4}[2]{*}{hsnr} & p=14 & 1 & 1/1/1 & 1 & 1/1/1 & 1 & 0.99/1/1 & 1 & 1/1/1 & 1 \\ 
   & p=30 & 1 & 0.91/1/1 & 1 & 0.9/1/0.99 & 1 & 0.79/0.98/0.96 & 1 & 0.91/1/1 & 0.95 \\ 
   & p=60 & 0.99 & 1/0.98/0.98 & 0.98 & 1/0.97/0.96 & 0.89 & 0.99/0.81/0.8 & 0.99 & 0.99/0.98/0.98 & - \\ 
   & p=180 & 1 & 0.74/0.89/0.89 & 0.99 & 0.71/0.91/0.9 & 0.91 & 0.85/0.77/0.76 & 1 & 0.65/0.95/0.95 & - \\ 
  \midrule\multirow{4}[2]{*}{msnr} & p=14 & 1 & 0.92/1/0.99 & 1 & 0.91/1/0.99 & 1 & 0.74/1/0.99 & 1 & 0.92/1/0.99 & 0.95 \\ 
   & p=30 & 1 & 0.93/0.99/0.99 & 0.99 & 0.92/0.98/0.98 & 0.85 & 0.81/0.87/0.82 & 1 & 0.93/0.99/0.99 & 0.93 \\ 
   & p=60 & 1 & 0.99/0.96/0.97 & 1 & 0.99/0.97/0.97 & 0.87 & 1/0.74/0.74 & 1 & 0.98/0.97/0.97 & - \\ 
   & p=180 & 1 & 0.71/0.88/0.88 & 0.98 & 0.67/0.91/0.91 & 0.89 & 0.87/0.67/0.69 & 1 & 0.62/0.95/0.95 & - \\ 
  \midrule\multirow{4}[2]{*}{lsnr} & p=14 & 0.98 & 0.85/0.97/0.96 & 0.97 & 0.84/0.96/0.94 & 0.6 & 0.69/0.6/0.6 & 0.98 & 0.85/0.97/0.95 & 0.86 \\ 
   & p=30 & 0.95 & 1/0.95/0.95 & 0.91 & 1/0.91/0.91 & 0.86 & 0.94/0.86/0.86 & 0.93 & 1/0.93/0.93 & 0.94 \\ 
   & p=60 & 0.98 & 0.89/0.95/0.96 & 0.99 & 0.89/0.99/0.99 & 0.98 & 1/0.98/0.98 & 0.98 & 0.88/0.97/0.98 & - \\ 
   & p=180 & 1 & 0.58/0.88/0.88 & 1 & 0.53/1/1 & 1 & 0.77/1/1 & 1 & 0.5/1/1 & - \\ 
   \midrule 
 \multicolumn{1}{|c}{} &       & \multicolumn{9}{c|}{Sparsistency (number of extra variables)} \\
\midrule\multirow{4}[2]{*}{hsnr} & p=14 & 14 & 14/14/14 & 14 & 14/14/14 & 14 & 14/14/14 & 14 & 14/14/14 & 14 \\ 
   & p=30 & 30 & 24.7/29.5/29 & 30 & 24.2/29.4/28.8 & 30 & 20.9/28.8/27.5 & 30 & 24.7/29.5/29 & 26.6 \\ 
   & p=60 & 29.8 & 30.5/38.4/35.8 & 22.2 & 29.4/25.6/24.5 & 17.8 & 22.5/16.8/16.5 & 28.6 & 31.3/36.8/34 & - \\ 
   & p=180 & 20.5 & 53.3/37.4/35.5 & 18.3 & 62.3/16.3/16.3 & 16.1 & 35/13.7/13.5 & 19.4 & 89.8/17.8/17.8 & - \\ 
  \midrule\multirow{4}[2]{*}{msnr} & p=14 & 14 & 13.2/14/13.9 & 14 & 13.2/14/13.9 & 14 & 11.8/13.9/13.8 & 14 & 13.2/14/13.9 & 13.4 \\ 
   & p=30 & 27.3 & 18.8/27.4/26.1 & 26.5 & 18.3/26.8/25.3 & 18 & 13.4/20.4/17.6 & 27.3 & 18.8/27.4/26.1 & 20.8 \\ 
   & p=60 & 19.4 & 24.1/29.6/27 & 13.9 & 23.4/15.6/15.2 & 9.3 & 14.5/7.5/7.4 & 18.3 & 25.2/26/24.1 & - \\ 
   & p=180 & 12.6 & 47.1/29.1/28.1 & 10.4 & 59/8.8/8.8 & 8.1 & 24.4/4.8/5 & 11.3 & 86.4/10/10 & - \\ 
  \midrule\multirow{4}[2]{*}{lsnr} & p=14 & 13.6 & 7.7/12.7/11.7 & 13.4 & 7.6/12.3/11.3 & 0.7 & 3.6/0.7/0.8 & 13.5 & 7.8/12.6/11.6 & 8.8 \\ 
   & p=30 & 12.8 & 10.5/14.6/13 & 7.6 & 10.3/8.5/7.6 & 0 & 4/0/0 & 11.3 & 10.8/12.3/11.2 & 7.5 \\ 
   & p=60 & 3.4 & 15.7/6.5/6 & 1 & 15.8/0.8/1 & 0 & 4.9/0/0 & 2 & 17.3/2.4/2.4 & - \\ 
   & p=180 & 0.8 & 39/14.5/13.7 & 0.3 & 55.2/0.2/0.3 & 0 & 11.8/0/0 & 0.4 & 81.7/0.3/0.4 & - \\ 
   \bottomrule 
\end{tabular}
}
\end{table}

% latex table generated in R 3.6.1 by xtable 1.8-4 package
% Sun Nov 10 01:27:13 2019
\begin{table}[ht]
\centering
\caption{The performance of BS using different selection rules, Orth-Dense, n=2000} 
\scalebox{0.75}{
\begin{tabular}{|c|c|cc|cc|cc|cc|c|}
  \toprule 
 \multicolumn{1}{|c}{} &       & \multicolumn{2}{c|}{C$_p$} & \multicolumn{2}{c|}{AICc} & \multicolumn{2}{c|}{BIC} & \multicolumn{2}{c|}{GCV} & \multirow{2}[2]{*}{CV} \\
 \multicolumn{1}{|c}{} &       & edf   & ndf/hdf/bdf & edf   & ndf/hdf/bdf & edf   & ndf/hdf/bdf & edf   & ndf/hdf/bdf &       \\
 \cmidrule{3-11}\multicolumn{1}{|c}{} &       & \multicolumn{9}{c|}{\% worse than the best possible BS} \\
 \midrule\multirow{4}[2]{*}{hsnr} & p=14 & 0 & 0/0/0 & 0 & 0/0/0 & 0 & 0/0/0 & 0 & 0/0/0 & 0 \\ 
   & p=30 & 0 & 1/0/0 & 0 & 1/0/0 & 0 & 18/0/1 & 0 & 1/0/0 & 1 \\ 
   & p=60 & 5 & 5/5/5 & 5 & 5/5/5 & 25 & 17/34/37 & 5 & 5/5/5 & - \\ 
   & p=180 & 6 & 34/8/8 & 6 & 34/8/8 & 19 & 7/36/37 & 6 & 35/8/8 & - \\ 
  \midrule\multirow{4}[2]{*}{msnr} & p=14 & 0 & 0/0/0 & 0 & 0/0/0 & 0 & 0/0/0 & 0 & 0/0/0 & 0 \\ 
   & p=30 & 1 & 9/1/1 & 1 & 9/1/1 & 1 & 40/2/5 & 1 & 9/1/1 & 5 \\ 
   & p=60 & 7 & 6/8/8 & 7 & 6/8/8 & 28 & 15/37/40 & 7 & 6/8/8 & - \\ 
   & p=180 & 6 & 39/9/9 & 6 & 39/8/9 & 21 & 7/38/40 & 6 & 40/8/9 & - \\ 
  \midrule\multirow{4}[2]{*}{lsnr} & p=14 & 0 & 5/0/0 & 0 & 5/0/0 & 0 & 49/0/1 & 0 & 5/0/0 & 4 \\ 
   & p=30 & 2 & 11/3/3 & 2 & 11/3/3 & 44 & 41/36/45 & 2 & 11/3/3 & 10 \\ 
   & p=60 & 10 & 10/13/12 & 10 & 10/13/12 & 32 & 16/45/48 & 10 & 10/13/12 & - \\ 
   & p=180 & 8 & 48/10/10 & 8 & 48/10/10 & 24 & 8/45/47 & 8 & 49/10/10 & - \\ 
   \midrule 
 \multicolumn{1}{|c}{} &       & \multicolumn{9}{c|}{Relative efficiency} \\
\midrule\multirow{4}[2]{*}{hsnr} & p=14 & 1 & 1/1/1 & 1 & 1/1/1 & 1 & 1/1/1 & 1 & 1/1/1 & 1 \\ 
   & p=30 & 1 & 0.99/1/1 & 1 & 0.99/1/1 & 1 & 0.85/1/0.99 & 1 & 0.99/1/1 & 0.99 \\ 
   & p=60 & 0.99 & 1/0.99/0.99 & 0.99 & 1/0.99/0.99 & 0.83 & 0.89/0.78/0.76 & 0.99 & 1/0.99/0.99 & - \\ 
   & p=180 & 1 & 0.79/0.98/0.98 & 1 & 0.79/0.98/0.98 & 0.89 & 1/0.78/0.78 & 1 & 0.79/0.98/0.98 & - \\ 
  \midrule\multirow{4}[2]{*}{msnr} & p=14 & 1 & 1/1/1 & 1 & 1/1/1 & 1 & 1/1/1 & 1 & 1/1/1 & 1 \\ 
   & p=30 & 1 & 0.92/1/1 & 1 & 0.92/1/1 & 1 & 0.72/0.99/0.96 & 1 & 0.92/1/1 & 0.96 \\ 
   & p=60 & 0.99 & 1/0.98/0.98 & 0.99 & 1/0.98/0.98 & 0.83 & 0.92/0.77/0.76 & 0.99 & 1/0.98/0.98 & - \\ 
   & p=180 & 1 & 0.76/0.98/0.98 & 1 & 0.76/0.98/0.98 & 0.88 & 1/0.77/0.76 & 1 & 0.76/0.98/0.98 & - \\ 
  \midrule\multirow{4}[2]{*}{lsnr} & p=14 & 1 & 0.95/1/1 & 1 & 0.95/1/1 & 1 & 0.67/1/0.99 & 1 & 0.95/1/1 & 0.96 \\ 
   & p=30 & 1 & 0.92/0.99/0.99 & 1 & 0.92/0.99/0.99 & 0.71 & 0.73/0.75/0.7 & 1 & 0.92/0.99/0.99 & 0.93 \\ 
   & p=60 & 1 & 1/0.97/0.98 & 1 & 1/0.97/0.98 & 0.83 & 0.94/0.75/0.74 & 1 & 1/0.97/0.98 & - \\ 
   & p=180 & 1 & 0.73/0.98/0.98 & 1 & 0.73/0.98/0.98 & 0.87 & 1/0.74/0.73 & 1 & 0.72/0.98/0.98 & - \\ 
   \midrule 
 \multicolumn{1}{|c}{} &       & \multicolumn{9}{c|}{Sparsistency (number of extra variables)} \\
\midrule\multirow{4}[2]{*}{hsnr} & p=14 & 14 & 14/14/14 & 14 & 14/14/14 & 14 & 14/14/14 & 14 & 14/14/14 & 14 \\ 
   & p=30 & 30 & 29.8/30/29.9 & 30 & 29.8/30/29.9 & 30 & 28.6/30/29.9 & 30 & 29.8/30/29.9 & 29.8 \\ 
   & p=60 & 44.9 & 39.8/50.9/48.7 & 44.2 & 39.7/50.4/48.2 & 28.5 & 30.5/27.6/27.4 & 45.1 & 39.8/50.8/48.6 & - \\ 
   & p=180 & 32.1 & 58.9/32.4/32.3 & 31.8 & 58.9/31.6/31.6 & 27 & 31.3/25/24.9 & 32 & 59.9/32.1/32 & - \\ 
  \midrule\multirow{4}[2]{*}{msnr} & p=14 & 14 & 14/14/14 & 14 & 14/14/14 & 14 & 14/14/14 & 14 & 14/14/14 & 14 \\ 
   & p=30 & 30 & 27.1/29.9/29.6 & 30 & 27.1/29.9/29.6 & 30 & 22.5/29.5/28.6 & 30 & 27.1/29.9/29.6 & 28.2 \\ 
   & p=60 & 34.8 & 33.3/42.8/40.1 & 33.9 & 33.2/41.9/39.2 & 20.4 & 22.9/19.4/19.2 & 34.6 & 33.3/42.8/40 & - \\ 
   & p=180 & 24.2 & 52.4/24.4/24.3 & 24 & 52.6/23.6/23.6 & 19.2 & 23.6/17.3/17.2 & 24.1 & 53.5/24.1/24.1 & - \\ 
  \midrule\multirow{4}[2]{*}{lsnr} & p=14 & 14 & 13.6/14/13.9 & 14 & 13.6/14/13.9 & 14 & 11.7/14/13.9 & 14 & 13.6/14/13.9 & 13.7 \\ 
   & p=30 & 28.8 & 19.9/28.2/26.9 & 28.8 & 19.9/28.1/26.8 & 13.5 & 12.5/16.7/14.1 & 28.8 & 19.9/28.2/26.9 & 22.3 \\ 
   & p=60 & 21.6 & 24.8/30.6/27.9 & 20.9 & 24.7/29.2/26.9 & 10 & 12.7/8.7/8.5 & 21.8 & 24.9/30.4/27.7 & - \\ 
   & p=180 & 13.9 & 43.8/14/14 & 13.6 & 44.1/13.3/13.3 & 9.1 & 13.4/7/6.8 & 13.8 & 45/13.7/13.6 & - \\ 
   \bottomrule 
\end{tabular}
}
\end{table}



% latex table generated in R 3.6.2 by xtable 1.8-4 package
% Sat Dec 21 23:17:11 2019
\begin{table}[ht]
\centering
\caption{The performance of BS compared to regularization methods, Orth-Sparse-Ex1, n=200} 
\scalebox{0.75}{
\begin{tabular}{|c|c|ccccc|}
  \toprule 
 \multicolumn{1}{|c}{} &       & BS    & LASSO & Gamma LASSO & SparseNet & \multicolumn{1}{c|}{rLASSO}  \\
 \multicolumn{1}{|c}{} &       & AICc-hdf & AICc/CV & AICc/CV & CV    & \multicolumn{1}{c|}{CV}       \\
 \midrule\multicolumn{1}{|c}{} &       & \multicolumn{5}{c|}{\% worse than the best possible BS} \\
 \midrule\multirow{4}[2]{*}{hsnr} & p=14 & 6 & 42/41 & 16/19 & 13 & 16 \\ 
   & p=30 & 2 & 70/68 & 28/22 & 14 & 18 \\ 
   & p=60 & 2 & 94/92 & 53/25 & 17 & 21 \\ 
   & p=180 & 1 & 128/132 & 129/28 & 20 & 22 \\ 
  \midrule\multirow{4}[2]{*}{msnr} & p=14 & 11 & 43/41 & 21/22 & 13 & 16 \\ 
   & p=30 & 8 & 70/68 & 44/26 & 16 & 19 \\ 
   & p=60 & 8 & 95/92 & 82/30 & 18 & 22 \\ 
   & p=180 & 7 & 127/132 & 228/32 & 19 & 22 \\ 
  \midrule\multirow{4}[2]{*}{lsnr} & p=14 & 24 & 11/11 & 15/17 & 20 & 17 \\ 
   & p=30 & 37 & 7/7 & 16/13 & 15 & 13 \\ 
   & p=60 & 31 & 3/3 & 27/9 & 9 & 7 \\ 
   & p=180 & 22 & 1/5 & 99/8 & 8 & 8 \\ 
   \midrule 
 \multicolumn{1}{|c}{} &       & \multicolumn{5}{c|}{Relative efficiency} \\
\midrule\multirow{4}[2]{*}{hsnr} & p=14 & 1 & 0.74/0.75 & 0.92/0.89 & 0.94 & 0.92 \\ 
   & p=30 & 1 & 0.6/0.61 & 0.8/0.84 & 0.9 & 0.87 \\ 
   & p=60 & 1 & 0.52/0.53 & 0.66/0.81 & 0.87 & 0.84 \\ 
   & p=180 & 1 & 0.44/0.43 & 0.44/0.79 & 0.84 & 0.83 \\ 
  \midrule\multirow{4}[2]{*}{msnr} & p=14 & 1 & 0.78/0.79 & 0.92/0.91 & 0.98 & 0.96 \\ 
   & p=30 & 1 & 0.64/0.64 & 0.75/0.86 & 0.94 & 0.91 \\ 
   & p=60 & 1 & 0.56/0.56 & 0.6/0.83 & 0.92 & 0.89 \\ 
   & p=180 & 1 & 0.47/0.46 & 0.33/0.81 & 0.9 & 0.88 \\ 
  \midrule\multirow{4}[2]{*}{lsnr} & p=14 & 0.89 & 1/1 & 0.97/0.95 & 0.92 & 0.95 \\ 
   & p=30 & 0.78 & 1/1 & 0.92/0.95 & 0.93 & 0.94 \\ 
   & p=60 & 0.78 & 1/1 & 0.81/0.95 & 0.94 & 0.96 \\ 
   & p=180 & 0.83 & 1/0.96 & 0.51/0.93 & 0.94 & 0.93 \\ 
   \midrule 
 \multicolumn{1}{|c}{} &       & \multicolumn{5}{c|}{Sparsistency (number of extra variables)} \\
\midrule\multirow{4}[2]{*}{hsnr} & p=14 & 6(0.2) & 6(3.8)/6(4.5) & 6(0.9)/6(1.3) & 6(0.6) & 6(0.6) \\ 
   & p=30 & 6(0.1) & 6(7.7)/6(8.6) & 6(2.1)/6(1.6) & 6(1.1) & 6(0.6) \\ 
   & p=60 & 6(0) & 6(11.1)/6(12.7) & 6(5.3)/6(2) & 6(1.7) & 6(0.9) \\ 
   & p=180 & 6(0) & 6(14.3)/6(22.9) & 6(20.5)/6(3.3) & 6(3.7) & 6(1.4) \\ 
  \midrule\multirow{4}[2]{*}{msnr} & p=14 & 6(0.5) & 6(3.8)/6(4.5) & 6(1.1)/6(1.3) & 6(0.5) & 6(0.6) \\ 
   & p=30 & 6(0.2) & 6(7.7)/6(8.6) & 6(2.8)/6(1.4) & 6(0.8) & 6(0.7) \\ 
   & p=60 & 6(0.1) & 6(11.1)/6(12.6) & 6(6.5)/6(1.8) & 6(1.2) & 6(0.9) \\ 
   & p=180 & 6(0.1) & 6(14.3)/6(23) & 6(33.6)/6(2.7) & 6(2.4) & 6(1.4) \\ 
  \midrule\multirow{4}[2]{*}{lsnr} & p=14 & 5.4(4.2) & 5.7(3.5)/5.7(4.1) & 5.2(1.4)/5.3(2.9) & 5.3(2.7) & 5.1(2.1) \\ 
   & p=30 & 3.3(2.2) & 5.3(6.7)/5.4(7.1) & 5.2(3.9)/4.9(4.8) & 4.9(5.1) & 4.6(3.5) \\ 
   & p=60 & 1.5(0.2) & 4.9(9.2)/4.9(9.9) & 5.1(8.6)/4.4(6.7) & 4.4(7.4) & 4.2(5) \\ 
   & p=180 & 0.5(0) & 3.9(9.6)/4(14.8) & 5.5(45.9)/3.5(10.5) & 3.6(11.3) & 3.3(8.1) \\ 
   \bottomrule 
\end{tabular}
}
\end{table}

% latex table generated in R 3.6.1 by xtable 1.8-4 package
% Sun Nov 10 01:26:57 2019
\begin{table}[ht]
\centering
\caption{The performance of BS using different selection rules, Orth-Sparse-Ex1, n=2000} 
\scalebox{0.65}{
\begin{tabular}{|c|c|cc|cc|cc|cc|c|}
  \toprule 
 \multicolumn{1}{|c}{} &       & \multicolumn{2}{c|}{C$_p$} & \multicolumn{2}{c|}{AICc} & \multicolumn{2}{c|}{BIC} & \multicolumn{2}{c|}{GCV} & \multirow{2}[2]{*}{CV} \\
 \multicolumn{1}{|c}{} &       & edf   & ndf/hdf/bdf & edf   & ndf/hdf/bdf & edf   & ndf/hdf/bdf & edf   & ndf/hdf/bdf &       \\
 \cmidrule{3-11}\multicolumn{1}{|c}{} &       & \multicolumn{9}{c|}{\% worse than the best possible BS} \\
 \midrule\multirow{4}[2]{*}{hsnr} & p=14 & 8 & 33/7/9 & 8 & 33/7/9 & 0 & 3/0/0 & 8 & 33/7/9 & 18 \\ 
   & p=30 & 3 & 85/3/6 & 3 & 85/3/6 & 0 & 9/0/0 & 3 & 86/3/6 & 23 \\ 
   & p=60 & 2 & 155/2/4 & 2 & 156/2/4 & 0 & 21/0/0 & 2 & 156/2/4 & - \\ 
   & p=180 & 0 & 334/1/3 & 1 & 337/1/3 & 0 & 60/0/0 & 1 & 340/1/3 & - \\ 
  \midrule\multirow{4}[2]{*}{msnr} & p=14 & 8 & 33/7/9 & 8 & 33/7/9 & 0 & 3/0/0 & 8 & 33/7/9 & 18 \\ 
   & p=30 & 3 & 85/3/6 & 3 & 85/3/6 & 0 & 9/0/0 & 3 & 86/3/6 & 23 \\ 
   & p=60 & 2 & 155/2/4 & 2 & 156/2/4 & 0 & 21/0/0 & 2 & 156/2/4 & - \\ 
   & p=180 & 0 & 334/1/3 & 1 & 337/1/3 & 0 & 60/0/0 & 1 & 340/1/3 & - \\ 
  \midrule\multirow{4}[2]{*}{lsnr} & p=14 & 8 & 33/9/9 & 8 & 33/9/9 & 0 & 3/0/0 & 8 & 33/9/9 & 18 \\ 
   & p=30 & 3 & 85/6/7 & 3 & 85/5/6 & 0 & 9/0/0 & 3 & 86/6/7 & 23 \\ 
   & p=60 & 2 & 155/5/5 & 2 & 156/5/5 & 0 & 21/0/0 & 2 & 156/5/5 & - \\ 
   & p=180 & 0 & 334/5/4 & 1 & 337/4/4 & 0 & 60/1/1 & 1 & 340/5/4 & - \\ 
   \midrule 
 \multicolumn{1}{|c}{} &       & \multicolumn{9}{c|}{Relative efficiency} \\
\midrule\multirow{4}[2]{*}{hsnr} & p=14 & 0.93 & 0.75/0.94/0.92 & 0.93 & 0.75/0.94/0.92 & 1 & 0.97/1/1 & 0.92 & 0.75/0.94/0.92 & 0.84 \\ 
   & p=30 & 0.97 & 0.54/0.97/0.94 & 0.97 & 0.54/0.97/0.94 & 1 & 0.92/1/1 & 0.97 & 0.54/0.97/0.94 & 0.81 \\ 
   & p=60 & 0.98 & 0.39/0.98/0.96 & 0.98 & 0.39/0.98/0.96 & 1 & 0.83/1/1 & 0.98 & 0.39/0.98/0.96 & - \\ 
   & p=180 & 1 & 0.23/0.99/0.97 & 0.99 & 0.23/0.99/0.97 & 1 & 0.62/1/1 & 0.99 & 0.23/0.99/0.97 & - \\ 
  \midrule\multirow{4}[2]{*}{msnr} & p=14 & 0.93 & 0.75/0.94/0.92 & 0.93 & 0.75/0.94/0.92 & 1 & 0.97/1/1 & 0.92 & 0.75/0.94/0.92 & 0.84 \\ 
   & p=30 & 0.97 & 0.54/0.97/0.94 & 0.97 & 0.54/0.97/0.94 & 1 & 0.92/1/1 & 0.97 & 0.54/0.97/0.94 & 0.81 \\ 
   & p=60 & 0.98 & 0.39/0.98/0.96 & 0.98 & 0.39/0.98/0.96 & 1 & 0.83/1/1 & 0.98 & 0.39/0.98/0.96 & - \\ 
   & p=180 & 1 & 0.23/0.99/0.97 & 0.99 & 0.23/0.99/0.97 & 1 & 0.62/1/1 & 0.99 & 0.23/0.99/0.97 & - \\ 
  \midrule\multirow{4}[2]{*}{lsnr} & p=14 & 0.93 & 0.75/0.92/0.92 & 0.93 & 0.75/0.92/0.92 & 1 & 0.97/1/1 & 0.92 & 0.75/0.92/0.92 & 0.84 \\ 
   & p=30 & 0.97 & 0.54/0.95/0.94 & 0.97 & 0.54/0.95/0.94 & 1 & 0.92/1/1 & 0.97 & 0.54/0.95/0.94 & 0.81 \\ 
   & p=60 & 0.98 & 0.39/0.95/0.95 & 0.98 & 0.39/0.95/0.95 & 1 & 0.83/1/1 & 0.98 & 0.39/0.95/0.95 & - \\ 
   & p=180 & 1 & 0.23/0.96/0.96 & 0.99 & 0.23/0.96/0.96 & 1 & 0.62/0.99/0.99 & 0.99 & 0.23/0.96/0.96 & - \\ 
   \midrule 
 \multicolumn{1}{|c}{} &       & \multicolumn{9}{c|}{Sparsistency (number of extra variables)} \\
\midrule\multirow{4}[2]{*}{hsnr} & p=14 & 6(0.3) & 6(1.2)/6(0.3)/6(0.3) & 6(0.3) & 6(1.2)/6(0.3)/6(0.3) & 6(0) & 6(0)/6(0)/6(0) & 6(0.3) & 6(1.2)/6(0.3)/6(0.3) & 6(0.6) \\ 
   & p=30 & 6(0.1) & 6(3.8)/6(0.1)/6(0.2) & 6(0.1) & 6(3.8)/6(0.1)/6(0.2) & 6(0) & 6(0.1)/6(0)/6(0) & 6(0.1) & 6(3.9)/6(0.1)/6(0.2) & 6(0.6) \\ 
   & p=60 & 6(0) & 6(8.6)/6(0)/6(0) & 6(0) & 6(8.6)/6(0)/6(0) & 6(0) & 6(0.3)/6(0)/6(0) & 6(0) & 6(8.7)/6(0)/6(0) & - \\ 
   & p=180 & 6(0) & 6(27.5)/6(0)/6(0) & 6(0) & 6(28.2)/6(0)/6(0) & 6(0) & 6(1.1)/6(0)/6(0) & 6(0) & 6(28.9)/6(0)/6(0) & - \\ 
  \midrule\multirow{4}[2]{*}{msnr} & p=14 & 6(0.3) & 6(1.2)/6(0.3)/6(0.3) & 6(0.3) & 6(1.2)/6(0.3)/6(0.3) & 6(0) & 6(0)/6(0)/6(0) & 6(0.3) & 6(1.2)/6(0.3)/6(0.3) & 6(0.6) \\ 
   & p=30 & 6(0.1) & 6(3.8)/6(0.1)/6(0.2) & 6(0.1) & 6(3.8)/6(0.1)/6(0.2) & 6(0) & 6(0.1)/6(0)/6(0) & 6(0.1) & 6(3.9)/6(0.1)/6(0.2) & 6(0.6) \\ 
   & p=60 & 6(0) & 6(8.6)/6(0)/6(0) & 6(0) & 6(8.6)/6(0)/6(0) & 6(0) & 6(0.3)/6(0)/6(0) & 6(0) & 6(8.7)/6(0)/6(0) & - \\ 
   & p=180 & 6(0) & 6(27.5)/6(0)/6(0) & 6(0) & 6(28.2)/6(0)/6(0) & 6(0) & 6(1.1)/6(0)/6(0) & 6(0) & 6(28.9)/6(0)/6(0) & - \\ 
  \midrule\multirow{4}[2]{*}{lsnr} & p=14 & 6(0.3) & 6(1.2)/6(0.4)/6(0.3) & 6(0.3) & 6(1.2)/6(0.4)/6(0.3) & 6(0) & 6(0)/6(0)/6(0) & 6(0.3) & 6(1.2)/6(0.4)/6(0.3) & 6(0.6) \\ 
   & p=30 & 6(0.1) & 6(3.8)/6(0.2)/6(0.2) & 6(0.1) & 6(3.8)/6(0.2)/6(0.2) & 6(0) & 6(0.1)/6(0)/6(0) & 6(0.1) & 6(3.9)/6(0.2)/6(0.2) & 6(0.6) \\ 
   & p=60 & 6(0) & 6(8.6)/6(0.1)/6(0.1) & 6(0) & 6(8.6)/6(0.1)/6(0.1) & 6(0) & 6(0.3)/6(0)/6(0) & 6(0) & 6(8.7)/6(0.1)/6(0.1) & - \\ 
   & p=180 & 6(0) & 6(27.5)/6(0.1)/6(0) & 6(0) & 6(28.2)/6(0.1)/6(0) & 6(0) & 6(1.1)/6(0)/6(0) & 6(0) & 6(28.9)/6(0.1)/6(0) & - \\ 
   \bottomrule 
\end{tabular}
}
\end{table}

% latex table generated in R 3.6.1 by xtable 1.8-4 package
% Sun Nov 10 01:45:47 2019
\begin{table}[ht]
\centering
\caption{The performance of BS compared to regularization methods, Orth-Sparse-Ex2, n=200} 
\scalebox{0.75}{
\begin{tabular}{|c|c|ccccc|}
  \toprule 
 \multicolumn{1}{|c}{} &       & BS    & lasso & gamma lasso & SparseNet & \multicolumn{1}{c|}{rlasso}  \\
 \multicolumn{1}{|c}{} &       & AICc-hdf & AICc/CV & AICc/CV & CV    & \multicolumn{1}{c|}{CV}       \\
 \midrule\multicolumn{1}{|c}{} &       & \multicolumn{5}{c|}{\% worse than the best possible BS} \\
 \midrule\multirow{4}[2]{*}{hsnr} & p=14 & 32 & 23/22 & 16/19 & 22 & 22 \\ 
   & p=30 & 25 & 32/30 & 32/20 & 19 & 23 \\ 
   & p=60 & 18 & 39/38 & 59/20 & 16 & 24 \\ 
   & p=180 & 11 & 51/55 & 189/19 & 14 & 23 \\ 
  \midrule\multirow{4}[2]{*}{msnr} & p=14 & 21 & 32/31 & 25/19 & 17 & 20 \\ 
   & p=30 & 15 & 52/51 & 43/23 & 18 & 26 \\ 
   & p=60 & 14 & 70/69 & 73/27 & 22 & 29 \\ 
   & p=180 & 15 & 93/99 & 293/28 & 26 & 32 \\ 
  \midrule\multirow{4}[2]{*}{lsnr} & p=14 & 34 & 19/18 & 18/24 & 24 & 23 \\ 
   & p=30 & 34 & 21/21 & 30/26 & 25 & 24 \\ 
   & p=60 & 33 & 23/24 & 57/26 & 25 & 25 \\ 
   & p=180 & 39 & 24/30 & 167/26 & 25 & 27 \\ 
   \midrule 
 \multicolumn{1}{|c}{} &       & \multicolumn{5}{c|}{Relative efficiency} \\
\midrule\multirow{4}[2]{*}{hsnr} & p=14 & 0.88 & 0.95/0.95 & 1/0.98 & 0.95 & 0.95 \\ 
   & p=30 & 0.95 & 0.9/0.91 & 0.9/0.99 & 1 & 0.96 \\ 
   & p=60 & 0.99 & 0.84/0.84 & 0.73/0.97 & 1 & 0.94 \\ 
   & p=180 & 1 & 0.74/0.72 & 0.38/0.94 & 0.97 & 0.9 \\ 
  \midrule\multirow{4}[2]{*}{msnr} & p=14 & 0.97 & 0.89/0.89 & 0.93/0.98 & 1 & 0.97 \\ 
   & p=30 & 1 & 0.76/0.76 & 0.81/0.94 & 0.98 & 0.92 \\ 
   & p=60 & 1 & 0.67/0.67 & 0.66/0.9 & 0.94 & 0.89 \\ 
   & p=180 & 1 & 0.59/0.58 & 0.29/0.9 & 0.91 & 0.87 \\ 
  \midrule\multirow{4}[2]{*}{lsnr} & p=14 & 0.88 & 1/1 & 1/0.96 & 0.95 & 0.96 \\ 
   & p=30 & 0.9 & 1/1 & 0.93/0.96 & 0.97 & 0.98 \\ 
   & p=60 & 0.92 & 1/1 & 0.78/0.98 & 0.99 & 0.99 \\ 
   & p=180 & 0.9 & 1/0.96 & 0.47/0.99 & 0.99 & 0.98 \\ 
   \midrule 
 \multicolumn{1}{|c}{} &       & \multicolumn{5}{c|}{Sparsistency (number of extra variables)} \\
\midrule\multirow{4}[2]{*}{hsnr} & p=14 & 5(1.3) & 5.9(3.7)/5.9(4.4) & 5.6(1.3)/5.4(1.5) & 5.3(1.2) & 5.3(1.1) \\ 
   & p=30 & 4.5(0.2) & 5.7(7.2)/5.8(8.2) & 5.5(3.4)/5.1(2) & 5.1(2) & 5.1(1.5) \\ 
   & p=60 & 4.2(0.1) & 5.6(10.5)/5.7(11.8) & 5.5(7.8)/4.9(2.6) & 5(3) & 4.8(1.8) \\ 
   & p=180 & 4.1(0) & 5.4(12.9)/5.4(20.3) & 5.8(45.9)/4.6(3.7) & 4.7(5.3) & 4.5(2.3) \\ 
  \midrule\multirow{4}[2]{*}{msnr} & p=14 & 4.4(1) & 5.2(3.2)/5.3(3.8) & 4.7(1.1)/4.5(0.9) & 4.4(0.8) & 4.5(0.8) \\ 
   & p=30 & 4.1(0.3) & 5(6.4)/5(7.1) & 4.6(2.9)/4.3(1.2) & 4.3(1.1) & 4.3(1.1) \\ 
   & p=60 & 4(0.2) & 4.8(9.4)/4.8(10.4) & 4.6(6.7)/4.2(1.4) & 4.2(1.9) & 4.2(1.3) \\ 
   & p=180 & 3.9(0.1) & 4.5(12)/4.5(18.2) & 5(46.3)/4.1(2.1) & 4.1(3.7) & 4.1(1.8) \\ 
  \midrule\multirow{4}[2]{*}{lsnr} & p=14 & 3.7(2.1) & 4.3(2.7)/4.4(3) & 3.7(1.3)/3.5(1.4) & 3.5(1.4) & 3.4(1.1) \\ 
   & p=30 & 2.5(0.9) & 3.9(5.1)/3.9(5.4) & 3.7(3.4)/3(2.3) & 3.1(2.7) & 3(1.9) \\ 
   & p=60 & 1.8(0.2) & 3.6(7.7)/3.6(8) & 3.7(8)/2.8(3.6) & 3(4.7) & 2.8(2.9) \\ 
   & p=180 & 1.1(0.1) & 3(9.6)/3.1(13.7) & 4.1(43.1)/2.4(6.1) & 2.6(7.7) & 2.4(4.6) \\ 
   \bottomrule 
\end{tabular}
}
\end{table}

% latex table generated in R 3.6.2 by xtable 1.8-4 package
% Sat Dec 21 23:17:24 2019
\begin{table}[ht]
\centering
\caption{The performance of BS compared to regularization methods, Orth-Sparse-Ex2, n=2000} 
\scalebox{0.75}{
\begin{tabular}{|c|c|ccccc|}
  \toprule 
 \multicolumn{1}{|c}{} &       & BS    & LASSO & Gamma LASSO & SparseNet & \multicolumn{1}{c|}{rLASSO}  \\
 \multicolumn{1}{|c}{} &       & AICc-hdf & AICc/CV & AICc/CV & CV    & \multicolumn{1}{c|}{CV}       \\
 \midrule\multicolumn{1}{|c}{} &       & \multicolumn{5}{c|}{\% worse than the best possible BS} \\
 \midrule\multirow{4}[2]{*}{hsnr} & p=14 & 9 & 40/40 & 20/16 & 14 & 15 \\ 
   & p=30 & 5 & 71/68 & 45/21 & 15 & 19 \\ 
   & p=60 & 4 & 95/91 & 83/24 & 14 & 18 \\ 
   & p=180 & 4 & 129/125 & 195/23 & 14 & 15 \\ 
  \midrule\multirow{4}[2]{*}{msnr} & p=14 & 32 & 34/33 & 20/21 & 23 & 23 \\ 
   & p=30 & 39 & 52/50 & 44/28 & 29 & 31 \\ 
   & p=60 & 38 & 60/57 & 74/28 & 29 & 32 \\ 
   & p=180 & 37 & 67/64 & 155/26 & 25 & 31 \\ 
  \midrule\multirow{4}[2]{*}{lsnr} & p=14 & 19 & 27/26 & 24/18 & 16 & 17 \\ 
   & p=30 & 13 & 46/44 & 57/20 & 15 & 18 \\ 
   & p=60 & 10 & 63/60 & 102/21 & 15 & 18 \\ 
   & p=180 & 8 & 88/84 & 230/19 & 15 & 16 \\ 
   \midrule 
 \multicolumn{1}{|c}{} &       & \multicolumn{5}{c|}{Relative efficiency} \\
\midrule\multirow{4}[2]{*}{hsnr} & p=14 & 1 & 0.78/0.78 & 0.91/0.94 & 0.96 & 0.94 \\ 
   & p=30 & 1 & 0.62/0.63 & 0.73/0.87 & 0.91 & 0.89 \\ 
   & p=60 & 1 & 0.54/0.55 & 0.57/0.84 & 0.91 & 0.89 \\ 
   & p=180 & 1 & 0.45/0.46 & 0.35/0.84 & 0.91 & 0.9 \\ 
  \midrule\multirow{4}[2]{*}{msnr} & p=14 & 0.91 & 0.9/0.9 & 1/0.99 & 0.98 & 0.98 \\ 
   & p=30 & 0.92 & 0.85/0.86 & 0.89/1 & 1 & 0.98 \\ 
   & p=60 & 0.93 & 0.8/0.82 & 0.74/1 & 1 & 0.97 \\ 
   & p=180 & 0.91 & 0.75/0.76 & 0.49/0.99 & 1 & 0.95 \\ 
  \midrule\multirow{4}[2]{*}{lsnr} & p=14 & 0.97 & 0.91/0.92 & 0.94/0.98 & 1 & 0.99 \\ 
   & p=30 & 1 & 0.77/0.78 & 0.72/0.94 & 0.98 & 0.95 \\ 
   & p=60 & 1 & 0.67/0.69 & 0.54/0.91 & 0.95 & 0.93 \\ 
   & p=180 & 1 & 0.58/0.59 & 0.33/0.91 & 0.95 & 0.93 \\ 
   \midrule 
 \multicolumn{1}{|c}{} &       & \multicolumn{5}{c|}{Sparsistency (number of extra variables)} \\
\midrule\multirow{4}[2]{*}{hsnr} & p=14 & 6(0.4) & 6(3.8)/6(4.4) & 6(1.1)/6(0.9) & 6(0.6) & 6(0.6) \\ 
   & p=30 & 6(0.2) & 6(8.5)/6(8.7) & 6(3)/6(1.3) & 6(0.9) & 6(0.7) \\ 
   & p=60 & 6(0.1) & 6(13.2)/6(12.4) & 6(6.8)/6(1.5) & 6(1) & 6(0.6) \\ 
   & p=180 & 6(0) & 6(21.5)/6(18.7) & 6(24)/6(1.6) & 6(1.4) & 6(0.5) \\ 
  \midrule\multirow{4}[2]{*}{msnr} & p=14 & 5.8(1.7) & 6(3.8)/6(4.4) & 6(1.2)/5.9(1.5) & 5.9(1.1) & 5.8(0.9) \\ 
   & p=30 & 5.5(0.9) & 6(8.5)/6(8.6) & 6(3.6)/5.8(2.3) & 5.8(2.2) & 5.7(1.4) \\ 
   & p=60 & 5.2(0.3) & 6(13.1)/6(12.2) & 5.9(7.9)/5.7(2.5) & 5.7(2.8) & 5.6(1.5) \\ 
   & p=180 & 4.8(0.1) & 5.9(21.2)/5.9(18.3) & 6(27.1)/5.5(3.2) & 5.5(4.3) & 5.3(1.6) \\ 
  \midrule\multirow{4}[2]{*}{lsnr} & p=14 & 4.5(0.9) & 5.4(3.4)/5.4(3.9) & 4.9(1.3)/4.6(1.1) & 4.5(0.7) & 4.5(0.7) \\ 
   & p=30 & 4.2(0.4) & 5.1(7.3)/5.2(7.3) & 4.8(3.8)/4.4(1.2) & 4.3(1.2) & 4.3(1) \\ 
   & p=60 & 4.1(0.1) & 5(11.1)/4.9(10) & 4.8(8.1)/4.2(1.2) & 4.2(1.3) & 4.2(0.9) \\ 
   & p=180 & 4(0.1) & 4.7(18.1)/4.6(15) & 4.9(27.3)/4.1(1) & 4.1(1.5) & 4.1(0.8) \\ 
   \bottomrule 
\end{tabular}
}
\end{table}

% latex table generated in R 3.6.1 by xtable 1.8-4 package
% Sun Nov 10 01:27:06 2019
\begin{table}[ht]
\centering
\caption{The performance of BS using different selection rules, Orth-Dense, n=200} 
\scalebox{0.75}{
\begin{tabular}{|c|c|cc|cc|cc|cc|c|}
  \toprule 
 \multicolumn{1}{|c}{} &       & \multicolumn{2}{c|}{C$_p$} & \multicolumn{2}{c|}{AICc} & \multicolumn{2}{c|}{BIC} & \multicolumn{2}{c|}{GCV} & \multirow{2}[2]{*}{CV} \\
 \multicolumn{1}{|c}{} &       & edf   & ndf/hdf/bdf & edf   & ndf/hdf/bdf & edf   & ndf/hdf/bdf & edf   & ndf/hdf/bdf &       \\
 \cmidrule{3-11}\multicolumn{1}{|c}{} &       & \multicolumn{9}{c|}{\% worse than the best possible BS} \\
 \midrule\multirow{4}[2]{*}{hsnr} & p=14 & 0 & 0/0/0 & 0 & 0/0/0 & 0 & 1/0/0 & 0 & 0/0/0 & 0 \\ 
   & p=30 & 1 & 11/1/2 & 1 & 13/1/2 & 1 & 28/3/5 & 1 & 11/1/2 & 7 \\ 
   & p=60 & 8 & 7/9/9 & 9 & 7/11/11 & 20 & 8/32/33 & 8 & 8/10/10 & - \\ 
   & p=180 & 7 & 45/21/20 & 9 & 52/18/19 & 18 & 26/39/42 & 7 & 64/13/13 & - \\ 
  \midrule\multirow{4}[2]{*}{msnr} & p=14 & 0 & 9/0/1 & 0 & 10/0/1 & 0 & 36/1/2 & 0 & 9/0/1 & 6 \\ 
   & p=30 & 3 & 10/3/4 & 3 & 11/4/5 & 21 & 27/19/25 & 3 & 10/4/4 & 11 \\ 
   & p=60 & 10 & 11/14/13 & 10 & 11/13/13 & 26 & 10/48/48 & 10 & 12/14/13 & - \\ 
   & p=180 & 8 & 52/23/23 & 10 & 62/18/19 & 21 & 25/61/56 & 8 & 74/14/14 & - \\ 
  \midrule\multirow{4}[2]{*}{lsnr} & p=14 & 5 & 22/6/8 & 7 & 23/8/10 & 73 & 50/73/72 & 6 & 22/7/8 & 19 \\ 
   & p=30 & 15 & 10/16/16 & 20 & 10/21/20 & 27 & 16/27/27 & 17 & 10/18/18 & 16 \\ 
   & p=60 & 13 & 25/17/16 & 13 & 25/13/13 & 13 & 11/13/13 & 13 & 26/14/14 & - \\ 
   & p=180 & 8 & 86/22/22 & 7 & 102/7/7 & 7 & 39/7/7 & 7 & 116/7/7 & - \\ 
   \midrule 
 \multicolumn{1}{|c}{} &       & \multicolumn{9}{c|}{Relative efficiency} \\
\midrule\multirow{4}[2]{*}{hsnr} & p=14 & 1 & 1/1/1 & 1 & 1/1/1 & 1 & 0.99/1/1 & 1 & 1/1/1 & 1 \\ 
   & p=30 & 1 & 0.91/1/1 & 1 & 0.9/1/0.99 & 1 & 0.79/0.98/0.96 & 1 & 0.91/1/1 & 0.95 \\ 
   & p=60 & 0.99 & 1/0.98/0.98 & 0.98 & 1/0.97/0.96 & 0.89 & 0.99/0.81/0.8 & 0.99 & 0.99/0.98/0.98 & - \\ 
   & p=180 & 1 & 0.74/0.89/0.89 & 0.99 & 0.71/0.91/0.9 & 0.91 & 0.85/0.77/0.76 & 1 & 0.65/0.95/0.95 & - \\ 
  \midrule\multirow{4}[2]{*}{msnr} & p=14 & 1 & 0.92/1/0.99 & 1 & 0.91/1/0.99 & 1 & 0.74/1/0.99 & 1 & 0.92/1/0.99 & 0.95 \\ 
   & p=30 & 1 & 0.93/0.99/0.99 & 0.99 & 0.92/0.98/0.98 & 0.85 & 0.81/0.87/0.82 & 1 & 0.93/0.99/0.99 & 0.93 \\ 
   & p=60 & 1 & 0.99/0.96/0.97 & 1 & 0.99/0.97/0.97 & 0.87 & 1/0.74/0.74 & 1 & 0.98/0.97/0.97 & - \\ 
   & p=180 & 1 & 0.71/0.88/0.88 & 0.98 & 0.67/0.91/0.91 & 0.89 & 0.87/0.67/0.69 & 1 & 0.62/0.95/0.95 & - \\ 
  \midrule\multirow{4}[2]{*}{lsnr} & p=14 & 0.98 & 0.85/0.97/0.96 & 0.97 & 0.84/0.96/0.94 & 0.6 & 0.69/0.6/0.6 & 0.98 & 0.85/0.97/0.95 & 0.86 \\ 
   & p=30 & 0.95 & 1/0.95/0.95 & 0.91 & 1/0.91/0.91 & 0.86 & 0.94/0.86/0.86 & 0.93 & 1/0.93/0.93 & 0.94 \\ 
   & p=60 & 0.98 & 0.89/0.95/0.96 & 0.99 & 0.89/0.99/0.99 & 0.98 & 1/0.98/0.98 & 0.98 & 0.88/0.97/0.98 & - \\ 
   & p=180 & 1 & 0.58/0.88/0.88 & 1 & 0.53/1/1 & 1 & 0.77/1/1 & 1 & 0.5/1/1 & - \\ 
   \midrule 
 \multicolumn{1}{|c}{} &       & \multicolumn{9}{c|}{Sparsistency (number of extra variables)} \\
\midrule\multirow{4}[2]{*}{hsnr} & p=14 & 14 & 14/14/14 & 14 & 14/14/14 & 14 & 14/14/14 & 14 & 14/14/14 & 14 \\ 
   & p=30 & 30 & 24.7/29.5/29 & 30 & 24.2/29.4/28.8 & 30 & 20.9/28.8/27.5 & 30 & 24.7/29.5/29 & 26.6 \\ 
   & p=60 & 29.8 & 30.5/38.4/35.8 & 22.2 & 29.4/25.6/24.5 & 17.8 & 22.5/16.8/16.5 & 28.6 & 31.3/36.8/34 & - \\ 
   & p=180 & 20.5 & 53.3/37.4/35.5 & 18.3 & 62.3/16.3/16.3 & 16.1 & 35/13.7/13.5 & 19.4 & 89.8/17.8/17.8 & - \\ 
  \midrule\multirow{4}[2]{*}{msnr} & p=14 & 14 & 13.2/14/13.9 & 14 & 13.2/14/13.9 & 14 & 11.8/13.9/13.8 & 14 & 13.2/14/13.9 & 13.4 \\ 
   & p=30 & 27.3 & 18.8/27.4/26.1 & 26.5 & 18.3/26.8/25.3 & 18 & 13.4/20.4/17.6 & 27.3 & 18.8/27.4/26.1 & 20.8 \\ 
   & p=60 & 19.4 & 24.1/29.6/27 & 13.9 & 23.4/15.6/15.2 & 9.3 & 14.5/7.5/7.4 & 18.3 & 25.2/26/24.1 & - \\ 
   & p=180 & 12.6 & 47.1/29.1/28.1 & 10.4 & 59/8.8/8.8 & 8.1 & 24.4/4.8/5 & 11.3 & 86.4/10/10 & - \\ 
  \midrule\multirow{4}[2]{*}{lsnr} & p=14 & 13.6 & 7.7/12.7/11.7 & 13.4 & 7.6/12.3/11.3 & 0.7 & 3.6/0.7/0.8 & 13.5 & 7.8/12.6/11.6 & 8.8 \\ 
   & p=30 & 12.8 & 10.5/14.6/13 & 7.6 & 10.3/8.5/7.6 & 0 & 4/0/0 & 11.3 & 10.8/12.3/11.2 & 7.5 \\ 
   & p=60 & 3.4 & 15.7/6.5/6 & 1 & 15.8/0.8/1 & 0 & 4.9/0/0 & 2 & 17.3/2.4/2.4 & - \\ 
   & p=180 & 0.8 & 39/14.5/13.7 & 0.3 & 55.2/0.2/0.3 & 0 & 11.8/0/0 & 0.4 & 81.7/0.3/0.4 & - \\ 
   \bottomrule 
\end{tabular}
}
\end{table}

% latex table generated in R 3.6.1 by xtable 1.8-4 package
% Sun Nov 10 01:27:13 2019
\begin{table}[ht]
\centering
\caption{The performance of BS using different selection rules, Orth-Dense, n=2000} 
\scalebox{0.75}{
\begin{tabular}{|c|c|cc|cc|cc|cc|c|}
  \toprule 
 \multicolumn{1}{|c}{} &       & \multicolumn{2}{c|}{C$_p$} & \multicolumn{2}{c|}{AICc} & \multicolumn{2}{c|}{BIC} & \multicolumn{2}{c|}{GCV} & \multirow{2}[2]{*}{CV} \\
 \multicolumn{1}{|c}{} &       & edf   & ndf/hdf/bdf & edf   & ndf/hdf/bdf & edf   & ndf/hdf/bdf & edf   & ndf/hdf/bdf &       \\
 \cmidrule{3-11}\multicolumn{1}{|c}{} &       & \multicolumn{9}{c|}{\% worse than the best possible BS} \\
 \midrule\multirow{4}[2]{*}{hsnr} & p=14 & 0 & 0/0/0 & 0 & 0/0/0 & 0 & 0/0/0 & 0 & 0/0/0 & 0 \\ 
   & p=30 & 0 & 1/0/0 & 0 & 1/0/0 & 0 & 18/0/1 & 0 & 1/0/0 & 1 \\ 
   & p=60 & 5 & 5/5/5 & 5 & 5/5/5 & 25 & 17/34/37 & 5 & 5/5/5 & - \\ 
   & p=180 & 6 & 34/8/8 & 6 & 34/8/8 & 19 & 7/36/37 & 6 & 35/8/8 & - \\ 
  \midrule\multirow{4}[2]{*}{msnr} & p=14 & 0 & 0/0/0 & 0 & 0/0/0 & 0 & 0/0/0 & 0 & 0/0/0 & 0 \\ 
   & p=30 & 1 & 9/1/1 & 1 & 9/1/1 & 1 & 40/2/5 & 1 & 9/1/1 & 5 \\ 
   & p=60 & 7 & 6/8/8 & 7 & 6/8/8 & 28 & 15/37/40 & 7 & 6/8/8 & - \\ 
   & p=180 & 6 & 39/9/9 & 6 & 39/8/9 & 21 & 7/38/40 & 6 & 40/8/9 & - \\ 
  \midrule\multirow{4}[2]{*}{lsnr} & p=14 & 0 & 5/0/0 & 0 & 5/0/0 & 0 & 49/0/1 & 0 & 5/0/0 & 4 \\ 
   & p=30 & 2 & 11/3/3 & 2 & 11/3/3 & 44 & 41/36/45 & 2 & 11/3/3 & 10 \\ 
   & p=60 & 10 & 10/13/12 & 10 & 10/13/12 & 32 & 16/45/48 & 10 & 10/13/12 & - \\ 
   & p=180 & 8 & 48/10/10 & 8 & 48/10/10 & 24 & 8/45/47 & 8 & 49/10/10 & - \\ 
   \midrule 
 \multicolumn{1}{|c}{} &       & \multicolumn{9}{c|}{Relative efficiency} \\
\midrule\multirow{4}[2]{*}{hsnr} & p=14 & 1 & 1/1/1 & 1 & 1/1/1 & 1 & 1/1/1 & 1 & 1/1/1 & 1 \\ 
   & p=30 & 1 & 0.99/1/1 & 1 & 0.99/1/1 & 1 & 0.85/1/0.99 & 1 & 0.99/1/1 & 0.99 \\ 
   & p=60 & 0.99 & 1/0.99/0.99 & 0.99 & 1/0.99/0.99 & 0.83 & 0.89/0.78/0.76 & 0.99 & 1/0.99/0.99 & - \\ 
   & p=180 & 1 & 0.79/0.98/0.98 & 1 & 0.79/0.98/0.98 & 0.89 & 1/0.78/0.78 & 1 & 0.79/0.98/0.98 & - \\ 
  \midrule\multirow{4}[2]{*}{msnr} & p=14 & 1 & 1/1/1 & 1 & 1/1/1 & 1 & 1/1/1 & 1 & 1/1/1 & 1 \\ 
   & p=30 & 1 & 0.92/1/1 & 1 & 0.92/1/1 & 1 & 0.72/0.99/0.96 & 1 & 0.92/1/1 & 0.96 \\ 
   & p=60 & 0.99 & 1/0.98/0.98 & 0.99 & 1/0.98/0.98 & 0.83 & 0.92/0.77/0.76 & 0.99 & 1/0.98/0.98 & - \\ 
   & p=180 & 1 & 0.76/0.98/0.98 & 1 & 0.76/0.98/0.98 & 0.88 & 1/0.77/0.76 & 1 & 0.76/0.98/0.98 & - \\ 
  \midrule\multirow{4}[2]{*}{lsnr} & p=14 & 1 & 0.95/1/1 & 1 & 0.95/1/1 & 1 & 0.67/1/0.99 & 1 & 0.95/1/1 & 0.96 \\ 
   & p=30 & 1 & 0.92/0.99/0.99 & 1 & 0.92/0.99/0.99 & 0.71 & 0.73/0.75/0.7 & 1 & 0.92/0.99/0.99 & 0.93 \\ 
   & p=60 & 1 & 1/0.97/0.98 & 1 & 1/0.97/0.98 & 0.83 & 0.94/0.75/0.74 & 1 & 1/0.97/0.98 & - \\ 
   & p=180 & 1 & 0.73/0.98/0.98 & 1 & 0.73/0.98/0.98 & 0.87 & 1/0.74/0.73 & 1 & 0.72/0.98/0.98 & - \\ 
   \midrule 
 \multicolumn{1}{|c}{} &       & \multicolumn{9}{c|}{Sparsistency (number of extra variables)} \\
\midrule\multirow{4}[2]{*}{hsnr} & p=14 & 14 & 14/14/14 & 14 & 14/14/14 & 14 & 14/14/14 & 14 & 14/14/14 & 14 \\ 
   & p=30 & 30 & 29.8/30/29.9 & 30 & 29.8/30/29.9 & 30 & 28.6/30/29.9 & 30 & 29.8/30/29.9 & 29.8 \\ 
   & p=60 & 44.9 & 39.8/50.9/48.7 & 44.2 & 39.7/50.4/48.2 & 28.5 & 30.5/27.6/27.4 & 45.1 & 39.8/50.8/48.6 & - \\ 
   & p=180 & 32.1 & 58.9/32.4/32.3 & 31.8 & 58.9/31.6/31.6 & 27 & 31.3/25/24.9 & 32 & 59.9/32.1/32 & - \\ 
  \midrule\multirow{4}[2]{*}{msnr} & p=14 & 14 & 14/14/14 & 14 & 14/14/14 & 14 & 14/14/14 & 14 & 14/14/14 & 14 \\ 
   & p=30 & 30 & 27.1/29.9/29.6 & 30 & 27.1/29.9/29.6 & 30 & 22.5/29.5/28.6 & 30 & 27.1/29.9/29.6 & 28.2 \\ 
   & p=60 & 34.8 & 33.3/42.8/40.1 & 33.9 & 33.2/41.9/39.2 & 20.4 & 22.9/19.4/19.2 & 34.6 & 33.3/42.8/40 & - \\ 
   & p=180 & 24.2 & 52.4/24.4/24.3 & 24 & 52.6/23.6/23.6 & 19.2 & 23.6/17.3/17.2 & 24.1 & 53.5/24.1/24.1 & - \\ 
  \midrule\multirow{4}[2]{*}{lsnr} & p=14 & 14 & 13.6/14/13.9 & 14 & 13.6/14/13.9 & 14 & 11.7/14/13.9 & 14 & 13.6/14/13.9 & 13.7 \\ 
   & p=30 & 28.8 & 19.9/28.2/26.9 & 28.8 & 19.9/28.1/26.8 & 13.5 & 12.5/16.7/14.1 & 28.8 & 19.9/28.2/26.9 & 22.3 \\ 
   & p=60 & 21.6 & 24.8/30.6/27.9 & 20.9 & 24.7/29.2/26.9 & 10 & 12.7/8.7/8.5 & 21.8 & 24.9/30.4/27.7 & - \\ 
   & p=180 & 13.9 & 43.8/14/14 & 13.6 & 44.1/13.3/13.3 & 9.1 & 13.4/7/6.8 & 13.8 & 45/13.7/13.6 & - \\ 
   \bottomrule 
\end{tabular}
}
\end{table}


% latex table generated in R 3.6.2 by xtable 1.8-4 package
% Sat Dec 21 23:35:10 2019
\begin{table}[ht]
\centering
\caption{The performance of BOSS compared to other methods, Sparse-Ex1, $\rho$=0, n=200} 
\scalebox{0.7}{
\begin{tabular}{|c|c|ccccccc|}
  \toprule 
 \multicolumn{1}{|c}{} &       & BOSS  & BS    & FS    & LASSO & Gamma LASSO & SparseNet & \multicolumn{1}{c|}{rLASSO} \\
 \multicolumn{1}{|c}{} &       & C$_p$-hdf/AICc-hdf/CV & CV    & CV    & AICc/CV & AICc/CV & CV    & \multicolumn{1}{c|}{CV}  \\
 \cmidrule{3-9}\multicolumn{1}{|c}{} &       & \multicolumn{7}{c|}{\% worse than the best possible BOSS}  \\
 \midrule\multirow{4}[2]{*}{hsnr} & p=14 & 8/6/18 & 19 & 19 & 42/41 & 16/20 & 13 & 15 \\ 
   & p=30 & 5/3/23 & 25 & 23 & 71/69 & 32/23 & 14 & 19 \\ 
   & p=60 & 4/2/21 & - & 23 & 87/85 & 51/24 & 16 & 19 \\ 
   & p=180 & 34/1/19 & - & 22 & 119/121 & 134/25 & 17 & 19 \\ 
  \midrule\multirow{4}[2]{*}{msnr} & p=14 & 17/14/18 & 19 & 19 & 43/42 & 23/23 & 14 & 16 \\ 
   & p=30 & 15/11/23 & 25 & 23 & 71/69 & 49/28 & 16 & 20 \\ 
   & p=60 & 13/9/22 & - & 24 & 87/85 & 82/28 & 17 & 20 \\ 
   & p=180 & 44/7/20 & - & 22 & 119/121 & 222/30 & 17 & 20 \\ 
  \midrule\multirow{4}[2]{*}{lsnr} & p=14 & 22/24/25 & 26 & 25 & 8/9 & 13/15 & 18 & 15 \\ 
   & p=30 & 32/34/26 & 26 & 26 & 2/2 & 14/9 & 10 & 7 \\ 
   & p=60 & 27/29/24 & - & 24 & -1/-2 & 27/5 & 6 & 3 \\ 
   & p=180 & 32/22/18 & - & 19 & -4/-2 & 83/2 & 1 & 1 \\ 
   \midrule 
 \multicolumn{1}{|c}{} &       & \multicolumn{7}{c|}{Relative efficiency} \\
\midrule\multirow{4}[2]{*}{hsnr} & p=14 & 0.98/1/0.89 & 0.89 & 0.89 & 0.74/0.75 & 0.91/0.88 & 0.93 & 0.92 \\ 
   & p=30 & 0.98/1/0.84 & 0.82 & 0.83 & 0.6/0.61 & 0.78/0.83 & 0.9 & 0.87 \\ 
   & p=60 & 0.99/1/0.84 & - & 0.83 & 0.55/0.55 & 0.68/0.83 & 0.88 & 0.86 \\ 
   & p=180 & 0.75/1/0.85 & - & 0.83 & 0.46/0.46 & 0.43/0.81 & 0.87 & 0.85 \\ 
  \midrule\multirow{4}[2]{*}{msnr} & p=14 & 0.98/1/0.96 & 0.96 & 0.96 & 0.8/0.81 & 0.93/0.93 & 1 & 0.98 \\ 
   & p=30 & 0.96/1/0.9 & 0.89 & 0.9 & 0.65/0.66 & 0.75/0.87 & 0.96 & 0.93 \\ 
   & p=60 & 0.97/1/0.9 & - & 0.88 & 0.58/0.59 & 0.6/0.85 & 0.94 & 0.91 \\ 
   & p=180 & 0.75/1/0.9 & - & 0.88 & 0.49/0.49 & 0.33/0.83 & 0.91 & 0.89 \\ 
  \midrule\multirow{4}[2]{*}{lsnr} & p=14 & 0.89/0.88/0.87 & 0.86 & 0.86 & 1/1 & 0.95/0.94 & 0.92 & 0.95 \\ 
   & p=30 & 0.78/0.76/0.81 & 0.81 & 0.81 & 1/1 & 0.9/0.94 & 0.93 & 0.96 \\ 
   & p=60 & 0.77/0.76/0.79 & - & 0.79 & 1/1 & 0.78/0.94 & 0.93 & 0.96 \\ 
   & p=180 & 0.73/0.79/0.81 & - & 0.81 & 1/0.98 & 0.52/0.94 & 0.95 & 0.95 \\ 
   \midrule 
 \multicolumn{1}{|c}{} &       & \multicolumn{7}{c|}{Sparsistency (number of extra variables)} \\
\midrule\multirow{4}[2]{*}{hsnr} & p=14 & 6(0.4)/6(0.2)/6(0.6) & 6(0.6) & 6(0.6) & 6(3.8)/6(4.6) & 6(0.9)/6(1.4) & 6(0.7) & 6(0.6) \\ 
   & p=30 & 6(0.1)/6(0)/6(0.6) & 6(0.7) & 6(0.6) & 6(7.3)/6(8.4) & 6(2.3)/6(1.7) & 6(1) & 6(0.7) \\ 
   & p=60 & 6(0.1)/6(0)/6(0.5) & - & 6(0.5) & 6(10)/6(11.3) & 6(4.6)/6(1.8) & 6(1.4) & 6(0.7) \\ 
   & p=180 & 6(8.9)/6(0)/6(0.4) & - & 6(0.4) & 6(15.3)/6(20.3) & 6(18.1)/6(2.1) & 6(2.3) & 6(0.7) \\ 
  \midrule\multirow{4}[2]{*}{msnr} & p=14 & 6(0.8)/6(0.6)/6(0.6) & 6(0.6) & 6(0.6) & 6(3.8)/6(4.6) & 6(1.1)/6(1.4) & 6(0.6) & 6(0.6) \\ 
   & p=30 & 6(0.4)/6(0.2)/6(0.6) & 6(0.7) & 6(0.6) & 6(7.4)/6(8.4) & 6(2.9)/6(1.6) & 6(0.7) & 6(0.7) \\ 
   & p=60 & 6(0.2)/6(0.1)/6(0.5) & - & 6(0.5) & 6(10)/6(11.3) & 6(6.2)/6(1.6) & 6(1) & 6(0.8) \\ 
   & p=180 & 6(9.1)/6(0.1)/6(0.4) & - & 6(0.4) & 6(15.3)/6(20.3) & 6(27.7)/6(1.7) & 6(1.4) & 6(0.8) \\ 
  \midrule\multirow{4}[2]{*}{lsnr} & p=14 & 5.4(4.5)/5.2(4.1)/4.6(1.7) & 4.5(1.6) & 4.6(1.7) & 5.5(3.3)/5.5(3.9) & 5(1.5)/5.1(2.9) & 5(2.7) & 5(2.1) \\ 
   & p=30 & 4(4.7)/3.1(2.1)/3.7(2) & 3.7(2) & 3.7(2) & 5.3(6.4)/5.3(7.1) & 4.9(3.8)/4.8(4.9) & 4.8(5.3) & 4.6(3.6) \\ 
   & p=60 & 2.8(1.9)/2(0.3)/2.9(1.4) & - & 3(1.4) & 5.1(8.5)/5.2(9.2) & 4.8(8.1)/4.5(6.3) & 4.5(7) & 4.3(4.4) \\ 
   & p=180 & 2.1(13.8)/1(0.1)/1.8(0.8) & - & 1.8(0.8) & 4.4(11.8)/4.5(15.4) & 4.7(36.6)/3.8(10.1) & 4.1(11.9) & 3.7(7.5) \\ 
   \bottomrule 
\end{tabular}
}
\end{table}

% latex table generated in R 3.6.2 by xtable 1.8-4 package
% Sat Dec 21 23:35:16 2019
\begin{table}[ht]
\centering
\caption{The performance of BOSS compared to other methods, Sparse-Ex1, $\rho$=0, n=2000} 
\scalebox{0.7}{
\begin{tabular}{|c|c|ccccccc|}
  \toprule 
 \multicolumn{1}{|c}{} &       & BOSS  & BS    & FS    & LASSO & Gamma LASSO & SparseNet & \multicolumn{1}{c|}{rLASSO} \\
 \multicolumn{1}{|c}{} &       & C$_p$-hdf/AICc-hdf/CV & CV    & CV    & AICc/CV & AICc/CV & CV    & \multicolumn{1}{c|}{CV}  \\
 \cmidrule{3-9}\multicolumn{1}{|c}{} &       & \multicolumn{7}{c|}{\% worse than the best possible BOSS}  \\
 \midrule\multirow{4}[2]{*}{hsnr} & p=14 & 6/6/17 & 18 & 17 & 40/40 & 12/14 & 11 & 13 \\ 
   & p=30 & 3/3/20 & 21 & 21 & 72/69 & 19/16 & 12 & 15 \\ 
   & p=60 & 2/2/23 & - & 22 & 100/96 & 30/19 & 15 & 17 \\ 
   & p=180 & 1/1/21 & - & 21 & 136/132 & 53/20 & 14 & 16 \\ 
  \midrule\multirow{4}[2]{*}{msnr} & p=14 & 6/6/17 & 18 & 17 & 41/40 & 14/17 & 12 & 13 \\ 
   & p=30 & 3/3/20 & 21 & 21 & 72/69 & 26/18 & 12 & 15 \\ 
   & p=60 & 2/2/23 & - & 22 & 100/96 & 46/22 & 15 & 18 \\ 
   & p=180 & 1/1/21 & - & 21 & 136/132 & 97/22 & 13 & 17 \\ 
  \midrule\multirow{4}[2]{*}{lsnr} & p=14 & 8/8/17 & 18 & 17 & 41/40 & 21/20 & 12 & 13 \\ 
   & p=30 & 5/4/20 & 21 & 21 & 72/69 & 44/23 & 12 & 15 \\ 
   & p=60 & 5/4/23 & - & 22 & 100/96 & 84/27 & 16 & 17 \\ 
   & p=180 & 5/5/21 & - & 21 & 136/132 & 192/27 & 13 & 17 \\ 
   \midrule 
 \multicolumn{1}{|c}{} &       & \multicolumn{7}{c|}{Relative efficiency} \\
\midrule\multirow{4}[2]{*}{hsnr} & p=14 & 1/1/0.9 & 0.9 & 0.91 & 0.76/0.76 & 0.95/0.92 & 0.95 & 0.94 \\ 
   & p=30 & 1/1/0.86 & 0.85 & 0.85 & 0.6/0.61 & 0.87/0.88 & 0.92 & 0.89 \\ 
   & p=60 & 1/1/0.83 & - & 0.83 & 0.51/0.52 & 0.78/0.85 & 0.89 & 0.87 \\ 
   & p=180 & 1/1/0.84 & - & 0.84 & 0.43/0.44 & 0.66/0.85 & 0.89 & 0.87 \\ 
  \midrule\multirow{4}[2]{*}{msnr} & p=14 & 1/1/0.9 & 0.9 & 0.91 & 0.75/0.76 & 0.93/0.91 & 0.95 & 0.94 \\ 
   & p=30 & 1/1/0.86 & 0.85 & 0.85 & 0.6/0.61 & 0.82/0.87 & 0.92 & 0.89 \\ 
   & p=60 & 1/1/0.83 & - & 0.83 & 0.51/0.52 & 0.7/0.84 & 0.89 & 0.86 \\ 
   & p=180 & 1/1/0.83 & - & 0.83 & 0.43/0.44 & 0.51/0.83 & 0.9 & 0.87 \\ 
  \midrule\multirow{4}[2]{*}{lsnr} & p=14 & 1/1/0.93 & 0.92 & 0.93 & 0.77/0.77 & 0.9/0.9 & 0.97 & 0.96 \\ 
   & p=30 & 1/1/0.87 & 0.86 & 0.87 & 0.61/0.62 & 0.73/0.85 & 0.93 & 0.91 \\ 
   & p=60 & 1/1/0.85 & - & 0.85 & 0.52/0.53 & 0.57/0.82 & 0.9 & 0.89 \\ 
   & p=180 & 1/1/0.86 & - & 0.86 & 0.44/0.45 & 0.36/0.83 & 0.93 & 0.9 \\ 
   \midrule 
 \multicolumn{1}{|c}{} &       & \multicolumn{7}{c|}{Sparsistency (number of extra variables)} \\
\midrule\multirow{4}[2]{*}{hsnr} & p=14 & 6(0.2)/6(0.2)/6(0.6) & 6(0.6) & 6(0.6) & 6(3.8)/6(4.4) & 6(0.9)/6(1.2) & 6(0.6) & 6(0.5) \\ 
   & p=30 & 6(0.1)/6(0.1)/6(0.5) & 6(0.6) & 6(0.6) & 6(8.5)/6(8.8) & 6(1.9)/6(1.7) & 6(1) & 6(0.5) \\ 
   & p=60 & 6(0)/6(0)/6(0.5) & - & 6(0.5) & 6(13.4)/6(12.6) & 6(4)/6(2.2) & 6(1.6) & 6(0.6) \\ 
   & p=180 & 6(0)/6(0)/6(0.4) & - & 6(0.3) & 6(23)/6(20.1) & 6(10.4)/6(2.5) & 6(2.1) & 6(0.4) \\ 
  \midrule\multirow{4}[2]{*}{msnr} & p=14 & 6(0.2)/6(0.2)/6(0.6) & 6(0.6) & 6(0.6) & 6(3.8)/6(4.4) & 6(1)/6(1.3) & 6(0.6) & 6(0.5) \\ 
   & p=30 & 6(0.1)/6(0.1)/6(0.5) & 6(0.6) & 6(0.6) & 6(8.5)/6(8.7) & 6(2.2)/6(1.6) & 6(0.9) & 6(0.5) \\ 
   & p=60 & 6(0)/6(0)/6(0.5) & - & 6(0.5) & 6(13.3)/6(12.6) & 6(5)/6(1.9) & 6(1.5) & 6(0.6) \\ 
   & p=180 & 6(0)/6(0)/6(0.4) & - & 6(0.3) & 6(22.8)/6(20) & 6(15.7)/6(1.9) & 6(1.8) & 6(0.4) \\ 
  \midrule\multirow{4}[2]{*}{lsnr} & p=14 & 6(0.4)/6(0.4)/6(0.6) & 6(0.6) & 6(0.6) & 6(3.8)/6(4.4) & 6(1.1)/6(1.3) & 6(0.5) & 6(0.5) \\ 
   & p=30 & 6(0.1)/6(0.1)/6(0.5) & 6(0.6) & 6(0.6) & 6(8.5)/6(8.7) & 6(2.9)/6(1.4) & 6(0.7) & 6(0.5) \\ 
   & p=60 & 6(0.1)/6(0.1)/6(0.5) & - & 6(0.5) & 6(13.4)/6(12.5) & 6(6.7)/6(1.6) & 6(1) & 6(0.6) \\ 
   & p=180 & 6(0.1)/6(0.1)/6(0.4) & - & 6(0.3) & 6(23)/6(20) & 6(23.6)/6(1.1) & 6(1) & 6(0.4) \\ 
   \bottomrule 
\end{tabular}
}
\end{table}

% latex table generated in R 3.6.2 by xtable 1.8-4 package
% Sat Dec 21 23:35:17 2019
\begin{table}[ht]
\centering
\caption{The performance of BOSS compared to other methods, Sparse-Ex1, $\rho$=0.5, n=200} 
\scalebox{0.7}{
\begin{tabular}{|c|c|ccccccc|}
  \toprule 
 \multicolumn{1}{|c}{} &       & BOSS  & BS    & FS    & LASSO & Gamma LASSO & SparseNet & \multicolumn{1}{c|}{rLASSO} \\
 \multicolumn{1}{|c}{} &       & C$_p$-hdf/AICc-hdf/CV & CV    & CV    & AICc/CV & AICc/CV & CV    & \multicolumn{1}{c|}{CV}  \\
 \cmidrule{3-9}\multicolumn{1}{|c}{} &       & \multicolumn{7}{c|}{\% worse than the best possible BOSS}  \\
 \midrule\multirow{4}[2]{*}{hsnr} & p=14 & 7/5/20 & 18 & 21 & 39/39 & 16/19 & 13 & 18 \\ 
   & p=30 & 4/2/20 & 22 & 21 & 66/65 & 34/21 & 15 & 18 \\ 
   & p=60 & 3/2/21 & - & 23 & 92/89 & 57/25 & 16 & 18 \\ 
   & p=180 & 65/1/19 & - & 22 & 139/134 & 136/25 & 16 & 16 \\ 
  \midrule\multirow{4}[2]{*}{msnr} & p=14 & 19/17/20 & 15 & 20 & 34/34 & 21/20 & 12 & 21 \\ 
   & p=30 & 13/9/22 & 23 & 23 & 66/65 & 50/26 & 17 & 26 \\ 
   & p=60 & 12/8/22 & - & 25 & 90/88 & 85/30 & 17 & 29 \\ 
   & p=180 & 46/9/21 & - & 25 & 126/125 & 211/31 & 18 & 39 \\ 
  \midrule\multirow{4}[2]{*}{lsnr} & p=14 & 19/22/23 & 23 & 21 & -2/-2 & 12/6 & 7 & 4 \\ 
   & p=30 & 26/27/25 & 24 & 23 & -4/-5 & 9/2 & 2 & -1 \\ 
   & p=60 & 24/26/22 & - & 20 & -4/-6 & 22/1 & 0 & -3 \\ 
   & p=180 & 48/12/14 & - & 14 & -2/-2 & 91/2 & 1 & 0 \\ 
   \midrule 
 \multicolumn{1}{|c}{} &       & \multicolumn{7}{c|}{Relative efficiency} \\
\midrule\multirow{4}[2]{*}{hsnr} & p=14 & 0.98/1/0.87 & 0.89 & 0.87 & 0.75/0.75 & 0.9/0.88 & 0.93 & 0.89 \\ 
   & p=30 & 0.99/1/0.85 & 0.84 & 0.84 & 0.61/0.62 & 0.77/0.85 & 0.89 & 0.87 \\ 
   & p=60 & 0.99/1/0.85 & - & 0.83 & 0.53/0.54 & 0.65/0.82 & 0.88 & 0.87 \\ 
   & p=180 & 0.61/1/0.85 & - & 0.83 & 0.42/0.43 & 0.43/0.81 & 0.87 & 0.87 \\ 
  \midrule\multirow{4}[2]{*}{msnr} & p=14 & 0.95/0.96/0.94 & 0.98 & 0.93 & 0.84/0.84 & 0.93/0.94 & 1 & 0.93 \\ 
   & p=30 & 0.97/1/0.9 & 0.89 & 0.89 & 0.66/0.66 & 0.73/0.87 & 0.93 & 0.87 \\ 
   & p=60 & 0.97/1/0.88 & - & 0.87 & 0.57/0.58 & 0.59/0.83 & 0.93 & 0.84 \\ 
   & p=180 & 0.75/1/0.9 & - & 0.88 & 0.48/0.49 & 0.35/0.83 & 0.93 & 0.79 \\ 
  \midrule\multirow{4}[2]{*}{lsnr} & p=14 & 0.82/0.8/0.8 & 0.8 & 0.81 & 1/1 & 0.88/0.92 & 0.91 & 0.94 \\ 
   & p=30 & 0.75/0.75/0.76 & 0.77 & 0.77 & 0.99/1 & 0.87/0.93 & 0.93 & 0.96 \\ 
   & p=60 & 0.76/0.75/0.77 & - & 0.78 & 0.99/1 & 0.77/0.94 & 0.95 & 0.97 \\ 
   & p=180 & 0.66/0.87/0.86 & - & 0.86 & 1/1 & 0.51/0.96 & 0.97 & 0.98 \\ 
   \midrule 
 \multicolumn{1}{|c}{} &       & \multicolumn{7}{c|}{Sparsistency (number of extra variables)} \\
\midrule\multirow{4}[2]{*}{hsnr} & p=14 & 6(0.3)/6(0.2)/6(0.8) & 6(0.6) & 6(0.8) & 6(3.8)/6(4.2) & 6(0.9)/6(1.3) & 6(0.6) & 6(1) \\ 
   & p=30 & 6(0.1)/6(0)/6(0.6) & 6(0.7) & 6(0.6) & 6(6)/6(8.6) & 6(2.5)/6(1.5) & 6(1.1) & 6(0.8) \\ 
   & p=60 & 6(0)/6(0)/6(0.5) & - & 6(0.6) & 6(8.8)/6(12.5) & 6(4.9)/6(1.7) & 6(1.4) & 6(0.9) \\ 
   & p=180 & 6(9.4)/6(0)/6(0.3) & - & 6(0.4) & 6(11.7)/6(21.3) & 6(16.6)/6(1.7) & 6(1.9) & 6(0.6) \\ 
  \midrule\multirow{4}[2]{*}{msnr} & p=14 & 6(1.2)/6(1)/6(1.1) & 6(0.7) & 6(1.1) & 6(3.8)/6(4.2) & 6(1.1)/6(1.5) & 6(0.7) & 6(1.7) \\ 
   & p=30 & 6(0.4)/6(0.2)/6(0.7) & 6(0.7) & 6(0.7) & 6(6.1)/6(8.6) & 6(3.1)/6(1.6) & 6(1) & 6(1.3) \\ 
   & p=60 & 6(0.2)/6(0.2)/6(0.6) & - & 6(0.6) & 6(8.8)/6(12.4) & 6(6.3)/6(1.8) & 6(1.1) & 6(1.6) \\ 
   & p=180 & 6(10.2)/6(0.1)/6(0.4) & - & 6(0.4) & 6(16)/6(21.1) & 6(26.5)/6(1.6) & 6(1.3) & 6(1.6) \\ 
  \midrule\multirow{4}[2]{*}{lsnr} & p=14 & 4.4(3.5)/4.1(3.2)/3.8(2.4) & 3.7(2) & 3.8(2.2) & 5(3.3)/5.1(3.5) & 4(1.6)/4.6(2.8) & 4.6(2.7) & 4.6(2.6) \\ 
   & p=30 & 3.7(4.4)/2.9(2.1)/3.4(2.7) & 3.3(2.3) & 3.4(2.3) & 4.8(4.8)/5.2(7.4) & 4.5(3.9)/4.7(5.2) & 4.8(5.7) & 4.6(4.5) \\ 
   & p=60 & 2.4(2)/1.6(0.5)/2.5(1.8) & - & 2.5(1.6) & 4.4(6.4)/4.8(10) & 4.4(8.2)/4.2(7) & 4.4(7.9) & 4.1(5.7) \\ 
   & p=180 & 1.3(14.1)/0.3(0.1)/1(0.8) & - & 1.1(0.7) & 2.5(4.9)/3.2(12) & 4.2(35.8)/2.9(9.2) & 3(9.7) & 2.7(7.5) \\ 
   \bottomrule 
\end{tabular}
}
\end{table}

% latex table generated in R 3.6.1 by xtable 1.8-4 package
% Sat Nov  9 19:29:14 2019
\begin{table}[ht]
\centering
\caption{The performance of BOSS compared to other methods, Sparse-Ex1, $\rho$=0.5, n=2000} 

\scalebox{0.7}{
\begin{tabular}{|c|c|ccccccc|}
  \toprule 
 \multicolumn{1}{|c}{} &       & BOSS  & BS    & FS    & LASSO & Gamma LASSO & SparseNet & \multicolumn{1}{c|}{rLASSO} \\
 \multicolumn{1}{|c}{} &       & C$_p$-hdf/AICc-hdf/CV & CV    & CV    & AICc/CV & AICc/CV & CV    & \multicolumn{1}{c|}{CV}  \\
 \cmidrule{3-9}\multicolumn{1}{|c}{} &       & \multicolumn{7}{c|}{\% worse than the best possible BOSS}  \\
 \midrule\multirow{4}[2]{*}{hsnr} & p=14 & 7/6/17 & 19 & 17 & 38/39 & 12/14 & 12 & 17 \\ 
   & p=30 & 3/3/20 & 23 & 22 & 67/65 & 21/18 & 14 & 23 \\ 
   & p=60 & 2/2/21 & - & 21 & 91/89 & 29/19 & 15 & 25 \\ 
   & p=180 & 2/1/22 & - & 23 & 126/123 & 52/20 & 15 & 28 \\ 
  \midrule\multirow{4}[2]{*}{msnr} & p=14 & 7/6/17 & 19 & 17 & 39/39 & 16/17 & 12 & 17 \\ 
   & p=30 & 3/3/20 & 23 & 22 & 67/66 & 29/20 & 14 & 23 \\ 
   & p=60 & 2/2/21 & - & 21 & 91/89 & 45/21 & 14 & 25 \\ 
   & p=180 & 2/1/22 & - & 23 & 126/123 & 95/22 & 15 & 27 \\ 
  \midrule\multirow{4}[2]{*}{lsnr} & p=14 & 13/13/17 & 18 & 17 & 39/39 & 23/21 & 13 & 20 \\ 
   & p=30 & 6/6/20 & 23 & 22 & 67/65 & 49/25 & 15 & 23 \\ 
   & p=60 & 5/5/21 & - & 21 & 91/89 & 83/26 & 15 & 25 \\ 
   & p=180 & 4/4/22 & - & 23 & 126/123 & 192/28 & 15 & 27 \\ 
   \midrule 
 \multicolumn{1}{|c}{} &       & \multicolumn{7}{c|}{Relative efficiency} \\
\midrule\multirow{4}[2]{*}{hsnr} & p=14 & 1/1/0.91 & 0.9 & 0.91 & 0.77/0.77 & 0.95/0.93 & 0.95 & 0.91 \\ 
   & p=30 & 1/1/0.86 & 0.84 & 0.84 & 0.62/0.62 & 0.85/0.87 & 0.9 & 0.84 \\ 
   & p=60 & 1/1/0.84 & - & 0.84 & 0.54/0.54 & 0.79/0.86 & 0.89 & 0.82 \\ 
   & p=180 & 1/1/0.83 & - & 0.83 & 0.45/0.45 & 0.67/0.85 & 0.88 & 0.8 \\ 
  \midrule\multirow{4}[2]{*}{msnr} & p=14 & 1/1/0.91 & 0.9 & 0.91 & 0.77/0.76 & 0.92/0.91 & 0.95 & 0.91 \\ 
   & p=30 & 1/1/0.86 & 0.84 & 0.84 & 0.62/0.62 & 0.8/0.86 & 0.9 & 0.84 \\ 
   & p=60 & 1/1/0.84 & - & 0.84 & 0.54/0.54 & 0.71/0.84 & 0.89 & 0.82 \\ 
   & p=180 & 1/1/0.83 & - & 0.83 & 0.45/0.45 & 0.52/0.83 & 0.89 & 0.8 \\ 
  \midrule\multirow{4}[2]{*}{lsnr} & p=14 & 1/1/0.96 & 0.95 & 0.96 & 0.81/0.81 & 0.92/0.93 & 1 & 0.94 \\ 
   & p=30 & 1/1/0.88 & 0.86 & 0.87 & 0.63/0.64 & 0.71/0.85 & 0.92 & 0.86 \\ 
   & p=60 & 1/1/0.87 & - & 0.87 & 0.55/0.56 & 0.57/0.84 & 0.91 & 0.84 \\ 
   & p=180 & 1/1/0.85 & - & 0.85 & 0.46/0.47 & 0.36/0.82 & 0.9 & 0.82 \\ 
   \midrule 
 \multicolumn{1}{|c}{} &       & \multicolumn{7}{c|}{Sparsistency (number of extra variables)} \\
\midrule\multirow{4}[2]{*}{hsnr} & p=14 & 6(0.3)/6(0.3)/6(0.6) & 6(0.6) & 6(0.6) & 6(3.8)/6(4.2) & 6(0.9)/6(1) & 6(0.6) & 6(0.8) \\ 
   & p=30 & 6(0.1)/6(0.1)/6(0.6) & 6(0.7) & 6(0.6) & 6(6.6)/6(8.3) & 6(2.1)/6(1.7) & 6(1) & 6(1) \\ 
   & p=60 & 6(0)/6(0)/6(0.5) & - & 6(0.4) & 6(10.3)/6(12.1) & 6(3.7)/6(2) & 6(1.3) & 6(0.9) \\ 
   & p=180 & 6(0)/6(0)/6(0.5) & - & 6(0.4) & 6(13.8)/6(18.9) & 6(9.3)/6(2.5) & 6(2.3) & 6(0.9) \\ 
  \midrule\multirow{4}[2]{*}{msnr} & p=14 & 6(0.3)/6(0.3)/6(0.6) & 6(0.6) & 6(0.6) & 6(3.8)/6(4.2) & 6(1)/6(1.2) & 6(0.6) & 6(0.8) \\ 
   & p=30 & 6(0.1)/6(0.1)/6(0.6) & 6(0.7) & 6(0.6) & 6(6.6)/6(8.3) & 6(2.5)/6(1.6) & 6(0.9) & 6(1) \\ 
   & p=60 & 6(0)/6(0)/6(0.5) & - & 6(0.4) & 6(10.4)/6(12) & 6(4.6)/6(1.7) & 6(1.2) & 6(0.9) \\ 
   & p=180 & 6(0)/6(0)/6(0.5) & - & 6(0.4) & 6(14)/6(18.7) & 6(14.2)/6(2) & 6(2.1) & 6(0.9) \\ 
  \midrule\multirow{4}[2]{*}{lsnr} & p=14 & 6(0.6)/6(0.6)/6(0.6) & 6(0.6) & 6(0.6) & 6(3.8)/6(4.3) & 6(1.2)/6(1.3) & 6(0.5) & 6(1.1) \\ 
   & p=30 & 6(0.2)/6(0.1)/6(0.6) & 6(0.7) & 6(0.6) & 6(6.6)/6(8.4) & 6(3.1)/6(1.5) & 6(0.7) & 6(1) \\ 
   & p=60 & 6(0.1)/6(0.1)/6(0.5) & - & 6(0.4) & 6(10.3)/6(12) & 6(6.5)/6(1.4) & 6(0.8) & 6(0.9) \\ 
   & p=180 & 6(0.1)/6(0)/6(0.5) & - & 6(0.4) & 6(13.8)/6(18.8) & 6(23)/6(1.4) & 6(1.4) & 6(0.9) \\ 
   \bottomrule 
\end{tabular}
}
\end{table}

% latex table generated in R 3.6.2 by xtable 1.8-4 package
% Sat Dec 21 23:35:24 2019
\begin{table}[ht]
\centering
\caption{The performance of BOSS compared to other methods, Sparse-Ex1, $\rho$=0.9, n=200} 
\scalebox{0.7}{
\begin{tabular}{|c|c|ccccccc|}
  \toprule 
 \multicolumn{1}{|c}{} &       & BOSS  & BS    & FS    & LASSO & Gamma LASSO & SparseNet & \multicolumn{1}{c|}{rLASSO} \\
 \multicolumn{1}{|c}{} &       & C$_p$-hdf/AICc-hdf/CV & CV    & CV    & AICc/CV & AICc/CV & CV    & \multicolumn{1}{c|}{CV}  \\
 \cmidrule{3-9}\multicolumn{1}{|c}{} &       & \multicolumn{7}{c|}{\% worse than the best possible BOSS}  \\
 \midrule\multirow{4}[2]{*}{hsnr} & p=14 & 20/21/19 & 16 & 18 & 6/5 & 10/6 & 12 & 7 \\ 
   & p=30 & 16/16/28 & 15 & 28 & 24/24 & 26/16 & 12 & 25 \\ 
   & p=60 & 15/15/34 & - & 34 & 58/59 & 66/38 & 28 & 23 \\ 
   & p=180 & 59/8/35 & - & 36 & 98/98 & 153/45 & 27 & 23 \\ 
  \midrule\multirow{4}[2]{*}{msnr} & p=14 & 26/27/18 & 17 & 16 & -8/-9 & 11/1 & 3 & -3 \\ 
   & p=30 & 24/27/20 & 19 & 15 & -11/-12 & 5/-5 & -3 & -7 \\ 
   & p=60 & 16/16/19 & - & 16 & 2/3 & 27/6 & 8 & 0 \\ 
   & p=180 & 26/7/15 & - & 16 & 31/31 & 111/17 & 15 & 12 \\ 
  \midrule\multirow{4}[2]{*}{lsnr} & p=14 & 28/27/24 & 22 & 19 & -9/-13 & 9/3 & 3 & -1 \\ 
   & p=30 & 19/18/21 & 19 & 15 & -18/-21 & 5/-7 & -9 & -10 \\ 
   & p=60 & 17/18/20 & - & 14 & -20/-21 & 6/-12 & -14 & -18 \\ 
   & p=180 & 47/21/18 & - & 14 & -13/-14 & 53/-9 & -10 & -12 \\ 
   \midrule 
 \multicolumn{1}{|c}{} &       & \multicolumn{7}{c|}{Relative efficiency} \\
\midrule\multirow{4}[2]{*}{hsnr} & p=14 & 0.87/0.87/0.89 & 0.91 & 0.89 & 0.99/1 & 0.95/0.99 & 0.94 & 0.98 \\ 
   & p=30 & 0.97/0.97/0.87 & 0.98 & 0.88 & 0.91/0.91 & 0.89/0.96 & 1 & 0.89 \\ 
   & p=60 & 1/0.99/0.85 & - & 0.85 & 0.73/0.72 & 0.69/0.83 & 0.9 & 0.93 \\ 
   & p=180 & 0.68/1/0.8 & - & 0.79 & 0.55/0.54 & 0.43/0.74 & 0.85 & 0.88 \\ 
  \midrule\multirow{4}[2]{*}{msnr} & p=14 & 0.72/0.71/0.77 & 0.77 & 0.78 & 0.98/1 & 0.82/0.9 & 0.88 & 0.93 \\ 
   & p=30 & 0.71/0.69/0.74 & 0.74 & 0.76 & 0.99/1 & 0.84/0.93 & 0.91 & 0.95 \\ 
   & p=60 & 0.86/0.85/0.83 & - & 0.86 & 0.98/0.97 & 0.78/0.94 & 0.92 & 1 \\ 
   & p=180 & 0.86/1/0.93 & - & 0.92 & 0.82/0.82 & 0.51/0.92 & 0.93 & 0.96 \\ 
  \midrule\multirow{4}[2]{*}{lsnr} & p=14 & 0.68/0.68/0.7 & 0.71 & 0.73 & 0.95/1 & 0.8/0.85 & 0.85 & 0.88 \\ 
   & p=30 & 0.67/0.67/0.66 & 0.67 & 0.69 & 0.96/1 & 0.76/0.85 & 0.87 & 0.88 \\ 
   & p=60 & 0.68/0.67/0.66 & - & 0.69 & 0.98/1 & 0.74/0.89 & 0.92 & 0.96 \\ 
   & p=180 & 0.59/0.71/0.73 & - & 0.76 & 1/1 & 0.57/0.95 & 0.96 & 0.98 \\ 
   \midrule 
 \multicolumn{1}{|c}{} &       & \multicolumn{7}{c|}{Sparsistency (number of extra variables)} \\
\midrule\multirow{4}[2]{*}{hsnr} & p=14 & 5.6(2.8)/5.5(2.6)/5.7(2.8) & 5.6(1.9) & 5.6(2.6) & 5.9(4)/6(4) & 5.6(1.4)/5.8(2) & 5.6(2.2) & 5.9(3.8) \\ 
   & p=30 & 5.6(1.5)/5.6(1.2)/5.8(3.2) & 5.8(1.5) & 5.8(2.8) & 6(7.5)/6(8.6) & 5.8(3.4)/5.8(3.4) & 5.8(2.3) & 6(7.3) \\ 
   & p=60 & 5.9(1)/5.8(0.9)/5.9(2.8) & - & 5.9(1.9) & 6(10.2)/6(12.1) & 5.9(6.6)/5.9(4) & 5.9(3) & 6(3.1) \\ 
   & p=180 & 6(9.9)/5.9(0.4)/6(2.4) & - & 6(1.2) & 6(13.6)/6(18) & 6(21.6)/6(4.8) & 6(4.3) & 6(2.5) \\ 
  \midrule\multirow{4}[2]{*}{msnr} & p=14 & 2.9(2)/2.7(1.7)/3.7(2.9) & 3.5(2.4) & 3.6(2.7) & 4.9(3.7)/5(3.7) & 3.3(1.6)/4.2(2.6) & 4(2.6) & 4.6(3.4) \\ 
   & p=30 & 3.3(2.9)/3(2.1)/3.9(4.9) & 3.6(3.6) & 3.8(4.1) & 5.1(7.3)/5.3(8.2) & 4.1(3.8)/4.7(5.7) & 4.6(5.5) & 4.9(7.4) \\ 
   & p=60 & 4.3(2.8)/4.1(2.1)/4.4(4.3) & - & 4.4(3.2) & 5.6(10.1)/5.6(11.8) & 4.7(7.4)/5(6.9) & 5(6.8) & 5.5(7.9) \\ 
   & p=180 & 5.1(11.7)/5(1.2)/5(2.3) & - & 5(1.7) & 5.9(14.8)/5.9(17.5) & 5.3(28.6)/5.4(6.8) & 5.4(6.3) & 5.6(5.5) \\ 
  \midrule\multirow{4}[2]{*}{lsnr} & p=14 & 1(1)/0.9(0.9)/1.5(1.7) & 1.4(1.4) & 1.5(1.4) & 2.6(2.5)/2.9(2.7) & 1.5(1.4)/2(1.8) & 2(1.9) & 2.1(2.1) \\ 
   & p=30 & 0.9(1.8)/0.8(1.4)/1.2(2.9) & 1.1(2.4) & 1.2(2.2) & 2.2(4.8)/2.6(6) & 1.6(3.2)/2(4.2) & 2.1(4.5) & 1.9(4.5) \\ 
   & p=60 & 0.9(2)/0.7(1.4)/1.1(3.2) & - & 1.1(2.4) & 2.6(7)/2.9(9) & 2(6.5)/2.3(6.6) & 2.5(7.2) & 2.6(7.3) \\ 
   & p=180 & 1.3(14.7)/0.5(0.6)/1(2.6) & - & 1.2(1.9) & 2.8(9.7)/3.1(13.9) & 2.6(26.4)/2.6(10) & 2.8(11.4) & 2.7(9.4) \\ 
   \bottomrule 
\end{tabular}
}
\end{table}

% latex table generated in R 3.6.2 by xtable 1.8-4 package
% Sat Dec 21 23:35:31 2019
\begin{table}[ht]
\centering
\caption{The performance of BOSS compared to other methods, Sparse-Ex1, $\rho$=0.9, n=2000} 
\scalebox{0.7}{
\begin{tabular}{|c|c|ccccccc|}
  \toprule 
 \multicolumn{1}{|c}{} &       & BOSS  & BS    & FS    & LASSO & Gamma LASSO & SparseNet & \multicolumn{1}{c|}{rLASSO} \\
 \multicolumn{1}{|c}{} &       & C$_p$-hdf/AICc-hdf/CV & CV    & CV    & AICc/CV & AICc/CV & CV    & \multicolumn{1}{c|}{CV}  \\
 \cmidrule{3-9}\multicolumn{1}{|c}{} &       & \multicolumn{7}{c|}{\% worse than the best possible BOSS}  \\
 \midrule\multirow{4}[2]{*}{hsnr} & p=14 & 7/7/21 & 18 & 22 & 30/30 & 6/8 & 17 & 23 \\ 
   & p=30 & 4/3/22 & 25 & 23 & 58/57 & 14/12 & 21 & 33 \\ 
   & p=60 & 2/2/21 & - & 22 & 81/82 & 34/19 & 18 & 18 \\ 
   & p=180 & 1/1/18 & - & 22 & 114/113 & 68/18 & 15 & 17 \\ 
  \midrule\multirow{4}[2]{*}{msnr} & p=14 & 15/15/21 & 14 & 21 & 22/20 & 14/12 & 14 & 23 \\ 
   & p=30 & 4/4/28 & 23 & 28 & 54/53 & 46/22 & 21 & 50 \\ 
   & p=60 & 2/2/22 & - & 24 & 81/82 & 76/24 & 20 & 24 \\ 
   & p=180 & 1/1/18 & - & 22 & 114/113 & 140/21 & 15 & 18 \\ 
  \midrule\multirow{4}[2]{*}{lsnr} & p=14 & 27/28/17 & 17 & 15 & -7/-9 & 10/2 & 4 & -2 \\ 
   & p=30 & 22/22/20 & 16 & 17 & -7/-7 & 10/-1 & 0 & -4 \\ 
   & p=60 & 9/9/17 & - & 16 & 15/15 & 42/13 & 16 & 12 \\ 
   & p=180 & 3/3/13 & - & 15 & 59/58 & 146/21 & 33 & 18 \\ 
   \midrule 
 \multicolumn{1}{|c}{} &       & \multicolumn{7}{c|}{Relative efficiency} \\
\midrule\multirow{4}[2]{*}{hsnr} & p=14 & 0.99/0.99/0.87 & 0.89 & 0.87 & 0.81/0.81 & 1/0.97 & 0.9 & 0.86 \\ 
   & p=30 & 1/1/0.85 & 0.83 & 0.84 & 0.65/0.66 & 0.9/0.92 & 0.86 & 0.77 \\ 
   & p=60 & 1/1/0.84 & - & 0.83 & 0.56/0.56 & 0.76/0.86 & 0.86 & 0.86 \\ 
   & p=180 & 1/1/0.85 & - & 0.83 & 0.47/0.47 & 0.6/0.85 & 0.88 & 0.86 \\ 
  \midrule\multirow{4}[2]{*}{msnr} & p=14 & 0.98/0.98/0.93 & 0.99 & 0.93 & 0.92/0.93 & 0.99/1 & 0.98 & 0.91 \\ 
   & p=30 & 1/1/0.81 & 0.85 & 0.81 & 0.67/0.68 & 0.71/0.85 & 0.86 & 0.69 \\ 
   & p=60 & 1/1/0.83 & - & 0.82 & 0.56/0.56 & 0.58/0.82 & 0.85 & 0.82 \\ 
   & p=180 & 1/1/0.85 & - & 0.83 & 0.47/0.47 & 0.42/0.84 & 0.87 & 0.86 \\ 
  \midrule\multirow{4}[2]{*}{lsnr} & p=14 & 0.72/0.72/0.78 & 0.78 & 0.79 & 0.98/1 & 0.83/0.9 & 0.88 & 0.93 \\ 
   & p=30 & 0.76/0.76/0.77 & 0.8 & 0.79 & 0.99/1 & 0.84/0.94 & 0.92 & 0.97 \\ 
   & p=60 & 1/1/0.93 & - & 0.93 & 0.94/0.94 & 0.77/0.96 & 0.94 & 0.97 \\ 
   & p=180 & 1/1/0.92 & - & 0.9 & 0.65/0.65 & 0.42/0.85 & 0.78 & 0.88 \\ 
   \midrule 
 \multicolumn{1}{|c}{} &       & \multicolumn{7}{c|}{Sparsistency (number of extra variables)} \\
\midrule\multirow{4}[2]{*}{hsnr} & p=14 & 6(0.3)/6(0.3)/6(0.9) & 6(0.7) & 6(0.9) & 6(3.4)/6(3.4) & 6(0.2)/6(0.3) & 6(0.6) & 6(1.9) \\ 
   & p=30 & 6(0.1)/6(0.1)/6(0.7) & 6(0.7) & 6(0.6) & 6(7.6)/6(8) & 6(0.9)/6(0.7) & 6(1.2) & 6(2.3) \\ 
   & p=60 & 6(0)/6(0)/6(1) & - & 6(0.5) & 6(11.1)/6(12.1) & 6(2.9)/6(1.4) & 6(1.4) & 6(0.8) \\ 
   & p=180 & 6(0)/6(0)/6(1.3) & - & 6(0.4) & 6(15.6)/6(18.5) & 6(9.5)/6(1.7) & 6(1.9) & 6(0.6) \\ 
  \midrule\multirow{4}[2]{*}{msnr} & p=14 & 5.9(1.2)/5.9(1.2)/6(1.7) & 6(1) & 6(1.6) & 6(4)/6(3.9) & 6(1)/6(1.2) & 5.9(1) & 6(3.3) \\ 
   & p=30 & 6(0.2)/6(0.2)/6(1.2) & 6(0.7) & 6(1) & 6(7.8)/6(8.2) & 6(3.1)/6(1.4) & 6(1.2) & 6(4.8) \\ 
   & p=60 & 6(0)/6(0)/6(1.1) & - & 6(0.6) & 6(11.1)/6(12.1) & 6(6.4)/6(1.5) & 6(1.2) & 6(1.1) \\ 
   & p=180 & 6(0)/6(0)/6(1.3) & - & 6(0.4) & 6(15.8)/6(18.5) & 6(18.7)/6(1.4) & 6(1.5) & 6(0.6) \\ 
  \midrule\multirow{4}[2]{*}{lsnr} & p=14 & 3.5(2.1)/3.5(2.1)/4.3(3) & 4(2.4) & 4.2(2.8) & 5.3(3.8)/5.4(3.8) & 3.9(1.8)/4.6(2.7) & 4.5(2.6) & 5(3.5) \\ 
   & p=30 & 4.1(2.7)/4.1(2.7)/4.6(4.5) & 4.5(3.2) & 4.5(3.7) & 5.6(7.7)/5.7(8) & 4.7(4)/5.1(5.3) & 5.1(5) & 5.5(7) \\ 
   & p=60 & 5(1.4)/5(1.4)/5.1(3.1) & - & 5.1(2.2) & 5.9(11.1)/5.9(12) & 5.3(8.2)/5.4(6) & 5.3(5.7) & 5.8(7.4) \\ 
   & p=180 & 5.6(0.6)/5.6(0.6)/5.6(2) & - & 5.6(0.9) & 6(15.7)/6(18.4) & 5.7(26.1)/5.6(3.6) & 5.5(5.3) & 5.9(3.1) \\ 
   \bottomrule 
\end{tabular}
}
\end{table}

% latex table generated in R 3.6.2 by xtable 1.8-4 package
% Sat Dec 21 23:35:32 2019
\begin{table}[ht]
\centering
\caption{The performance of BOSS compared to other methods, Sparse-Ex2, $\rho$=0, n=200} 
\scalebox{0.7}{
\begin{tabular}{|c|c|ccccccc|}
  \toprule 
 \multicolumn{1}{|c}{} &       & BOSS  & BS    & FS    & LASSO & Gamma LASSO & SparseNet & \multicolumn{1}{c|}{rLASSO} \\
 \multicolumn{1}{|c}{} &       & C$_p$-hdf/AICc-hdf/CV & CV    & CV    & AICc/CV & AICc/CV & CV    & \multicolumn{1}{c|}{CV}  \\
 \cmidrule{3-9}\multicolumn{1}{|c}{} &       & \multicolumn{7}{c|}{\% worse than the best possible BOSS}  \\
 \midrule\multirow{4}[2]{*}{hsnr} & p=14 & 8/6/20 & 21 & 20 & 41/41 & 17/20 & 15 & 16 \\ 
   & p=30 & 5/3/24 & 25 & 25 & 69/68 & 32/22 & 15 & 20 \\ 
   & p=60 & 4/2/21 & - & 23 & 95/94 & 53/23 & 16 & 19 \\ 
   & p=180 & 34/1/19 & - & 21 & 129/130 & 139/27 & 18 & 17 \\ 
  \midrule\multirow{4}[2]{*}{msnr} & p=14 & 17/14/20 & 21 & 20 & 42/41 & 23/23 & 16 & 17 \\ 
   & p=30 & 17/13/24 & 25 & 25 & 69/68 & 48/27 & 16 & 21 \\ 
   & p=60 & 13/9/21 & - & 23 & 95/94 & 84/28 & 16 & 23 \\ 
   & p=180 & 49/10/20 & - & 22 & 129/130 & 224/32 & 18 & 29 \\ 
  \midrule\multirow{4}[2]{*}{lsnr} & p=14 & 21/22/24 & 25 & 24 & 7/7 & 14/14 & 16 & 12 \\ 
   & p=30 & 29/31/26 & 25 & 26 & 1/1 & 13/8 & 8 & 6 \\ 
   & p=60 & 26/28/23 & - & 22 & 0/0 & 25/5 & 6 & 5 \\ 
   & p=180 & 31/18/15 & - & 16 & -2/0 & 85/4 & 3 & 3 \\ 
   \midrule 
 \multicolumn{1}{|c}{} &       & \multicolumn{7}{c|}{Relative efficiency} \\
\midrule\multirow{4}[2]{*}{hsnr} & p=14 & 0.98/1/0.89 & 0.88 & 0.89 & 0.75/0.75 & 0.91/0.89 & 0.92 & 0.91 \\ 
   & p=30 & 0.98/1/0.83 & 0.82 & 0.82 & 0.61/0.61 & 0.78/0.84 & 0.9 & 0.86 \\ 
   & p=60 & 0.99/1/0.85 & - & 0.83 & 0.52/0.53 & 0.67/0.83 & 0.88 & 0.86 \\ 
   & p=180 & 0.75/1/0.85 & - & 0.84 & 0.44/0.44 & 0.42/0.8 & 0.86 & 0.86 \\ 
  \midrule\multirow{4}[2]{*}{msnr} & p=14 & 0.98/1/0.95 & 0.94 & 0.95 & 0.8/0.81 & 0.93/0.93 & 0.98 & 0.98 \\ 
   & p=30 & 0.96/1/0.91 & 0.9 & 0.9 & 0.67/0.67 & 0.76/0.89 & 0.97 & 0.93 \\ 
   & p=60 & 0.97/1/0.9 & - & 0.89 & 0.56/0.56 & 0.59/0.85 & 0.94 & 0.89 \\ 
   & p=180 & 0.74/1/0.92 & - & 0.9 & 0.48/0.48 & 0.34/0.83 & 0.93 & 0.85 \\ 
  \midrule\multirow{4}[2]{*}{lsnr} & p=14 & 0.89/0.88/0.86 & 0.86 & 0.86 & 1/1 & 0.94/0.94 & 0.92 & 0.95 \\ 
   & p=30 & 0.78/0.77/0.8 & 0.8 & 0.8 & 1/1 & 0.89/0.93 & 0.93 & 0.95 \\ 
   & p=60 & 0.79/0.78/0.81 & - & 0.81 & 1/0.99 & 0.8/0.95 & 0.94 & 0.95 \\ 
   & p=180 & 0.75/0.83/0.85 & - & 0.85 & 1/0.98 & 0.53/0.94 & 0.95 & 0.95 \\ 
   \midrule 
 \multicolumn{1}{|c}{} &       & \multicolumn{7}{c|}{Sparsistency (number of extra variables)} \\
\midrule\multirow{4}[2]{*}{hsnr} & p=14 & 6(0.3)/6(0.2)/6(0.6) & 6(0.7) & 6(0.6) & 6(3.6)/6(4.4) & 6(1)/6(1.4) & 6(0.6) & 6(0.6) \\ 
   & p=30 & 6(0.1)/6(0)/6(0.6) & 6(0.7) & 6(0.7) & 6(7.5)/6(8.4) & 6(2.4)/6(1.7) & 6(1.1) & 6(0.8) \\ 
   & p=60 & 6(0.1)/6(0)/6(0.5) & - & 6(0.5) & 6(11.4)/6(13.2) & 6(4.8)/6(1.6) & 6(1.6) & 6(0.7) \\ 
   & p=180 & 6(8.8)/6(0)/6(0.4) & - & 6(0.4) & 6(15.9)/6(22.1) & 6(18.5)/6(2.2) & 6(2.6) & 6(0.7) \\ 
  \midrule\multirow{4}[2]{*}{msnr} & p=14 & 6(0.8)/6(0.6)/6(0.6) & 6(0.7) & 6(0.6) & 6(3.6)/6(4.4) & 6(1.1)/6(1.4) & 6(0.6) & 6(0.6) \\ 
   & p=30 & 6(0.5)/6(0.2)/6(0.6) & 6(0.7) & 6(0.7) & 6(7.4)/6(8.5) & 6(2.9)/6(1.6) & 6(0.8) & 6(0.9) \\ 
   & p=60 & 6(0.2)/6(0.1)/6(0.5) & - & 6(0.5) & 6(11.3)/6(13.2) & 6(6.4)/6(1.5) & 6(1) & 6(0.9) \\ 
   & p=180 & 6(9.8)/6(0.1)/6(0.4) & - & 6(0.4) & 6(15.9)/6(22) & 6(27.5)/6(2) & 6(1.6) & 6(1.3) \\ 
  \midrule\multirow{4}[2]{*}{lsnr} & p=14 & 5.5(4.5)/5.4(4.2)/4.8(1.8) & 4.8(1.7) & 4.8(1.8) & 5.7(3.3)/5.7(4) & 5.1(1.5)/5.3(2.9) & 5.2(2.8) & 5.2(2) \\ 
   & p=30 & 3.9(4.3)/3.1(1.8)/3.6(2.2) & 3.6(2) & 3.6(2.1) & 5.3(6.6)/5.3(7.1) & 4.8(3.8)/4.7(4.9) & 4.8(5.3) & 4.5(3.6) \\ 
   & p=60 & 2.3(1.8)/1.4(0.3)/2.7(1.4) & - & 2.7(1.4) & 4.7(8.8)/4.7(9.9) & 4.9(8.6)/4.2(6.9) & 4.3(7.5) & 4(5.5) \\ 
   & p=180 & 1.7(14.1)/0.5(0.1)/1.4(0.6) & - & 1.4(0.7) & 3.7(10.7)/3.8(13.5) & 4.6(36.7)/3.2(9.4) & 3.3(10.3) & 3.1(8) \\ 
   \bottomrule 
\end{tabular}
}
\end{table}

% latex table generated in R 3.6.2 by xtable 1.8-4 package
% Sat Dec 21 23:35:38 2019
\begin{table}[ht]
\centering
\caption{The performance of BOSS compared to other methods, Sparse-Ex2, $\rho$=0, n=2000} 
\scalebox{0.7}{
\begin{tabular}{|c|c|ccccccc|}
  \toprule 
 \multicolumn{1}{|c}{} &       & BOSS  & BS    & FS    & LASSO & Gamma LASSO & SparseNet & \multicolumn{1}{c|}{rLASSO} \\
 \multicolumn{1}{|c}{} &       & C$_p$-hdf/AICc-hdf/CV & CV    & CV    & AICc/CV & AICc/CV & CV    & \multicolumn{1}{c|}{CV}  \\
 \cmidrule{3-9}\multicolumn{1}{|c}{} &       & \multicolumn{7}{c|}{\% worse than the best possible BOSS}  \\
 \midrule\multirow{4}[2]{*}{hsnr} & p=14 & 6/6/17 & 17 & 17 & 40/41 & 12/14 & 11 & 13 \\ 
   & p=30 & 3/3/22 & 22 & 23 & 74/71 & 20/18 & 13 & 16 \\ 
   & p=60 & 2/2/24 & - & 23 & 97/93 & 29/19 & 15 & 18 \\ 
   & p=180 & 1/1/21 & - & 21 & 131/128 & 52/18 & 13 & 16 \\ 
  \midrule\multirow{4}[2]{*}{msnr} & p=14 & 6/6/17 & 17 & 17 & 41/41 & 14/17 & 11 & 13 \\ 
   & p=30 & 3/3/22 & 22 & 23 & 74/71 & 27/20 & 13 & 16 \\ 
   & p=60 & 2/2/24 & - & 23 & 97/93 & 44/21 & 15 & 18 \\ 
   & p=180 & 1/1/21 & - & 21 & 131/127 & 97/22 & 13 & 16 \\ 
  \midrule\multirow{4}[2]{*}{lsnr} & p=14 & 9/8/17 & 17 & 17 & 41/41 & 21/21 & 12 & 13 \\ 
   & p=30 & 5/5/22 & 22 & 23 & 74/71 & 46/25 & 13 & 16 \\ 
   & p=60 & 5/4/24 & - & 23 & 97/93 & 82/26 & 15 & 18 \\ 
   & p=180 & 5/5/21 & - & 21 & 131/128 & 192/26 & 12 & 17 \\ 
   \midrule 
 \multicolumn{1}{|c}{} &       & \multicolumn{7}{c|}{Relative efficiency} \\
\midrule\multirow{4}[2]{*}{hsnr} & p=14 & 1/1/0.91 & 0.91 & 0.91 & 0.76/0.76 & 0.95/0.93 & 0.96 & 0.94 \\ 
   & p=30 & 1/1/0.84 & 0.84 & 0.84 & 0.59/0.6 & 0.86/0.87 & 0.91 & 0.89 \\ 
   & p=60 & 1/1/0.83 & - & 0.83 & 0.52/0.53 & 0.79/0.86 & 0.89 & 0.86 \\ 
   & p=180 & 1/1/0.83 & - & 0.84 & 0.44/0.44 & 0.66/0.86 & 0.89 & 0.87 \\ 
  \midrule\multirow{4}[2]{*}{msnr} & p=14 & 1/1/0.91 & 0.91 & 0.91 & 0.75/0.75 & 0.93/0.91 & 0.96 & 0.94 \\ 
   & p=30 & 1/1/0.84 & 0.84 & 0.84 & 0.59/0.6 & 0.81/0.86 & 0.91 & 0.89 \\ 
   & p=60 & 1/1/0.83 & - & 0.83 & 0.52/0.53 & 0.71/0.84 & 0.89 & 0.86 \\ 
   & p=180 & 1/1/0.83 & - & 0.84 & 0.44/0.44 & 0.51/0.83 & 0.9 & 0.87 \\ 
  \midrule\multirow{4}[2]{*}{lsnr} & p=14 & 1/1/0.93 & 0.93 & 0.93 & 0.77/0.77 & 0.9/0.9 & 0.97 & 0.96 \\ 
   & p=30 & 1/1/0.86 & 0.86 & 0.85 & 0.6/0.61 & 0.72/0.84 & 0.93 & 0.91 \\ 
   & p=60 & 1/1/0.84 & - & 0.85 & 0.53/0.54 & 0.57/0.83 & 0.91 & 0.89 \\ 
   & p=180 & 1/1/0.86 & - & 0.86 & 0.45/0.46 & 0.36/0.83 & 0.94 & 0.9 \\ 
   \midrule 
 \multicolumn{1}{|c}{} &       & \multicolumn{7}{c|}{Sparsistency (number of extra variables)} \\
\midrule\multirow{4}[2]{*}{hsnr} & p=14 & 6(0.3)/6(0.3)/6(0.6) & 6(0.6) & 6(0.6) & 6(3.8)/6(4.5) & 6(0.9)/6(1.3) & 6(0.6) & 6(0.5) \\ 
   & p=30 & 6(0.1)/6(0.1)/6(0.6) & 6(0.6) & 6(0.6) & 6(8.4)/6(8.7) & 6(2)/6(1.8) & 6(0.9) & 6(0.5) \\ 
   & p=60 & 6(0)/6(0)/6(0.5) & - & 6(0.5) & 6(13.1)/6(12.2) & 6(3.8)/6(2.1) & 6(1.5) & 6(0.6) \\ 
   & p=180 & 6(0)/6(0)/6(0.3) & - & 6(0.3) & 6(21.6)/6(19.3) & 6(10.3)/6(2.4) & 6(2.1) & 6(0.5) \\ 
  \midrule\multirow{4}[2]{*}{msnr} & p=14 & 6(0.3)/6(0.3)/6(0.6) & 6(0.6) & 6(0.6) & 6(3.8)/6(4.5) & 6(0.9)/6(1.3) & 6(0.6) & 6(0.5) \\ 
   & p=30 & 6(0.1)/6(0.1)/6(0.6) & 6(0.6) & 6(0.6) & 6(8.5)/6(8.7) & 6(2.2)/6(1.7) & 6(0.9) & 6(0.5) \\ 
   & p=60 & 6(0)/6(0)/6(0.5) & - & 6(0.5) & 6(13.1)/6(12.2) & 6(4.8)/6(1.8) & 6(1.4) & 6(0.6) \\ 
   & p=180 & 6(0)/6(0)/6(0.3) & - & 6(0.3) & 6(21.8)/6(19.4) & 6(15.6)/6(1.9) & 6(1.9) & 6(0.5) \\ 
  \midrule\multirow{4}[2]{*}{lsnr} & p=14 & 6(0.4)/6(0.4)/6(0.6) & 6(0.6) & 6(0.6) & 6(3.8)/6(4.5) & 6(1.1)/6(1.4) & 6(0.5) & 6(0.5) \\ 
   & p=30 & 6(0.1)/6(0.1)/6(0.6) & 6(0.6) & 6(0.6) & 6(8.4)/6(8.7) & 6(3)/6(1.6) & 6(0.7) & 6(0.5) \\ 
   & p=60 & 6(0.1)/6(0.1)/6(0.5) & - & 6(0.5) & 6(13.1)/6(12.2) & 6(6.8)/6(1.4) & 6(1) & 6(0.6) \\ 
   & p=180 & 6(0.1)/6(0.1)/6(0.3) & - & 6(0.3) & 6(21.7)/6(19.4) & 6(23.9)/6(1.1) & 6(0.9) & 6(0.5) \\ 
   \bottomrule 
\end{tabular}
}
\end{table}

% latex table generated in R 3.6.2 by xtable 1.8-4 package
% Sat Dec 21 23:35:39 2019
\begin{table}[ht]
\centering
\caption{The performance of BOSS compared to other methods, Sparse-Ex2, $\rho$=0.5, n=200} 
\scalebox{0.7}{
\begin{tabular}{|c|c|ccccccc|}
  \toprule 
 \multicolumn{1}{|c}{} &       & BOSS  & BS    & FS    & LASSO & Gamma LASSO & SparseNet & \multicolumn{1}{c|}{rLASSO} \\
 \multicolumn{1}{|c}{} &       & C$_p$-hdf/AICc-hdf/CV & CV    & CV    & AICc/CV & AICc/CV & CV    & \multicolumn{1}{c|}{CV}  \\
 \cmidrule{3-9}\multicolumn{1}{|c}{} &       & \multicolumn{7}{c|}{\% worse than the best possible BOSS}  \\
 \midrule\multirow{4}[2]{*}{hsnr} & p=14 & 8/6/17 & 18 & 17 & 46/44 & 15/20 & 13 & 15 \\ 
   & p=30 & 5/3/24 & 24 & 24 & 91/88 & 28/23 & 14 & 18 \\ 
   & p=60 & 4/2/22 & - & 23 & 123/119 & 43/24 & 16 & 17 \\ 
   & p=180 & 34/1/19 & - & 22 & 168/165 & 100/28 & 17 & 15 \\ 
  \midrule\multirow{4}[2]{*}{msnr} & p=14 & 19/16/26 & 18 & 17 & 46/44 & 21/23 & 14 & 16 \\ 
   & p=30 & 22/18/29 & 23 & 25 & 89/87 & 42/28 & 14 & 41 \\ 
   & p=60 & 16/11/29 & - & 25 & 121/117 & 70/29 & 16 & 42 \\ 
   & p=180 & 48/9/29 & - & 30 & 160/157 & 179/32 & 14 & 52 \\ 
  \midrule\multirow{4}[2]{*}{lsnr} & p=14 & 24/25/28 & 23 & 26 & 18/17 & 15/20 & 22 & 23 \\ 
   & p=30 & 33/34/26 & 20 & 30 & 17/17 & 15/19 & 18 & 22 \\ 
   & p=60 & 28/29/23 & - & 28 & 15/16 & 23/17 & 16 & 19 \\ 
   & p=180 & 28/14/14 & - & 17 & 8/9 & 71/11 & 10 & 12 \\ 
   \midrule 
 \multicolumn{1}{|c}{} &       & \multicolumn{7}{c|}{Relative efficiency} \\
\midrule\multirow{4}[2]{*}{hsnr} & p=14 & 0.98/1/0.91 & 0.9 & 0.91 & 0.73/0.74 & 0.92/0.89 & 0.94 & 0.92 \\ 
   & p=30 & 0.98/1/0.83 & 0.83 & 0.83 & 0.54/0.55 & 0.8/0.84 & 0.9 & 0.87 \\ 
   & p=60 & 0.98/1/0.83 & - & 0.83 & 0.46/0.46 & 0.71/0.82 & 0.88 & 0.87 \\ 
   & p=180 & 0.75/1/0.85 & - & 0.83 & 0.38/0.38 & 0.5/0.79 & 0.86 & 0.88 \\ 
  \midrule\multirow{4}[2]{*}{msnr} & p=14 & 0.96/0.98/0.9 & 0.97 & 0.97 & 0.78/0.79 & 0.94/0.92 & 1 & 0.98 \\ 
   & p=30 & 0.93/0.97/0.88 & 0.92 & 0.91 & 0.6/0.61 & 0.8/0.89 & 1 & 0.81 \\ 
   & p=60 & 0.96/1/0.86 & - & 0.89 & 0.5/0.51 & 0.66/0.87 & 0.96 & 0.78 \\ 
   & p=180 & 0.74/1/0.84 & - & 0.83 & 0.42/0.42 & 0.39/0.83 & 0.95 & 0.71 \\ 
  \midrule\multirow{4}[2]{*}{lsnr} & p=14 & 0.92/0.92/0.9 & 0.94 & 0.91 & 0.97/0.98 & 1/0.96 & 0.94 & 0.94 \\ 
   & p=30 & 0.87/0.86/0.92 & 0.96 & 0.88 & 0.99/0.98 & 1/0.96 & 0.98 & 0.94 \\ 
   & p=60 & 0.89/0.89/0.93 & - & 0.89 & 1/0.99 & 0.93/0.98 & 0.99 & 0.96 \\ 
   & p=180 & 0.84/0.94/0.94 & - & 0.92 & 1/0.99 & 0.63/0.97 & 0.98 & 0.96 \\ 
   \midrule 
 \multicolumn{1}{|c}{} &       & \multicolumn{7}{c|}{Sparsistency (number of extra variables)} \\
\midrule\multirow{4}[2]{*}{hsnr} & p=14 & 6(0.4)/6(0.3)/6(0.6) & 6(0.6) & 6(0.6) & 6(4.6)/6(5.5) & 6(0.9)/6(1.5) & 6(0.7) & 6(0.6) \\ 
   & p=30 & 6(0.1)/6(0)/6(0.7) & 6(0.7) & 6(0.7) & 6(11.2)/6(12.8) & 6(2.2)/6(1.5) & 6(1.1) & 6(0.7) \\ 
   & p=60 & 6(0.1)/6(0)/6(0.5) & - & 6(0.5) & 6(16)/6(19.1) & 6(4.3)/6(1.5) & 6(1.7) & 6(0.6) \\ 
   & p=180 & 6(9.3)/6(0)/6(0.4) & - & 6(0.4) & 6(22.5)/6(31.5) & 6(14.2)/6(2.1) & 6(2.8) & 6(0.6) \\ 
  \midrule\multirow{4}[2]{*}{msnr} & p=14 & 6(0.9)/6(0.8)/6(0.6) & 6(0.6) & 6(0.6) & 6(4.6)/6(5.5) & 6(1.1)/6(1.6) & 6(0.7) & 6(0.6) \\ 
   & p=30 & 6(0.6)/6(0.4)/6(0.7) & 6(0.7) & 6(0.7) & 6(11.2)/6(12.8) & 6(2.7)/6(1.5) & 6(0.9) & 6(2) \\ 
   & p=60 & 6(0.3)/6(0.2)/6(0.5) & - & 6(0.6) & 6(16)/6(19.1) & 6(5.6)/6(1.3) & 6(1.4) & 6(1.9) \\ 
   & p=180 & 6(10.1)/6(0.1)/6(0.5) & - & 6(0.7) & 6(22.3)/6(31.6) & 6(23.9)/6(2) & 6(2.1) & 6(2.5) \\ 
  \midrule\multirow{4}[2]{*}{lsnr} & p=14 & 5.7(4.8)/5.6(4.5)/5.1(1.6) & 5.1(1.4) & 5.1(1.9) & 5.7(4.3)/5.7(5.1) & 5.4(1.6)/5.5(3.7) & 5.3(2.7) & 5.4(3.2) \\ 
   & p=30 & 3.6(5.3)/2.4(2.2)/3.9(2.4) & 3.9(1.9) & 3.5(2.8) & 4.5(8)/4.4(8.9) & 5(4.7)/4.3(6.5) & 4.2(5.3) & 3.9(6.1) \\ 
   & p=60 & 2.3(2.4)/1(0.3)/3.1(2.1) & - & 2.5(1.8) & 3.8(9.8)/3.7(10.9) & 5(9)/3.7(8.1) & 3.7(7) & 3.3(7.8) \\ 
   & p=180 & 1.7(14.9)/0.4(0.1)/1.4(1.1) & - & 0.9(0.8) & 2.4(9.8)/2.4(12.9) & 4.5(35.8)/2.3(11) & 2.3(9.3) & 2.1(9.1) \\ 
   \bottomrule 
\end{tabular}
}
\end{table}

% latex table generated in R 3.6.1 by xtable 1.8-4 package
% Sat Nov  9 19:29:40 2019
\begin{table}[ht]
\centering
\caption{The performance of BOSS compared to other methods, Sparse-Ex2, $\rho$=0.5, n=2000} 

\scalebox{0.7}{
\begin{tabular}{|c|c|ccccccc|}
  \toprule 
 \multicolumn{1}{|c}{} &       & BOSS  & BS    & FS    & LASSO & Gamma LASSO & SparseNet & \multicolumn{1}{c|}{rLASSO} \\
 \multicolumn{1}{|c}{} &       & C$_p$-hdf/AICc-hdf/CV & CV    & CV    & AICc/CV & AICc/CV & CV    & \multicolumn{1}{c|}{CV}  \\
 \cmidrule{3-9}\multicolumn{1}{|c}{} &       & \multicolumn{7}{c|}{\% worse than the best possible BOSS}  \\
 \midrule\multirow{4}[2]{*}{hsnr} & p=14 & 7/7/18 & 19 & 18 & 48/47 & 14/18 & 13 & 18 \\ 
   & p=30 & 4/3/23 & 22 & 23 & 86/83 & 20/18 & 12 & 23 \\ 
   & p=60 & 2/2/23 & - & 23 & 124/120 & 28/20 & 14 & 24 \\ 
   & p=180 & 1/1/21 & - & 21 & 174/171 & 46/20 & 14 & 24 \\ 
  \midrule\multirow{4}[2]{*}{msnr} & p=14 & 7/7/18 & 19 & 18 & 49/48 & 16/19 & 13 & 18 \\ 
   & p=30 & 4/3/23 & 22 & 23 & 85/83 & 25/20 & 12 & 23 \\ 
   & p=60 & 2/2/23 & - & 23 & 125/120 & 39/22 & 14 & 24 \\ 
   & p=180 & 1/1/21 & - & 21 & 174/171 & 77/23 & 13 & 24 \\ 
  \midrule\multirow{4}[2]{*}{lsnr} & p=14 & 12/12/23 & 19 & 18 & 50/48 & 21/23 & 13 & 18 \\ 
   & p=30 & 9/8/27 & 22 & 23 & 85/83 & 40/25 & 13 & 23 \\ 
   & p=60 & 8/8/26 & - & 23 & 124/120 & 72/28 & 14 & 24 \\ 
   & p=180 & 10/9/24 & - & 21 & 175/170 & 163/29 & 12 & 25 \\ 
   \midrule 
 \multicolumn{1}{|c}{} &       & \multicolumn{7}{c|}{Relative efficiency} \\
\midrule\multirow{4}[2]{*}{hsnr} & p=14 & 1/1/0.91 & 0.9 & 0.91 & 0.72/0.72 & 0.94/0.91 & 0.94 & 0.9 \\ 
   & p=30 & 1/1/0.84 & 0.85 & 0.84 & 0.56/0.56 & 0.86/0.87 & 0.92 & 0.84 \\ 
   & p=60 & 1/1/0.83 & - & 0.83 & 0.45/0.46 & 0.79/0.85 & 0.89 & 0.82 \\ 
   & p=180 & 1/1/0.84 & - & 0.84 & 0.37/0.37 & 0.69/0.84 & 0.89 & 0.81 \\ 
  \midrule\multirow{4}[2]{*}{msnr} & p=14 & 1/1/0.91 & 0.9 & 0.91 & 0.72/0.72 & 0.92/0.89 & 0.94 & 0.9 \\ 
   & p=30 & 1/1/0.84 & 0.85 & 0.84 & 0.56/0.56 & 0.83/0.86 & 0.92 & 0.84 \\ 
   & p=60 & 1/1/0.83 & - & 0.83 & 0.45/0.46 & 0.73/0.83 & 0.89 & 0.82 \\ 
   & p=180 & 1/1/0.84 & - & 0.84 & 0.37/0.37 & 0.57/0.82 & 0.89 & 0.81 \\ 
  \midrule\multirow{4}[2]{*}{lsnr} & p=14 & 1/1/0.91 & 0.94 & 0.95 & 0.75/0.76 & 0.92/0.91 & 0.99 & 0.95 \\ 
   & p=30 & 1/1/0.85 & 0.89 & 0.88 & 0.58/0.59 & 0.77/0.87 & 0.96 & 0.88 \\ 
   & p=60 & 1/1/0.85 & - & 0.88 & 0.48/0.49 & 0.63/0.84 & 0.95 & 0.87 \\ 
   & p=180 & 0.99/1/0.88 & - & 0.9 & 0.4/0.4 & 0.41/0.85 & 0.97 & 0.87 \\ 
   \midrule 
 \multicolumn{1}{|c}{} &       & \multicolumn{7}{c|}{Sparsistency (number of extra variables)} \\
\midrule\multirow{4}[2]{*}{hsnr} & p=14 & 6(0.3)/6(0.3)/6(0.6) & 6(0.7) & 6(0.6) & 6(4.7)/6(5.5) & 6(1)/6(1.4) & 6(0.7) & 6(0.7) \\ 
   & p=30 & 6(0.1)/6(0.1)/6(0.6) & 6(0.6) & 6(0.6) & 6(10.9)/6(11.3) & 6(2.1)/6(1.9) & 6(0.9) & 6(0.7) \\ 
   & p=60 & 6(0)/6(0)/6(0.5) & - & 6(0.5) & 6(18.1)/6(17.7) & 6(3.8)/6(2.1) & 6(1.4) & 6(0.7) \\ 
   & p=180 & 6(0)/6(0)/6(0.3) & - & 6(0.3) & 6(32.2)/6(29.7) & 6(9.2)/6(2.4) & 6(2.1) & 6(0.5) \\ 
  \midrule\multirow{4}[2]{*}{msnr} & p=14 & 6(0.3)/6(0.3)/6(0.6) & 6(0.7) & 6(0.6) & 6(4.8)/6(5.5) & 6(1)/6(1.5) & 6(0.6) & 6(0.7) \\ 
   & p=30 & 6(0.1)/6(0.1)/6(0.6) & 6(0.6) & 6(0.6) & 6(11)/6(11.3) & 6(2.3)/6(1.7) & 6(0.9) & 6(0.7) \\ 
   & p=60 & 6(0)/6(0)/6(0.5) & - & 6(0.5) & 6(18.1)/6(17.7) & 6(4.6)/6(1.8) & 6(1.4) & 6(0.7) \\ 
   & p=180 & 6(0)/6(0)/6(0.3) & - & 6(0.3) & 6(32.3)/6(29.7) & 6(13.4)/6(1.8) & 6(2) & 6(0.5) \\ 
  \midrule\multirow{4}[2]{*}{lsnr} & p=14 & 6(0.6)/6(0.5)/6(0.6) & 6(0.7) & 6(0.6) & 6(4.8)/6(5.5) & 6(1.1)/6(1.5) & 6(0.6) & 6(0.7) \\ 
   & p=30 & 6(0.2)/6(0.2)/6(0.6) & 6(0.6) & 6(0.6) & 6(10.9)/6(11.3) & 6(2.8)/6(1.5) & 6(0.8) & 6(0.7) \\ 
   & p=60 & 6(0.1)/6(0.1)/6(0.5) & - & 6(0.5) & 6(18.1)/6(17.8) & 6(6.4)/6(1.4) & 6(1.1) & 6(0.7) \\ 
   & p=180 & 6(0.1)/6(0.1)/6(0.3) & - & 6(0.3) & 6(32.5)/6(29.7) & 6(21.7)/6(1.1) & 6(1.2) & 6(0.5) \\ 
   \bottomrule 
\end{tabular}
}
\end{table}

% latex table generated in R 3.6.1 by xtable 1.8-4 package
% Sat Nov  9 19:29:41 2019
\begin{table}[ht]
\centering
\caption{The performance of BOSS compared to other methods, Sparse-Ex2, $\rho$=0.9, n=200} 

\scalebox{0.7}{
\begin{tabular}{|c|c|ccccccc|}
  \toprule 
 \multicolumn{1}{|c}{} &       & BOSS  & BS    & FS    & LASSO & Gamma LASSO & SparseNet & \multicolumn{1}{c|}{rLASSO} \\
 \multicolumn{1}{|c}{} &       & C$_p$-hdf/AICc-hdf/CV & CV    & CV    & AICc/CV & AICc/CV & CV    & \multicolumn{1}{c|}{CV}  \\
 \cmidrule{3-9}\multicolumn{1}{|c}{} &       & \multicolumn{7}{c|}{\% worse than the best possible BOSS}  \\
 \midrule\multirow{4}[2]{*}{hsnr} & p=14 & 18/15/25 & 17 & 23 & 53/51 & 20/27 & 14 & 29 \\ 
   & p=30 & 12/9/42 & 9 & 53 & 91/89 & 18/18 & 1 & 57 \\ 
   & p=60 & 9/6/49 & - & 69 & 131/126 & 16/12 & -6 & 74 \\ 
   & p=180 & 22/5/75 & - & 111 & 138/120 & -7/-16 & -30 & 70 \\ 
  \midrule\multirow{4}[2]{*}{msnr} & p=14 & 25/23/29 & 3 & 24 & 35/34 & 10/16 & 3 & 27 \\ 
   & p=30 & 23/20/31 & -15 & 36 & 50/48 & -1/2 & -17 & 40 \\ 
   & p=60 & 21/18/30 & - & 48 & 65/61 & -7/-7 & -26 & 52 \\ 
   & p=180 & 23/22/20 & - & 64 & 48/35 & -16/-1 & -24 & 35 \\ 
  \midrule\multirow{4}[2]{*}{lsnr} & p=14 & 40/41/31 & 32 & 50 & 41/40 & 38/39 & 39 & 46 \\ 
   & p=30 & 42/42/28 & 28 & 61 & 53/53 & 44/51 & 46 & 57 \\ 
   & p=60 & 36/35/25 & - & 50 & 45/45 & 52/47 & 45 & 49 \\ 
   & p=180 & 25/9/11 & - & 15 & 10/11 & 59/12 & 12 & 14 \\ 
   \midrule 
 \multicolumn{1}{|c}{} &       & \multicolumn{7}{c|}{Relative efficiency} \\
\midrule\multirow{4}[2]{*}{hsnr} & p=14 & 0.97/0.99/0.91 & 0.97 & 0.93 & 0.75/0.75 & 0.95/0.89 & 1 & 0.89 \\ 
   & p=30 & 0.9/0.92/0.71 & 0.93 & 0.66 & 0.53/0.53 & 0.86/0.85 & 1 & 0.64 \\ 
   & p=60 & 0.87/0.89/0.63 & - & 0.56 & 0.41/0.42 & 0.81/0.84 & 1 & 0.54 \\ 
   & p=180 & 0.58/0.67/0.4 & - & 0.33 & 0.3/0.32 & 0.76/0.84 & 1 & 0.41 \\ 
  \midrule\multirow{4}[2]{*}{msnr} & p=14 & 0.82/0.83/0.79 & 0.99 & 0.83 & 0.76/0.77 & 0.94/0.88 & 1 & 0.81 \\ 
   & p=30 & 0.68/0.69/0.63 & 0.98 & 0.61 & 0.55/0.56 & 0.84/0.82 & 1 & 0.59 \\ 
   & p=60 & 0.61/0.62/0.57 & - & 0.5 & 0.45/0.46 & 0.79/0.79 & 1 & 0.49 \\ 
   & p=180 & 0.62/0.62/0.63 & - & 0.46 & 0.51/0.56 & 0.9/0.76 & 1 & 0.56 \\ 
  \midrule\multirow{4}[2]{*}{lsnr} & p=14 & 0.94/0.93/1 & 0.99 & 0.87 & 0.92/0.93 & 0.95/0.94 & 0.94 & 0.9 \\ 
   & p=30 & 0.9/0.9/1 & 1 & 0.8 & 0.83/0.84 & 0.89/0.85 & 0.88 & 0.81 \\ 
   & p=60 & 0.92/0.93/1 & - & 0.83 & 0.86/0.86 & 0.83/0.85 & 0.87 & 0.84 \\ 
   & p=180 & 0.87/1/0.98 & - & 0.95 & 0.98/0.98 & 0.68/0.97 & 0.97 & 0.96 \\ 
   \midrule 
 \multicolumn{1}{|c}{} &       & \multicolumn{7}{c|}{Sparsistency (number of extra variables)} \\
\midrule\multirow{4}[2]{*}{hsnr} & p=14 & 6(0.9)/6(0.7)/6(0.7) & 6(0.6) & 6(1) & 6(6.5)/6(7.2) & 6(1.3)/6(2.2) & 6(0.7) & 6(1.6) \\ 
   & p=30 & 6(1.1)/6(0.9)/6(2.5) & 6(0.7) & 6(3.7) & 6(18.3)/6(19.9) & 6(2.9)/6(2.5) & 6(1.1) & 6(7.6) \\ 
   & p=60 & 6(1.7)/6(1.6)/6(3.3) & - & 6(4.9) & 6(31.8)/6(36.4) & 6(4.2)/6(2.7) & 6(1.8) & 6(10.6) \\ 
   & p=180 & 6(14.9)/6(6.8)/6(11.8) & - & 5.9(19.7) & 6(51.3)/6(76.1) & 6(8.6)/6(4.1) & 6(3.2) & 6(23.4) \\ 
  \midrule\multirow{4}[2]{*}{msnr} & p=14 & 6(3)/6(2.7)/6(1.2) & 6(0.6) & 6(2.4) & 6(6.5)/6(7.2) & 6(1.5)/6(3) & 6(1.3) & 6(3.2) \\ 
   & p=30 & 6(5.1)/6(4)/6(3.5) & 6(0.7) & 6(7) & 6(18.2)/6(19.9) & 6(3.6)/6(5.3) & 6(3.4) & 6(11.7) \\ 
   & p=60 & 6(6.1)/6(4.7)/6(5.1) & - & 6(10.3) & 6(31.8)/6(36.3) & 6(6.5)/6(7.7) & 6(6) & 6(19.4) \\ 
   & p=180 & 5.9(38.2)/5.5(14.8)/5.8(18.2) & - & 4.2(16.8) & 5.8(49.6)/6(75.8) & 6(23.5)/5.9(36.2) & 5.9(31.7) & 5.9(49.1) \\ 
  \midrule\multirow{4}[2]{*}{lsnr} & p=14 & 5.7(4.9)/5.6(4.6)/5.6(2.7) & 5.2(1.2) & 5.3(4) & 5.8(6.3)/5.8(6.9) & 5.4(3.7)/5.7(5.9) & 5.4(4.7) & 5.6(5.8) \\ 
   & p=30 & 4(7.5)/2.9(4.4)/4.6(6.6) & 4.1(1.7) & 2.5(4.5) & 2.7(8.3)/2.8(9.2) & 4.4(9.7)/3.3(9.4) & 3.4(8.7) & 2.5(7.7) \\ 
   & p=60 & 2.6(7.1)/1.4(2.5)/3.8(9.8) & - & 0.8(1.7) & 0.8(4.4)/0.9(5.7) & 3.7(16)/1.3(7.2) & 1.8(9.1) & 0.8(4.6) \\ 
   & p=180 & 1.2(20.1)/0.3(0.5)/0.7(2.7) & - & 0.1(0.5) & 0.3(4.4)/0.3(4.6) & 2(35.8)/0.3(4.5) & 0.3(4.9) & 0.2(3.2) \\ 
   \bottomrule 
\end{tabular}
}
\end{table}

% latex table generated in R 3.6.2 by xtable 1.8-4 package
% Sat Dec 21 23:35:52 2019
\begin{table}[ht]
\centering
\caption{The performance of BOSS compared to other methods, Sparse-Ex2, $\rho$=0.9, n=2000} 
\scalebox{0.7}{
\begin{tabular}{|c|c|ccccccc|}
  \toprule 
 \multicolumn{1}{|c}{} &       & BOSS  & BS    & FS    & LASSO & Gamma LASSO & SparseNet & \multicolumn{1}{c|}{rLASSO} \\
 \multicolumn{1}{|c}{} &       & C$_p$-hdf/AICc-hdf/CV & CV    & CV    & AICc/CV & AICc/CV & CV    & \multicolumn{1}{c|}{CV}  \\
 \cmidrule{3-9}\multicolumn{1}{|c}{} &       & \multicolumn{7}{c|}{\% worse than the best possible BOSS}  \\
 \midrule\multirow{4}[2]{*}{hsnr} & p=14 & 7/7/18 & 19 & 18 & 56/55 & 22/28 & 14 & 15 \\ 
   & p=30 & 4/3/23 & 22 & 23 & 118/116 & 29/29 & 14 & 17 \\ 
   & p=60 & 2/2/23 & - & 23 & 186/182 & 37/33 & 15 & 15 \\ 
   & p=180 & 1/1/21 & - & 21 & 299/294 & 51/41 & 15 & 13 \\ 
  \midrule\multirow{4}[2]{*}{msnr} & p=14 & 8/8/18 & 19 & 18 & 56/55 & 23/29 & 14 & 14 \\ 
   & p=30 & 4/4/24 & 22 & 23 & 118/115 & 31/30 & 13 & 17 \\ 
   & p=60 & 3/3/23 & - & 23 & 185/182 & 42/36 & 16 & 15 \\ 
   & p=180 & 3/3/22 & - & 21 & 298/293 & 61/43 & 14 & 13 \\ 
  \midrule\multirow{4}[2]{*}{lsnr} & p=14 & 36/36/42 & 15 & 19 & 52/50 & 22/30 & 12 & 23 \\ 
   & p=30 & 37/36/43 & 12 & 32 & 100/97 & 28/26 & 7 & 40 \\ 
   & p=60 & 39/38/47 & - & 47 & 141/138 & 35/22 & 1 & 61 \\ 
   & p=180 & 38/38/37 & - & 72 & 178/175 & 45/16 & -12 & 100 \\ 
   \midrule 
 \multicolumn{1}{|c}{} &       & \multicolumn{7}{c|}{Relative efficiency} \\
\midrule\multirow{4}[2]{*}{hsnr} & p=14 & 1/1/0.91 & 0.9 & 0.91 & 0.68/0.69 & 0.88/0.84 & 0.94 & 0.93 \\ 
   & p=30 & 1/1/0.84 & 0.85 & 0.84 & 0.47/0.48 & 0.8/0.8 & 0.9 & 0.88 \\ 
   & p=60 & 1/1/0.83 & - & 0.83 & 0.35/0.36 & 0.74/0.76 & 0.88 & 0.88 \\ 
   & p=180 & 1/1/0.84 & - & 0.84 & 0.25/0.26 & 0.67/0.72 & 0.88 & 0.89 \\ 
  \midrule\multirow{4}[2]{*}{msnr} & p=14 & 1/1/0.91 & 0.91 & 0.92 & 0.69/0.7 & 0.88/0.83 & 0.94 & 0.94 \\ 
   & p=30 & 1/1/0.84 & 0.85 & 0.84 & 0.48/0.48 & 0.79/0.8 & 0.91 & 0.89 \\ 
   & p=60 & 1/1/0.84 & - & 0.84 & 0.36/0.36 & 0.72/0.76 & 0.88 & 0.9 \\ 
   & p=180 & 1/1/0.84 & - & 0.85 & 0.26/0.26 & 0.64/0.72 & 0.9 & 0.91 \\ 
  \midrule\multirow{4}[2]{*}{lsnr} & p=14 & 0.83/0.83/0.79 & 0.98 & 0.94 & 0.74/0.75 & 0.92/0.87 & 1 & 0.91 \\ 
   & p=30 & 0.79/0.79/0.75 & 0.96 & 0.81 & 0.54/0.54 & 0.84/0.85 & 1 & 0.76 \\ 
   & p=60 & 0.72/0.73/0.68 & - & 0.69 & 0.42/0.42 & 0.75/0.83 & 1 & 0.63 \\ 
   & p=180 & 0.64/0.64/0.64 & - & 0.51 & 0.32/0.32 & 0.61/0.76 & 1 & 0.44 \\ 
   \midrule 
 \multicolumn{1}{|c}{} &       & \multicolumn{7}{c|}{Sparsistency (number of extra variables)} \\
\midrule\multirow{4}[2]{*}{hsnr} & p=14 & 6(0.3)/6(0.3)/6(0.6) & 6(0.7) & 6(0.6) & 6(6.6)/6(7.2) & 6(1.5)/6(2.1) & 6(0.7) & 6(0.5) \\ 
   & p=30 & 6(0.1)/6(0.1)/6(0.6) & 6(0.6) & 6(0.6) & 6(17.8)/6(18.9) & 6(2.7)/6(2.6) & 6(0.9) & 6(0.5) \\ 
   & p=60 & 6(0)/6(0)/6(0.5) & - & 6(0.5) & 6(34.2)/6(35.3) & 6(4.5)/6(2.9) & 6(1.4) & 6(0.4) \\ 
   & p=180 & 6(0)/6(0)/6(0.3) & - & 6(0.3) & 6(72.8)/6(73.6) & 6(8.7)/6(3.9) & 6(2.2) & 6(0.2) \\ 
  \midrule\multirow{4}[2]{*}{msnr} & p=14 & 6(0.3)/6(0.3)/6(0.6) & 6(0.7) & 6(0.6) & 6(6.6)/6(7.2) & 6(1.5)/6(2.2) & 6(0.7) & 6(0.5) \\ 
   & p=30 & 6(0.1)/6(0.1)/6(0.6) & 6(0.6) & 6(0.6) & 6(17.8)/6(18.8) & 6(2.8)/6(2.3) & 6(0.9) & 6(0.5) \\ 
   & p=60 & 6(0)/6(0)/6(0.5) & - & 6(0.5) & 6(34.2)/6(35.2) & 6(4.8)/6(2.6) & 6(1.4) & 6(0.4) \\ 
   & p=180 & 6(0)/6(0)/6(0.3) & - & 6(0.3) & 6(72.8)/6(73.6) & 6(10.5)/6(3.3) & 6(2.1) & 6(0.2) \\ 
  \midrule\multirow{4}[2]{*}{lsnr} & p=14 & 6(2.7)/6(2.6)/6(0.6) & 6(0.7) & 6(0.9) & 6(6.6)/6(7.2) & 6(1.5)/6(2.5) & 6(0.8) & 6(1.3) \\ 
   & p=30 & 6(2.1)/6(2)/6(0.8) & 6(0.6) & 6(1.5) & 6(17.8)/6(18.8) & 6(3.2)/6(2.7) & 6(1.4) & 6(2.5) \\ 
   & p=60 & 6(1)/6(0.9)/6(0.9) & - & 6(2.2) & 6(34.2)/6(35.1) & 6(6.3)/6(3.7) & 6(2.6) & 6(3.9) \\ 
   & p=180 & 6(1.4)/6(1.4)/6(1.8) & - & 6(4.6) & 6(72.3)/6(73.5) & 6(17.1)/6(8.4) & 6(7.1) & 6(10.4) \\ 
   \bottomrule 
\end{tabular}
}
\end{table}

% latex table generated in R 3.6.1 by xtable 1.8-4 package
% Sat Nov  9 19:29:49 2019
\begin{table}[ht]
\centering
\caption{The performance of BOSS compared to other methods, Sparse-Ex3, $\rho$=0, n=200} 

\scalebox{0.7}{
\begin{tabular}{|c|c|ccccccc|}
  \toprule 
 \multicolumn{1}{|c}{} &       & BOSS  & BS    & FS    & lasso & gamma lasso & SparseNet & \multicolumn{1}{c|}{rlasso} \\
 \multicolumn{1}{|c}{} &       & C$_p$-hdf/AICc-hdf/CV & CV    & CV    & AICc/CV & AICc/CV & CV    & \multicolumn{1}{c|}{CV}  \\
 \cmidrule{3-9}\multicolumn{1}{|c}{} &       & \multicolumn{7}{c|}{\% worse than the best possible BOSS}  \\
 \midrule\multirow{4}[2]{*}{hsnr} & p=14 & 8/6/20 & 21 & 20 & 44/43 & 17/20 & 15 & 20 \\ 
   & p=30 & 5/3/24 & 25 & 25 & 69/67 & 32/23 & 15 & 25 \\ 
   & p=60 & 4/2/21 & - & 23 & 97/96 & 52/24 & 16 & 25 \\ 
   & p=180 & 34/1/19 & - & 21 & 133/133 & 137/28 & 19 & 23 \\ 
  \midrule\multirow{4}[2]{*}{msnr} & p=14 & 17/14/20 & 21 & 20 & 44/43 & 24/24 & 16 & 20 \\ 
   & p=30 & 18/13/24 & 25 & 25 & 69/67 & 48/27 & 16 & 25 \\ 
   & p=60 & 14/9/21 & - & 23 & 97/95 & 84/29 & 16 & 28 \\ 
   & p=180 & 50/11/20 & - & 22 & 132/133 & 224/33 & 19 & 36 \\ 
  \midrule\multirow{4}[2]{*}{lsnr} & p=14 & 22/23/26 & 26 & 26 & 8/8 & 13/15 & 17 & 16 \\ 
   & p=30 & 29/32/26 & 26 & 25 & 1/1 & 14/8 & 8 & 8 \\ 
   & p=60 & 27/29/22 & - & 22 & 0/1 & 24/6 & 6 & 7 \\ 
   & p=180 & 30/16/14 & - & 14 & -2/1 & 84/4 & 3 & 6 \\ 
   \midrule 
 \multicolumn{1}{|c}{} &       & \multicolumn{7}{c|}{Relative efficiency} \\
\midrule\multirow{4}[2]{*}{hsnr} & p=14 & 0.98/1/0.89 & 0.88 & 0.89 & 0.74/0.75 & 0.91/0.88 & 0.93 & 0.89 \\ 
   & p=30 & 0.98/1/0.83 & 0.82 & 0.82 & 0.61/0.61 & 0.78/0.84 & 0.89 & 0.83 \\ 
   & p=60 & 0.99/1/0.85 & - & 0.83 & 0.52/0.52 & 0.67/0.83 & 0.88 & 0.82 \\ 
   & p=180 & 0.75/1/0.85 & - & 0.84 & 0.43/0.43 & 0.43/0.79 & 0.85 & 0.82 \\ 
  \midrule\multirow{4}[2]{*}{msnr} & p=14 & 0.98/1/0.95 & 0.95 & 0.96 & 0.8/0.8 & 0.92/0.92 & 0.99 & 0.95 \\ 
   & p=30 & 0.96/1/0.92 & 0.91 & 0.91 & 0.67/0.68 & 0.76/0.89 & 0.98 & 0.9 \\ 
   & p=60 & 0.96/1/0.9 & - & 0.89 & 0.56/0.56 & 0.59/0.85 & 0.94 & 0.85 \\ 
   & p=180 & 0.74/1/0.92 & - & 0.91 & 0.48/0.47 & 0.34/0.83 & 0.93 & 0.81 \\ 
  \midrule\multirow{4}[2]{*}{lsnr} & p=14 & 0.89/0.88/0.86 & 0.86 & 0.86 & 1/1 & 0.95/0.94 & 0.92 & 0.93 \\ 
   & p=30 & 0.78/0.76/0.8 & 0.8 & 0.81 & 1/1 & 0.88/0.93 & 0.93 & 0.93 \\ 
   & p=60 & 0.79/0.78/0.82 & - & 0.82 & 1/1 & 0.81/0.95 & 0.94 & 0.93 \\ 
   & p=180 & 0.76/0.85/0.86 & - & 0.86 & 1/0.98 & 0.53/0.94 & 0.95 & 0.93 \\ 
   \midrule 
 \multicolumn{1}{|c}{} &       & \multicolumn{7}{c|}{Sparsistency (number of extra variables)} \\
\midrule\multirow{4}[2]{*}{hsnr} & p=14 & 6(0.3)/6(0.2)/6(0.6) & 6(0.7) & 6(0.6) & 6(3.7)/6(4.5) & 6(1)/6(1.3) & 6(0.6) & 6(0.8) \\ 
   & p=30 & 6(0.1)/6(0)/6(0.6) & 6(0.7) & 6(0.7) & 6(7.4)/6(8.2) & 6(2.4)/6(1.7) & 6(1) & 6(0.9) \\ 
   & p=60 & 6(0.1)/6(0)/6(0.5) & - & 6(0.5) & 6(11.3)/6(13.1) & 6(4.8)/6(1.7) & 6(1.5) & 6(0.9) \\ 
   & p=180 & 6(8.8)/6(0)/6(0.4) & - & 6(0.4) & 6(16.5)/6(22.7) & 6(18)/6(2.5) & 6(2.7) & 6(0.7) \\ 
  \midrule\multirow{4}[2]{*}{msnr} & p=14 & 6(0.8)/6(0.6)/6(0.6) & 6(0.7) & 6(0.6) & 6(3.7)/6(4.5) & 6(1.2)/6(1.4) & 6(0.6) & 6(0.8) \\ 
   & p=30 & 6(0.5)/6(0.3)/6(0.6) & 6(0.7) & 6(0.7) & 6(7.4)/6(8.2) & 6(2.9)/6(1.6) & 6(0.8) & 6(0.9) \\ 
   & p=60 & 6(0.3)/6(0.1)/6(0.5) & - & 6(0.5) & 6(11.4)/6(13.1) & 6(6.4)/6(1.5) & 6(1.1) & 6(1) \\ 
   & p=180 & 6(9.4)/6(0.1)/6(0.4) & - & 6(0.4) & 6(16.5)/6(22.8) & 6(27.4)/6(2) & 6(1.7) & 6(1.3) \\ 
  \midrule\multirow{4}[2]{*}{lsnr} & p=14 & 5.4(4.4)/5.2(4.1)/4.7(1.8) & 4.7(1.7) & 4.7(1.7) & 5.6(3.3)/5.6(4) & 5.1(1.5)/5.3(3) & 5.1(2.8) & 5(1.9) \\ 
   & p=30 & 4(4.4)/3.1(1.9)/3.7(2.1) & 3.6(2.1) & 3.7(2) & 5.3(6.4)/5.4(7) & 4.9(3.8)/4.8(4.9) & 4.8(5.3) & 4.5(3.3) \\ 
   & p=60 & 2.2(1.8)/1.2(0.2)/2.6(1.4) & - & 2.6(1.3) & 4.6(8.6)/4.6(9.6) & 4.9(8.5)/4.1(6.6) & 4.2(7.2) & 3.9(5) \\ 
   & p=180 & 1.6(14.2)/0.5(0.1)/1.3(0.6) & - & 1.3(0.6) & 3.4(10.4)/3.5(13.1) & 4.6(36.9)/3(9) & 3.2(10.5) & 2.8(7.1) \\ 
   \bottomrule 
\end{tabular}
}
\end{table}

% latex table generated in R 3.6.2 by xtable 1.8-4 package
% Sat Dec 21 23:35:58 2019
\begin{table}[ht]
\centering
\caption{The performance of BOSS compared to other methods, Sparse-Ex3, $\rho$=0, n=2000} 
\scalebox{0.7}{
\begin{tabular}{|c|c|ccccccc|}
  \toprule 
 \multicolumn{1}{|c}{} &       & BOSS  & BS    & FS    & LASSO & Gamma LASSO & SparseNet & \multicolumn{1}{c|}{rLASSO} \\
 \multicolumn{1}{|c}{} &       & C$_p$-hdf/AICc-hdf/CV & CV    & CV    & AICc/CV & AICc/CV & CV    & \multicolumn{1}{c|}{CV}  \\
 \cmidrule{3-9}\multicolumn{1}{|c}{} &       & \multicolumn{7}{c|}{\% worse than the best possible BOSS}  \\
 \midrule\multirow{4}[2]{*}{hsnr} & p=14 & 6/6/17 & 17 & 17 & 41/41 & 12/15 & 11 & 13 \\ 
   & p=30 & 3/3/22 & 22 & 23 & 72/69 & 20/18 & 13 & 16 \\ 
   & p=60 & 2/2/24 & - & 23 & 97/93 & 29/19 & 14 & 17 \\ 
   & p=180 & 1/1/21 & - & 21 & 132/129 & 53/19 & 13 & 17 \\ 
  \midrule\multirow{4}[2]{*}{msnr} & p=14 & 6/6/17 & 17 & 17 & 41/41 & 14/17 & 11 & 13 \\ 
   & p=30 & 3/3/22 & 22 & 23 & 72/69 & 26/20 & 13 & 17 \\ 
   & p=60 & 2/2/24 & - & 23 & 97/93 & 43/21 & 14 & 18 \\ 
   & p=180 & 1/1/21 & - & 21 & 132/129 & 97/22 & 13 & 17 \\ 
  \midrule\multirow{4}[2]{*}{lsnr} & p=14 & 9/9/17 & 17 & 17 & 42/41 & 21/20 & 12 & 13 \\ 
   & p=30 & 5/5/22 & 22 & 23 & 72/69 & 45/25 & 13 & 16 \\ 
   & p=60 & 5/4/24 & - & 23 & 97/93 & 82/26 & 15 & 18 \\ 
   & p=180 & 5/4/21 & - & 21 & 132/129 & 192/26 & 13 & 17 \\ 
   \midrule 
 \multicolumn{1}{|c}{} &       & \multicolumn{7}{c|}{Relative efficiency} \\
\midrule\multirow{4}[2]{*}{hsnr} & p=14 & 1/1/0.91 & 0.91 & 0.91 & 0.76/0.76 & 0.95/0.93 & 0.96 & 0.94 \\ 
   & p=30 & 1/1/0.84 & 0.84 & 0.84 & 0.6/0.61 & 0.86/0.87 & 0.91 & 0.88 \\ 
   & p=60 & 1/1/0.83 & - & 0.83 & 0.52/0.53 & 0.79/0.86 & 0.89 & 0.87 \\ 
   & p=180 & 1/1/0.83 & - & 0.84 & 0.44/0.44 & 0.66/0.85 & 0.89 & 0.86 \\ 
  \midrule\multirow{4}[2]{*}{msnr} & p=14 & 1/1/0.91 & 0.91 & 0.91 & 0.75/0.75 & 0.93/0.91 & 0.96 & 0.94 \\ 
   & p=30 & 1/1/0.84 & 0.84 & 0.84 & 0.6/0.61 & 0.81/0.86 & 0.91 & 0.88 \\ 
   & p=60 & 1/1/0.83 & - & 0.83 & 0.52/0.53 & 0.71/0.84 & 0.89 & 0.87 \\ 
   & p=180 & 1/1/0.83 & - & 0.84 & 0.44/0.44 & 0.51/0.83 & 0.89 & 0.86 \\ 
  \midrule\multirow{4}[2]{*}{lsnr} & p=14 & 1/1/0.93 & 0.93 & 0.93 & 0.77/0.77 & 0.9/0.9 & 0.97 & 0.96 \\ 
   & p=30 & 0.99/1/0.86 & 0.86 & 0.85 & 0.61/0.62 & 0.72/0.84 & 0.93 & 0.9 \\ 
   & p=60 & 1/1/0.84 & - & 0.85 & 0.53/0.54 & 0.57/0.83 & 0.91 & 0.89 \\ 
   & p=180 & 0.99/1/0.86 & - & 0.86 & 0.45/0.46 & 0.36/0.82 & 0.92 & 0.89 \\ 
   \midrule 
 \multicolumn{1}{|c}{} &       & \multicolumn{7}{c|}{Sparsistency (number of extra variables)} \\
\midrule\multirow{4}[2]{*}{hsnr} & p=14 & 6(0.3)/6(0.3)/6(0.6) & 6(0.6) & 6(0.6) & 6(3.8)/6(4.5) & 6(0.9)/6(1.3) & 6(0.6) & 6(0.5) \\ 
   & p=30 & 6(0.1)/6(0.1)/6(0.6) & 6(0.6) & 6(0.6) & 6(8.3)/6(8.5) & 6(2)/6(1.8) & 6(0.9) & 6(0.5) \\ 
   & p=60 & 6(0)/6(0)/6(0.5) & - & 6(0.5) & 6(13)/6(12.2) & 6(3.9)/6(2.1) & 6(1.5) & 6(0.5) \\ 
   & p=180 & 6(0)/6(0)/6(0.3) & - & 6(0.3) & 6(21.7)/6(19.1) & 6(10.3)/6(2.4) & 6(2.1) & 6(0.4) \\ 
  \midrule\multirow{4}[2]{*}{msnr} & p=14 & 6(0.3)/6(0.3)/6(0.6) & 6(0.6) & 6(0.6) & 6(3.9)/6(4.5) & 6(1)/6(1.3) & 6(0.6) & 6(0.5) \\ 
   & p=30 & 6(0.1)/6(0.1)/6(0.6) & 6(0.6) & 6(0.6) & 6(8.3)/6(8.6) & 6(2.2)/6(1.8) & 6(0.9) & 6(0.5) \\ 
   & p=60 & 6(0)/6(0)/6(0.5) & - & 6(0.5) & 6(13)/6(12.2) & 6(4.8)/6(1.8) & 6(1.4) & 6(0.6) \\ 
   & p=180 & 6(0)/6(0)/6(0.3) & - & 6(0.3) & 6(21.6)/6(19.1) & 6(15.6)/6(1.8) & 6(1.8) & 6(0.4) \\ 
  \midrule\multirow{4}[2]{*}{lsnr} & p=14 & 6(0.4)/6(0.4)/6(0.6) & 6(0.6) & 6(0.6) & 6(3.9)/6(4.5) & 6(1.1)/6(1.4) & 6(0.5) & 6(0.5) \\ 
   & p=30 & 6(0.1)/6(0.1)/6(0.6) & 6(0.6) & 6(0.6) & 6(8.3)/6(8.5) & 6(3)/6(1.6) & 6(0.7) & 6(0.5) \\ 
   & p=60 & 6(0.1)/6(0.1)/6(0.5) & - & 6(0.5) & 6(13.2)/6(12.2) & 6(6.8)/6(1.5) & 6(1) & 6(0.6) \\ 
   & p=180 & 6(0.1)/6(0)/6(0.3) & - & 6(0.3) & 6(21.9)/6(19.1) & 6(23.7)/6(1) & 6(1) & 6(0.5) \\ 
   \bottomrule 
\end{tabular}
}
\end{table}

% latex table generated in R 3.6.1 by xtable 1.8-4 package
% Sat Nov  9 19:29:57 2019
\begin{table}[ht]
\centering
\caption{The performance of BOSS compared to other methods, Sparse-Ex3, $\rho$=0.5, n=200} 

\scalebox{0.7}{
\begin{tabular}{|c|c|ccccccc|}
  \toprule 
 \multicolumn{1}{|c}{} &       & BOSS  & BS    & FS    & LASSO & Gamma LASSO & SparseNet & \multicolumn{1}{c|}{rLASSO} \\
 \multicolumn{1}{|c}{} &       & C$_p$-hdf/AICc-hdf/CV & CV    & CV    & AICc/CV & AICc/CV & CV    & \multicolumn{1}{c|}{CV}  \\
 \cmidrule{3-9}\multicolumn{1}{|c}{} &       & \multicolumn{7}{c|}{\% worse than the best possible BOSS}  \\
 \midrule\multirow{4}[2]{*}{hsnr} & p=14 & 7/6/18 & 19 & 18 & 40/39 & 15/18 & 14 & 18 \\ 
   & p=30 & 5/2/22 & 24 & 22 & 70/68 & 30/21 & 14 & 22 \\ 
   & p=60 & 3/1/22 & - & 23 & 93/92 & 51/22 & 16 & 24 \\ 
   & p=180 & 35/1/18 & - & 21 & 135/135 & 134/26 & 17 & 24 \\ 
  \midrule\multirow{4}[2]{*}{msnr} & p=14 & 15/13/18 & 19 & 18 & 40/39 & 20/21 & 15 & 19 \\ 
   & p=30 & 14/10/22 & 25 & 23 & 69/68 & 45/25 & 15 & 25 \\ 
   & p=60 & 13/9/23 & - & 24 & 91/89 & 78/26 & 15 & 33 \\ 
   & p=180 & 48/11/20 & - & 23 & 132/133 & 218/31 & 18 & 51 \\ 
  \midrule\multirow{4}[2]{*}{lsnr} & p=14 & 19/21/24 & 24 & 24 & 5/4 & 11/11 & 13 & 12 \\ 
   & p=30 & 28/30/25 & 25 & 25 & 0/0 & 12/6 & 7 & 8 \\ 
   & p=60 & 23/25/21 & - & 21 & -3/-3 & 22/3 & 3 & 4 \\ 
   & p=180 & 27/11/13 & - & 13 & -3/-1 & 83/3 & 3 & 4 \\ 
   \midrule 
 \multicolumn{1}{|c}{} &       & \multicolumn{7}{c|}{Relative efficiency} \\
\midrule\multirow{4}[2]{*}{hsnr} & p=14 & 0.98/1/0.9 & 0.89 & 0.9 & 0.75/0.76 & 0.92/0.9 & 0.92 & 0.9 \\ 
   & p=30 & 0.98/1/0.84 & 0.82 & 0.84 & 0.6/0.61 & 0.79/0.85 & 0.9 & 0.84 \\ 
   & p=60 & 0.98/1/0.83 & - & 0.83 & 0.53/0.53 & 0.67/0.83 & 0.88 & 0.82 \\ 
   & p=180 & 0.75/1/0.85 & - & 0.84 & 0.43/0.43 & 0.43/0.8 & 0.86 & 0.81 \\ 
  \midrule\multirow{4}[2]{*}{msnr} & p=14 & 0.98/1/0.95 & 0.95 & 0.95 & 0.8/0.81 & 0.93/0.93 & 0.98 & 0.94 \\ 
   & p=30 & 0.97/1/0.9 & 0.88 & 0.89 & 0.65/0.66 & 0.76/0.88 & 0.96 & 0.88 \\ 
   & p=60 & 0.97/1/0.88 & - & 0.88 & 0.57/0.57 & 0.61/0.86 & 0.94 & 0.82 \\ 
   & p=180 & 0.75/1/0.92 & - & 0.9 & 0.48/0.47 & 0.35/0.84 & 0.94 & 0.73 \\ 
  \midrule\multirow{4}[2]{*}{lsnr} & p=14 & 0.88/0.86/0.84 & 0.84 & 0.84 & 0.99/1 & 0.94/0.94 & 0.93 & 0.93 \\ 
   & p=30 & 0.78/0.77/0.8 & 0.8 & 0.8 & 1/1 & 0.9/0.94 & 0.93 & 0.92 \\ 
   & p=60 & 0.79/0.78/0.8 & - & 0.8 & 1/1 & 0.79/0.94 & 0.94 & 0.93 \\ 
   & p=180 & 0.76/0.87/0.86 & - & 0.86 & 1/0.98 & 0.53/0.94 & 0.94 & 0.93 \\ 
   \midrule 
 \multicolumn{1}{|c}{} &       & \multicolumn{7}{c|}{Sparsistency (number of extra variables)} \\
\midrule\multirow{4}[2]{*}{hsnr} & p=14 & 6(0.4)/6(0.2)/6(0.6) & 6(0.7) & 6(0.6) & 6(3.7)/6(4.5) & 6(0.9)/6(1.3) & 6(0.7) & 6(0.8) \\ 
   & p=30 & 6(0.2)/6(0)/6(0.6) & 6(0.7) & 6(0.6) & 6(7.9)/6(9) & 6(2.4)/6(1.5) & 6(1.1) & 6(1) \\ 
   & p=60 & 6(0.1)/6(0)/6(0.5) & - & 6(0.5) & 6(11.5)/6(13.3) & 6(4.9)/6(1.6) & 6(1.6) & 6(1) \\ 
   & p=180 & 6(9.6)/6(0)/6(0.3) & - & 6(0.4) & 6(16.6)/6(23.8) & 6(18.1)/6(2.1) & 6(2.4) & 6(1) \\ 
  \midrule\multirow{4}[2]{*}{msnr} & p=14 & 6(0.7)/6(0.6)/6(0.6) & 6(0.7) & 6(0.6) & 6(3.7)/6(4.5) & 6(1)/6(1.3) & 6(0.6) & 6(1) \\ 
   & p=30 & 6(0.4)/6(0.2)/6(0.7) & 6(0.8) & 6(0.7) & 6(8)/6(9) & 6(2.9)/6(1.5) & 6(0.8) & 6(1.1) \\ 
   & p=60 & 6(0.3)/6(0.2)/6(0.6) & - & 6(0.6) & 6(11.5)/6(13.3) & 6(6.1)/6(1.5) & 6(1.1) & 6(1.7) \\ 
   & p=180 & 6(10)/6(0.1)/6(0.4) & - & 6(0.5) & 6(17)/6(23.8) & 6(27.9)/6(1.8) & 6(1.7) & 6(2.3) \\ 
  \midrule\multirow{4}[2]{*}{lsnr} & p=14 & 5.3(4.3)/5.2(4)/4.6(2.1) & 4.6(2) & 4.6(2) & 5.5(3.4)/5.6(4) & 4.9(1.4)/5.2(3) & 5.2(2.6) & 5(2.3) \\ 
   & p=30 & 3.8(4.3)/2.9(2)/3.6(2.4) & 3.4(2.1) & 3.5(2.3) & 5.1(6.9)/5.2(7.5) & 4.6(3.9)/4.6(5.3) & 4.7(5.3) & 4.4(4) \\ 
   & p=60 & 2.1(1.7)/1.3(0.4)/2.3(1.5) & - & 2.3(1.4) & 4.4(8.6)/4.5(9.7) & 4.4(8.3)/3.9(6.7) & 4(7.5) & 3.7(5.2) \\ 
   & p=180 & 1.4(15.1)/0.3(0.1)/1(0.7) & - & 1(0.7) & 3(9.7)/3(12.5) & 4.3(37.2)/2.5(8.3) & 2.6(9.3) & 2.3(7.1) \\ 
   \bottomrule 
\end{tabular}
}
\end{table}

% latex table generated in R 3.6.1 by xtable 1.8-4 package
% Sat Nov  9 19:30:05 2019
\begin{table}[ht]
\centering
\caption{The performance of BOSS compared to other methods, Sparse-Ex3, $\rho$=0.5, n=2000} 

\scalebox{0.7}{
\begin{tabular}{|c|c|ccccccc|}
  \toprule 
 \multicolumn{1}{|c}{} &       & BOSS  & BS    & FS    & LASSO & Gamma LASSO & SparseNet & \multicolumn{1}{c|}{rLASSO} \\
 \multicolumn{1}{|c}{} &       & C$_p$-hdf/AICc-hdf/CV & CV    & CV    & AICc/CV & AICc/CV & CV    & \multicolumn{1}{c|}{CV}  \\
 \cmidrule{3-9}\multicolumn{1}{|c}{} &       & \multicolumn{7}{c|}{\% worse than the best possible BOSS}  \\
 \midrule\multirow{4}[2]{*}{hsnr} & p=14 & 7/6/18 & 19 & 19 & 43/42 & 13/16 & 13 & 18 \\ 
   & p=30 & 3/3/22 & 22 & 21 & 73/71 & 21/18 & 14 & 23 \\ 
   & p=60 & 1/1/21 & - & 20 & 96/92 & 29/18 & 13 & 26 \\ 
   & p=180 & 1/1/22 & - & 22 & 130/126 & 53/20 & 14 & 25 \\ 
  \midrule\multirow{4}[2]{*}{msnr} & p=14 & 7/6/18 & 19 & 19 & 44/43 & 16/18 & 13 & 18 \\ 
   & p=30 & 3/3/22 & 22 & 21 & 73/71 & 28/20 & 14 & 23 \\ 
   & p=60 & 1/1/21 & - & 20 & 96/92 & 43/21 & 13 & 26 \\ 
   & p=180 & 1/1/22 & - & 22 & 130/126 & 98/22 & 13 & 25 \\ 
  \midrule\multirow{4}[2]{*}{lsnr} & p=14 & 9/9/18 & 19 & 19 & 44/43 & 22/22 & 13 & 18 \\ 
   & p=30 & 5/5/22 & 21 & 21 & 73/71 & 47/25 & 14 & 23 \\ 
   & p=60 & 4/4/21 & - & 20 & 96/92 & 80/26 & 14 & 26 \\ 
   & p=180 & 5/5/22 & - & 22 & 129/126 & 192/27 & 13 & 24 \\ 
   \midrule 
 \multicolumn{1}{|c}{} &       & \multicolumn{7}{c|}{Relative efficiency} \\
\midrule\multirow{4}[2]{*}{hsnr} & p=14 & 1/1/0.9 & 0.9 & 0.9 & 0.75/0.75 & 0.94/0.92 & 0.95 & 0.9 \\ 
   & p=30 & 1/1/0.85 & 0.85 & 0.85 & 0.59/0.6 & 0.85/0.87 & 0.91 & 0.84 \\ 
   & p=60 & 1/1/0.84 & - & 0.84 & 0.52/0.53 & 0.79/0.86 & 0.89 & 0.8 \\ 
   & p=180 & 1/1/0.83 & - & 0.83 & 0.44/0.45 & 0.66/0.84 & 0.89 & 0.81 \\ 
  \midrule\multirow{4}[2]{*}{msnr} & p=14 & 1/1/0.9 & 0.9 & 0.9 & 0.74/0.75 & 0.92/0.9 & 0.94 & 0.9 \\ 
   & p=30 & 1/1/0.85 & 0.85 & 0.85 & 0.59/0.6 & 0.8/0.85 & 0.9 & 0.84 \\ 
   & p=60 & 1/1/0.84 & - & 0.84 & 0.52/0.53 & 0.71/0.84 & 0.9 & 0.8 \\ 
   & p=180 & 1/1/0.83 & - & 0.83 & 0.44/0.45 & 0.51/0.83 & 0.89 & 0.81 \\ 
  \midrule\multirow{4}[2]{*}{lsnr} & p=14 & 1/1/0.92 & 0.92 & 0.92 & 0.75/0.76 & 0.89/0.89 & 0.96 & 0.92 \\ 
   & p=30 & 1/1/0.86 & 0.86 & 0.86 & 0.61/0.61 & 0.71/0.84 & 0.92 & 0.85 \\ 
   & p=60 & 1/1/0.86 & - & 0.86 & 0.53/0.54 & 0.58/0.83 & 0.91 & 0.82 \\ 
   & p=180 & 1/1/0.86 & - & 0.86 & 0.46/0.46 & 0.36/0.82 & 0.93 & 0.84 \\ 
   \midrule 
 \multicolumn{1}{|c}{} &       & \multicolumn{7}{c|}{Sparsistency (number of extra variables)} \\
\midrule\multirow{4}[2]{*}{hsnr} & p=14 & 6(0.3)/6(0.3)/6(0.6) & 6(0.6) & 6(0.6) & 6(3.9)/6(4.4) & 6(0.9)/6(1.3) & 6(0.6) & 6(0.8) \\ 
   & p=30 & 6(0.1)/6(0.1)/6(0.6) & 6(0.6) & 6(0.6) & 6(8.4)/6(8.6) & 6(2.1)/6(1.7) & 6(1) & 6(0.8) \\ 
   & p=60 & 6(0)/6(0)/6(0.5) & - & 6(0.5) & 6(13.2)/6(12.3) & 6(3.9)/6(2.1) & 6(1.6) & 6(0.9) \\ 
   & p=180 & 6(0)/6(0)/6(0.4) & - & 6(0.4) & 6(21.5)/6(18.7) & 6(10.4)/6(2.6) & 6(2.3) & 6(0.7) \\ 
  \midrule\multirow{4}[2]{*}{msnr} & p=14 & 6(0.3)/6(0.3)/6(0.6) & 6(0.6) & 6(0.6) & 6(3.9)/6(4.5) & 6(1)/6(1.4) & 6(0.6) & 6(0.8) \\ 
   & p=30 & 6(0.1)/6(0.1)/6(0.6) & 6(0.6) & 6(0.6) & 6(8.3)/6(8.6) & 6(2.4)/6(1.6) & 6(0.9) & 6(0.8) \\ 
   & p=60 & 6(0)/6(0)/6(0.5) & - & 6(0.5) & 6(13)/6(12.3) & 6(4.7)/6(1.8) & 6(1.4) & 6(0.9) \\ 
   & p=180 & 6(0)/6(0)/6(0.4) & - & 6(0.4) & 6(21.5)/6(18.7) & 6(15.7)/6(1.9) & 6(2.1) & 6(0.7) \\ 
  \midrule\multirow{4}[2]{*}{lsnr} & p=14 & 6(0.4)/6(0.4)/6(0.6) & 6(0.6) & 6(0.6) & 6(3.9)/6(4.5) & 6(1.1)/6(1.4) & 6(0.5) & 6(0.8) \\ 
   & p=30 & 6(0.1)/6(0.1)/6(0.6) & 6(0.6) & 6(0.6) & 6(8.3)/6(8.6) & 6(3)/6(1.4) & 6(0.7) & 6(0.8) \\ 
   & p=60 & 6(0.1)/6(0.1)/6(0.5) & - & 6(0.5) & 6(13.1)/6(12.2) & 6(6.5)/6(1.5) & 6(1) & 6(0.9) \\ 
   & p=180 & 6(0.1)/6(0.1)/6(0.4) & - & 6(0.4) & 6(21.2)/6(18.7) & 6(23.5)/6(1.2) & 6(0.9) & 6(0.7) \\ 
   \bottomrule 
\end{tabular}
}
\end{table}

% latex table generated in R 3.6.1 by xtable 1.8-4 package
% Sat Nov  9 19:30:06 2019
\begin{table}[ht]
\centering
\caption{The performance of BOSS compared to other methods, Sparse-Ex3, $\rho$=0.9, n=200} 

\scalebox{0.7}{
\begin{tabular}{|c|c|ccccccc|}
  \toprule 
 \multicolumn{1}{|c}{} &       & BOSS  & BS    & FS    & lasso & gamma lasso & SparseNet & \multicolumn{1}{c|}{rlasso} \\
 \multicolumn{1}{|c}{} &       & C$_p$-hdf/AICc-hdf/CV & CV    & CV    & AICc/CV & AICc/CV & CV    & \multicolumn{1}{c|}{CV}  \\
 \cmidrule{3-9}\multicolumn{1}{|c}{} &       & \multicolumn{7}{c|}{\% worse than the best possible BOSS}  \\
 \midrule\multirow{4}[2]{*}{hsnr} & p=14 & 7/6/24 & 13 & 24 & 33/33 & 14/16 & 12 & 20 \\ 
   & p=30 & 7/5/41 & 17 & 41 & 66/64 & 26/29 & 12 & 33 \\ 
   & p=60 & 6/4/43 & - & 43 & 84/83 & 44/38 & 13 & 39 \\ 
   & p=180 & 35/3/27 & - & 29 & 126/126 & 132/35 & 16 & 30 \\ 
  \midrule\multirow{4}[2]{*}{msnr} & p=14 & 14/13/19 & 16 & 19 & 18/17 & 17/17 & 6 & 11 \\ 
   & p=30 & 15/13/24 & 8 & 25 & 30/29 & 29/24 & 0 & 12 \\ 
   & p=60 & 12/10/20 & - & 20 & 43/42 & 50/25 & -6 & 14 \\ 
   & p=180 & 35/7/16 & - & 18 & 87/87 & 164/23 & 2 & 21 \\ 
  \midrule\multirow{4}[2]{*}{lsnr} & p=14 & 17/20/22 & 22 & 21 & -2/-3 & 5/3 & 5 & 5 \\ 
   & p=30 & 26/28/24 & 23 & 23 & -2/-2 & 8/6 & 5 & 5 \\ 
   & p=60 & 23/26/22 & - & 22 & -3/-3 & 20/3 & 4 & 3 \\ 
   & p=180 & 29/16/16 & - & 15 & -5/-3 & 79/1 & 1 & 2 \\ 
   \midrule 
 \multicolumn{1}{|c}{} &       & \multicolumn{7}{c|}{Relative efficiency} \\
\midrule\multirow{4}[2]{*}{hsnr} & p=14 & 0.99/1/0.85 & 0.93 & 0.85 & 0.79/0.79 & 0.92/0.91 & 0.95 & 0.88 \\ 
   & p=30 & 0.98/1/0.74 & 0.9 & 0.74 & 0.63/0.64 & 0.83/0.81 & 0.93 & 0.79 \\ 
   & p=60 & 0.98/1/0.73 & - & 0.73 & 0.57/0.57 & 0.73/0.75 & 0.92 & 0.75 \\ 
   & p=180 & 0.76/1/0.8 & - & 0.79 & 0.45/0.45 & 0.44/0.76 & 0.88 & 0.79 \\ 
  \midrule\multirow{4}[2]{*}{msnr} & p=14 & 0.93/0.94/0.9 & 0.92 & 0.9 & 0.9/0.91 & 0.91/0.91 & 1 & 0.96 \\ 
   & p=30 & 0.87/0.88/0.8 & 0.93 & 0.8 & 0.77/0.77 & 0.77/0.8 & 1 & 0.89 \\ 
   & p=60 & 0.84/0.85/0.78 & - & 0.78 & 0.66/0.66 & 0.63/0.75 & 1 & 0.83 \\ 
   & p=180 & 0.75/0.95/0.88 & - & 0.87 & 0.55/0.55 & 0.39/0.83 & 1 & 0.84 \\ 
  \midrule\multirow{4}[2]{*}{lsnr} & p=14 & 0.83/0.81/0.8 & 0.8 & 0.81 & 0.99/1 & 0.93/0.94 & 0.93 & 0.93 \\ 
   & p=30 & 0.78/0.76/0.79 & 0.8 & 0.8 & 1/1 & 0.91/0.93 & 0.93 & 0.94 \\ 
   & p=60 & 0.78/0.77/0.79 & - & 0.8 & 1/1 & 0.8/0.94 & 0.93 & 0.94 \\ 
   & p=180 & 0.74/0.82/0.82 & - & 0.83 & 1/0.98 & 0.53/0.94 & 0.95 & 0.93 \\ 
   \midrule 
 \multicolumn{1}{|c}{} &       & \multicolumn{7}{c|}{Sparsistency (number of extra variables)} \\
\midrule\multirow{4}[2]{*}{hsnr} & p=14 & 6(0.6)/6(0.5)/6(1.4) & 6(0.6) & 6(1.4) & 6(3.9)/6(4.5) & 6(1)/6(1.4) & 6(0.7) & 6(1.8) \\ 
   & p=30 & 6(0.7)/6(0.6)/6(2.1) & 6(0.8) & 6(2.1) & 6(9.2)/6(10.3) & 6(2.5)/6(3.4) & 6(1.6) & 6(3.2) \\ 
   & p=60 & 6(0.8)/6(0.7)/6(2) & - & 6(1.9) & 6(12.4)/6(14.2) & 6(5.1)/6(4.7) & 6(2.5) & 6(3.1) \\ 
   & p=180 & 6(9.2)/6(0.1)/6(0.6) & - & 6(0.6) & 6(16.2)/6(22.2) & 6(18.1)/6(3.4) & 6(2.4) & 6(1.7) \\ 
  \midrule\multirow{4}[2]{*}{msnr} & p=14 & 5.6(1.2)/5.6(1.1)/5.8(2.1) & 5.8(1.8) & 5.8(2.1) & 6(4)/6(4.5) & 5.7(1.5)/5.8(2.6) & 5.8(1) & 5.9(2.3) \\ 
   & p=30 & 5.1(1.5)/5.1(1.2)/5.3(2.6) & 5.5(1.4) & 5.3(2.6) & 6(9.1)/6(10.2) & 5.2(4.1)/5.4(5.8) & 5.7(1.8) & 5.8(3.8) \\ 
   & p=60 & 5.2(1.2)/5.2(1)/5.2(1.7) & - & 5.2(1.6) & 5.9(12.4)/6(14.1) & 5.2(7.5)/5.3(5.2) & 5.8(1.7) & 5.8(3.4) \\ 
   & p=180 & 5.6(10)/5.6(0.5)/5.6(0.8) & - & 5.6(0.8) & 6(16.2)/6(22.2) & 5.6(27.5)/5.6(3.1) & 5.9(1.7) & 5.9(2.6) \\ 
  \midrule\multirow{4}[2]{*}{lsnr} & p=14 & 4.4(4)/4.2(3.6)/3.7(2.5) & 3.6(2.4) & 3.7(2.3) & 4.8(3.6)/4.9(4.1) & 3.9(2.1)/4.4(3.2) & 4.5(2.5) & 4.4(3) \\ 
   & p=30 & 2.6(4.4)/1.9(2.3)/2.4(3) & 2.5(2.8) & 2.4(2.9) & 3.9(7.5)/4(8.2) & 3.3(4.9)/3.3(6.3) & 3.7(6.1) & 3.3(5.1) \\ 
   & p=60 & 1.7(2)/1.1(0.8)/1.8(2) & - & 1.8(1.9) & 3.7(9.4)/3.8(10.4) & 3.4(8.9)/3.2(7.6) & 3.4(7.7) & 3(6) \\ 
   & p=180 & 1.4(14.4)/0.5(0.2)/1.1(1.1) & - & 1.1(1.1) & 3.2(11.1)/3.3(14.7) & 3.5(36.7)/2.8(10.2) & 3(10.8) & 2.6(7.5) \\ 
   \bottomrule 
\end{tabular}
}
\end{table}

% latex table generated in R 3.6.1 by xtable 1.8-4 package
% Sat Nov  9 19:30:13 2019
\begin{table}[ht]
\centering
\caption{The performance of BOSS compared to other methods, Sparse-Ex3, $\rho$=0.9, n=2000} 

\scalebox{0.7}{
\begin{tabular}{|c|c|ccccccc|}
  \toprule 
 \multicolumn{1}{|c}{} &       & BOSS  & BS    & FS    & lasso & gamma lasso & SparseNet & \multicolumn{1}{c|}{rlasso} \\
 \multicolumn{1}{|c}{} &       & C$_p$-hdf/AICc-hdf/CV & CV    & CV    & AICc/CV & AICc/CV & CV    & \multicolumn{1}{c|}{CV}  \\
 \cmidrule{3-9}\multicolumn{1}{|c}{} &       & \multicolumn{7}{c|}{\% worse than the best possible BOSS}  \\
 \midrule\multirow{4}[2]{*}{hsnr} & p=14 & 6/6/19 & 19 & 19 & 40/39 & 12/14 & 13 & 17 \\ 
   & p=30 & 2/2/21 & 21 & 22 & 74/72 & 20/18 & 13 & 22 \\ 
   & p=60 & 1/1/22 & - & 22 & 101/97 & 29/19 & 14 & 25 \\ 
   & p=180 & 1/1/21 & - & 22 & 135/131 & 52/18 & 14 & 25 \\ 
  \midrule\multirow{4}[2]{*}{msnr} & p=14 & 6/6/19 & 19 & 19 & 42/40 & 18/18 & 14 & 17 \\ 
   & p=30 & 2/2/21 & 21 & 22 & 74/72 & 27/20 & 14 & 22 \\ 
   & p=60 & 1/1/22 & - & 22 & 101/97 & 43/21 & 14 & 25 \\ 
   & p=180 & 1/1/21 & - & 22 & 135/132 & 96/21 & 13 & 26 \\ 
  \midrule\multirow{4}[2]{*}{lsnr} & p=14 & 9/9/19 & 19 & 19 & 27/26 & 17/18 & 6 & 15 \\ 
   & p=30 & 5/5/21 & 20 & 21 & 53/51 & 34/23 & 3 & 18 \\ 
   & p=60 & 4/4/20 & - & 20 & 75/71 & 62/22 & 1 & 17 \\ 
   & p=180 & 4/4/17 & - & 17 & 92/89 & 145/23 & -5 & 16 \\ 
   \midrule 
 \multicolumn{1}{|c}{} &       & \multicolumn{7}{c|}{Relative efficiency} \\
\midrule\multirow{4}[2]{*}{hsnr} & p=14 & 1/1/0.89 & 0.89 & 0.89 & 0.76/0.76 & 0.95/0.93 & 0.94 & 0.9 \\ 
   & p=30 & 1/1/0.84 & 0.85 & 0.84 & 0.59/0.6 & 0.85/0.87 & 0.9 & 0.84 \\ 
   & p=60 & 1/1/0.83 & - & 0.83 & 0.5/0.51 & 0.79/0.85 & 0.89 & 0.81 \\ 
   & p=180 & 1/1/0.84 & - & 0.83 & 0.43/0.44 & 0.67/0.86 & 0.89 & 0.81 \\ 
  \midrule\multirow{4}[2]{*}{msnr} & p=14 & 1/1/0.89 & 0.89 & 0.89 & 0.75/0.76 & 0.9/0.9 & 0.93 & 0.9 \\ 
   & p=30 & 1/1/0.84 & 0.85 & 0.84 & 0.59/0.6 & 0.81/0.85 & 0.9 & 0.84 \\ 
   & p=60 & 1/1/0.83 & - & 0.83 & 0.5/0.51 & 0.71/0.84 & 0.89 & 0.81 \\ 
   & p=180 & 1/1/0.84 & - & 0.83 & 0.43/0.44 & 0.52/0.83 & 0.9 & 0.81 \\ 
  \midrule\multirow{4}[2]{*}{lsnr} & p=14 & 0.97/0.97/0.89 & 0.89 & 0.89 & 0.83/0.84 & 0.91/0.89 & 1 & 0.92 \\ 
   & p=30 & 0.98/0.98/0.85 & 0.86 & 0.85 & 0.67/0.68 & 0.77/0.83 & 1 & 0.87 \\ 
   & p=60 & 0.97/0.97/0.84 & - & 0.84 & 0.58/0.59 & 0.62/0.82 & 1 & 0.86 \\ 
   & p=180 & 0.91/0.91/0.81 & - & 0.81 & 0.49/0.5 & 0.39/0.77 & 1 & 0.81 \\ 
   \midrule 
 \multicolumn{1}{|c}{} &       & \multicolumn{7}{c|}{Sparsistency (number of extra variables)} \\
\midrule\multirow{4}[2]{*}{hsnr} & p=14 & 6(0.3)/6(0.3)/6(0.7) & 6(0.6) & 6(0.6) & 6(4.1)/6(4.6) & 6(0.8)/6(1) & 6(0.7) & 6(0.8) \\ 
   & p=30 & 6(0)/6(0)/6(0.6) & 6(0.6) & 6(0.6) & 6(9.2)/6(9.5) & 6(2)/6(1.8) & 6(1) & 6(1) \\ 
   & p=60 & 6(0)/6(0)/6(0.5) & - & 6(0.5) & 6(14)/6(13.3) & 6(3.7)/6(2.1) & 6(1.5) & 6(1) \\ 
   & p=180 & 6(0)/6(0)/6(0.4) & - & 6(0.4) & 6(23.2)/6(20.6) & 6(10.1)/6(2.4) & 6(2.2) & 6(1) \\ 
  \midrule\multirow{4}[2]{*}{msnr} & p=14 & 6(0.3)/6(0.3)/6(0.7) & 6(0.6) & 6(0.6) & 6(4.2)/6(4.6) & 6(1.1)/6(1.3) & 6(0.7) & 6(0.8) \\ 
   & p=30 & 6(0)/6(0)/6(0.6) & 6(0.6) & 6(0.6) & 6(9.2)/6(9.5) & 6(2.3)/6(1.7) & 6(1) & 6(1) \\ 
   & p=60 & 6(0)/6(0)/6(0.5) & - & 6(0.5) & 6(14.1)/6(13.3) & 6(4.7)/6(1.8) & 6(1.4) & 6(1) \\ 
   & p=180 & 6(0)/6(0)/6(0.4) & - & 6(0.4) & 6(23.2)/6(20.5) & 6(15.3)/6(1.9) & 6(1.9) & 6(1) \\ 
  \midrule\multirow{4}[2]{*}{lsnr} & p=14 & 5.8(0.4)/5.8(0.4)/5.9(1.4) & 5.9(1.5) & 5.9(1.5) & 6(4.2)/6(4.6) & 5.9(1.2)/5.9(2.1) & 6(0.8) & 6(1.7) \\ 
   & p=30 & 5.8(0.3)/5.8(0.3)/5.8(1.1) & 5.8(1.1) & 5.8(1.1) & 6(9.2)/6(9.5) & 5.8(2.8)/5.9(3) & 6(0.8) & 6(1.9) \\ 
   & p=60 & 5.8(0.3)/5.8(0.3)/5.8(0.8) & - & 5.8(0.8) & 6(14)/6(13.3) & 5.8(6.3)/5.8(2.3) & 6(1.1) & 6(1.6) \\ 
   & p=180 & 5.7(0.3)/5.7(0.3)/5.7(0.7) & - & 5.7(0.7) & 6(23)/6(20.6) & 5.7(23.2)/5.7(2.6) & 6(1) & 5.9(2.1) \\ 
   \bottomrule 
\end{tabular}
}
\end{table}

% latex table generated in R 3.6.1 by xtable 1.8-4 package
% Sat Nov  9 19:30:14 2019
\begin{table}[ht]
\centering
\caption{The performance of BOSS compared to other methods, Sparse-Ex4, $\rho$=0, n=200} 

\scalebox{0.7}{
\begin{tabular}{|c|c|ccccccc|}
  \toprule 
 \multicolumn{1}{|c}{} &       & BOSS  & BS    & FS    & LASSO & Gamma LASSO & SparseNet & \multicolumn{1}{c|}{rLASSO} \\
 \multicolumn{1}{|c}{} &       & C$_p$-hdf/AICc-hdf/CV & CV    & CV    & AICc/CV & AICc/CV & CV    & \multicolumn{1}{c|}{CV}  \\
 \cmidrule{3-9}\multicolumn{1}{|c}{} &       & \multicolumn{7}{c|}{\% worse than the best possible BOSS}  \\
 \midrule\multirow{4}[2]{*}{hsnr} & p=14 & 30/30/22 & 22 & 22 & 19/19 & 16/18 & 20 & 20 \\ 
   & p=30 & 24/21/24 & 25 & 25 & 29/28 & 32/19 & 16 & 21 \\ 
   & p=60 & 16/15/20 & - & 21 & 37/36 & 57/17 & 13 & 24 \\ 
   & p=180 & 29/5/14 & - & 15 & 50/51 & 168/14 & 11 & 19 \\ 
  \midrule\multirow{4}[2]{*}{msnr} & p=14 & 25/22/22 & 23 & 22 & 31/31 & 26/21 & 18 & 21 \\ 
   & p=30 & 26/20/27 & 28 & 28 & 52/51 & 43/24 & 21 & 27 \\ 
   & p=60 & 18/14/24 & - & 25 & 69/68 & 72/25 & 21 & 31 \\ 
   & p=180 & 55/16/26 & - & 27 & 97/99 & 237/29 & 27 & 39 \\ 
  \midrule\multirow{4}[2]{*}{lsnr} & p=14 & 34/34/29 & 30 & 29 & 16/16 & 19/25 & 25 & 19 \\ 
   & p=30 & 37/34/31 & 31 & 31 & 20/18 & 33/25 & 24 & 23 \\ 
   & p=60 & 32/33/29 & - & 29 & 20/20 & 55/24 & 22 & 23 \\ 
   & p=180 & 49/29/27 & - & 26 & 18/20 & 142/23 & 21 & 18 \\ 
   \midrule 
 \multicolumn{1}{|c}{} &       & \multicolumn{7}{c|}{Relative efficiency} \\
\midrule\multirow{4}[2]{*}{hsnr} & p=14 & 0.89/0.89/0.95 & 0.95 & 0.95 & 0.98/0.98 & 1/0.98 & 0.97 & 0.97 \\ 
   & p=30 & 0.94/0.96/0.94 & 0.93 & 0.93 & 0.9/0.91 & 0.88/0.98 & 1 & 0.96 \\ 
   & p=60 & 0.97/0.99/0.94 & - & 0.94 & 0.83/0.83 & 0.72/0.97 & 1 & 0.92 \\ 
   & p=180 & 0.82/1/0.92 & - & 0.92 & 0.7/0.7 & 0.39/0.92 & 0.94 & 0.89 \\ 
  \midrule\multirow{4}[2]{*}{msnr} & p=14 & 0.95/0.97/0.97 & 0.97 & 0.97 & 0.9/0.91 & 0.94/0.98 & 1 & 0.98 \\ 
   & p=30 & 0.96/1/0.95 & 0.94 & 0.94 & 0.79/0.8 & 0.84/0.97 & 0.99 & 0.95 \\ 
   & p=60 & 0.96/1/0.92 & - & 0.92 & 0.68/0.68 & 0.66/0.91 & 0.95 & 0.87 \\ 
   & p=180 & 0.74/1/0.92 & - & 0.91 & 0.59/0.58 & 0.34/0.9 & 0.91 & 0.83 \\ 
  \midrule\multirow{4}[2]{*}{lsnr} & p=14 & 0.86/0.87/0.9 & 0.89 & 0.9 & 1/1 & 0.97/0.93 & 0.93 & 0.98 \\ 
   & p=30 & 0.87/0.89/0.9 & 0.91 & 0.9 & 0.99/1 & 0.89/0.95 & 0.96 & 0.97 \\ 
   & p=60 & 0.91/0.9/0.93 & - & 0.93 & 1/1 & 0.77/0.97 & 0.98 & 0.97 \\ 
   & p=180 & 0.79/0.91/0.93 & - & 0.93 & 1/0.98 & 0.49/0.95 & 0.97 & 1 \\ 
   \midrule 
 \multicolumn{1}{|c}{} &       & \multicolumn{7}{c|}{Sparsistency (number of extra variables)} \\
\midrule\multirow{4}[2]{*}{hsnr} & p=14 & 5(1.7)/4.9(1.3)/5.2(0.9) & 5.2(0.9) & 5.2(0.9) & 5.8(3.4)/5.9(4.2) & 5.5(1.3)/5.3(1.5) & 5.3(1.3) & 5.3(1.1) \\ 
   & p=30 & 4.6(0.7)/4.4(0.2)/4.9(0.9) & 4.9(1.1) & 4.9(1) & 5.7(7)/5.8(8) & 5.4(3.3)/5(1.9) & 5(2) & 5(1.5) \\ 
   & p=60 & 4.3(0.2)/4.2(0)/4.6(0.7) & - & 4.6(0.7) & 5.5(10.4)/5.6(12.1) & 5.4(7.5)/4.8(2.2) & 4.9(2.8) & 4.7(1.9) \\ 
   & p=180 & 4.3(10.2)/4.1(0)/4.2(0.4) & - & 4.3(0.5) & 5.2(14)/5.3(19) & 5.4(35.4)/4.4(2.3) & 4.5(3.9) & 4.3(1.7) \\ 
  \midrule\multirow{4}[2]{*}{msnr} & p=14 & 4.5(1.3)/4.4(1)/4.4(0.6) & 4.4(0.7) & 4.4(0.7) & 5.2(3.1)/5.2(3.6) & 4.6(1.1)/4.5(1) & 4.4(0.8) & 4.5(0.8) \\ 
   & p=30 & 4.2(0.8)/4.1(0.4)/4.2(0.8) & 4.2(0.8) & 4.2(0.8) & 5(6.5)/5(7.1) & 4.6(2.9)/4.3(1.3) & 4.3(1.3) & 4.3(1.2) \\ 
   & p=60 & 4(0.3)/4(0.2)/4.1(0.6) & - & 4.1(0.6) & 4.7(9.5)/4.7(10.7) & 4.5(6.5)/4.1(1.3) & 4.2(1.7) & 4.1(1.4) \\ 
   & p=180 & 4.1(10.4)/3.9(0.1)/4(0.5) & - & 4(0.5) & 4.4(13.1)/4.5(17.1) & 4.6(31.6)/4(1.6) & 4.1(2.6) & 4(1.8) \\ 
  \midrule\multirow{4}[2]{*}{lsnr} & p=14 & 3.9(2.4)/3.7(2)/3.3(0.9) & 3.3(1.1) & 3.2(0.9) & 4.4(2.7)/4.5(3) & 3.7(1.3)/3.6(1.5) & 3.6(1.6) & 3.6(1.2) \\ 
   & p=30 & 2.6(1.8)/2.2(0.7)/2.5(1.1) & 2.5(1.1) & 2.5(1.1) & 3.8(5.3)/3.9(5.5) & 3.4(3.5)/2.9(2.5) & 3(2.9) & 2.9(2.1) \\ 
   & p=60 & 2(0.8)/1.7(0.2)/2.2(0.8) & - & 2.2(0.8) & 3.4(7.3)/3.5(8.1) & 3.5(7.8)/2.7(3.3) & 2.8(3.9) & 2.7(2.9) \\ 
   & p=180 & 1.8(12)/1.3(0.1)/1.6(0.6) & - & 1.7(0.6) & 2.9(9.7)/2.9(11.2) & 3.7(34)/2.2(5) & 2.4(6.2) & 2.2(3.7) \\ 
   \bottomrule 
\end{tabular}
}
\end{table}

% latex table generated in R 3.6.1 by xtable 1.8-4 package
% Sat Nov  9 19:30:22 2019
\begin{table}[ht]
\centering
\caption{The performance of BOSS compared to other methods, Sparse-Ex4, $\rho$=0, n=2000} 

\scalebox{0.7}{
\begin{tabular}{|c|c|ccccccc|}
  \toprule 
 \multicolumn{1}{|c}{} &       & BOSS  & BS    & FS    & LASSO & Gamma LASSO & SparseNet & \multicolumn{1}{c|}{rLASSO} \\
 \multicolumn{1}{|c}{} &       & C$_p$-hdf/AICc-hdf/CV & CV    & CV    & AICc/CV & AICc/CV & CV    & \multicolumn{1}{c|}{CV}  \\
 \cmidrule{3-9}\multicolumn{1}{|c}{} &       & \multicolumn{7}{c|}{\% worse than the best possible BOSS}  \\
 \midrule\multirow{4}[2]{*}{hsnr} & p=14 & 9/9/17 & 17 & 17 & 41/41 & 19/15 & 12 & 17 \\ 
   & p=30 & 5/5/22 & 22 & 23 & 74/71 & 45/22 & 13 & 24 \\ 
   & p=60 & 4/4/24 & - & 23 & 97/93 & 83/23 & 16 & 26 \\ 
   & p=180 & 4/4/21 & - & 21 & 131/127 & 200/23 & 13 & 24 \\ 
  \midrule\multirow{4}[2]{*}{msnr} & p=14 & 31/31/21 & 21 & 22 & 34/33 & 20/20 & 22 & 21 \\ 
   & p=30 & 40/40/29 & 31 & 29 & 52/50 & 42/29 & 28 & 33 \\ 
   & p=60 & 40/40/32 & - & 32 & 64/61 & 76/29 & 29 & 33 \\ 
   & p=180 & 39/40/31 & - & 32 & 67/64 & 157/27 & 27 & 30 \\ 
  \midrule\multirow{4}[2]{*}{lsnr} & p=14 & 19/19/19 & 19 & 19 & 28/28 & 24/18 & 16 & 19 \\ 
   & p=30 & 11/11/21 & 22 & 21 & 49/47 & 57/20 & 14 & 21 \\ 
   & p=60 & 10/9/20 & - & 20 & 64/61 & 102/19 & 13 & 23 \\ 
   & p=180 & 9/8/20 & - & 20 & 91/88 & 235/21 & 14 & 23 \\ 
   \midrule 
 \multicolumn{1}{|c}{} &       & \multicolumn{7}{c|}{Relative efficiency} \\
\midrule\multirow{4}[2]{*}{hsnr} & p=14 & 1/1/0.93 & 0.93 & 0.93 & 0.77/0.77 & 0.91/0.95 & 0.97 & 0.93 \\ 
   & p=30 & 1/1/0.86 & 0.86 & 0.86 & 0.6/0.61 & 0.72/0.86 & 0.93 & 0.85 \\ 
   & p=60 & 1/1/0.84 & - & 0.84 & 0.53/0.54 & 0.57/0.85 & 0.9 & 0.83 \\ 
   & p=180 & 1/1/0.86 & - & 0.86 & 0.45/0.46 & 0.35/0.85 & 0.92 & 0.84 \\ 
  \midrule\multirow{4}[2]{*}{msnr} & p=14 & 0.91/0.91/0.99 & 0.99 & 0.98 & 0.9/0.9 & 1/0.99 & 0.98 & 0.99 \\ 
   & p=30 & 0.91/0.91/0.99 & 0.97 & 0.99 & 0.84/0.85 & 0.9/0.99 & 1 & 0.96 \\ 
   & p=60 & 0.92/0.92/0.97 & - & 0.98 & 0.78/0.8 & 0.73/1 & 1 & 0.97 \\ 
   & p=180 & 0.91/0.91/0.96 & - & 0.96 & 0.76/0.77 & 0.49/1 & 1 & 0.97 \\ 
  \midrule\multirow{4}[2]{*}{lsnr} & p=14 & 0.97/0.97/0.97 & 0.97 & 0.97 & 0.9/0.91 & 0.93/0.98 & 1 & 0.97 \\ 
   & p=30 & 1/1/0.92 & 0.91 & 0.92 & 0.74/0.76 & 0.71/0.93 & 0.98 & 0.92 \\ 
   & p=60 & 1/1/0.91 & - & 0.91 & 0.67/0.68 & 0.54/0.92 & 0.97 & 0.89 \\ 
   & p=180 & 1/1/0.9 & - & 0.9 & 0.57/0.58 & 0.32/0.9 & 0.95 & 0.88 \\ 
   \midrule 
 \multicolumn{1}{|c}{} &       & \multicolumn{7}{c|}{Sparsistency (number of extra variables)} \\
\midrule\multirow{4}[2]{*}{hsnr} & p=14 & 6(0.4)/6(0.4)/6(0.6) & 6(0.6) & 6(0.6) & 6(3.8)/6(4.5) & 6(1.1)/6(0.8) & 6(0.5) & 6(0.7) \\ 
   & p=30 & 6(0.1)/6(0.1)/6(0.6) & 6(0.6) & 6(0.6) & 6(8.4)/6(8.7) & 6(2.9)/6(1.3) & 6(0.7) & 6(0.9) \\ 
   & p=60 & 6(0.1)/6(0.1)/6(0.5) & - & 6(0.5) & 6(13)/6(12.2) & 6(6.8)/6(1.5) & 6(1.1) & 6(0.8) \\ 
   & p=180 & 6(0)/6(0)/6(0.3) & - & 6(0.3) & 6(21.7)/6(19.4) & 6(24.4)/6(1.5) & 6(1.2) & 6(0.6) \\ 
  \midrule\multirow{4}[2]{*}{msnr} & p=14 & 5.8(1.9)/5.8(1.8)/5.8(0.8) & 5.8(0.8) & 5.8(0.8) & 6(3.8)/6(4.4) & 5.9(1.2)/5.9(1.5) & 5.8(1.1) & 5.9(1) \\ 
   & p=30 & 5.4(1)/5.4(0.9)/5.6(0.9) & 5.6(0.9) & 5.6(0.9) & 6(8.5)/6(8.6) & 5.9(3.4)/5.7(2.2) & 5.7(1.9) & 5.7(1.5) \\ 
   & p=60 & 5.2(0.3)/5.2(0.3)/5.6(0.8) & - & 5.6(0.8) & 6(13)/6(12.1) & 5.9(7.8)/5.7(2.6) & 5.7(2.7) & 5.6(1.3) \\ 
   & p=180 & 4.7(0.1)/4.7(0.1)/5.2(0.6) & - & 5.2(0.6) & 5.9(21.7)/5.9(19) & 5.9(26.9)/5.4(2.9) & 5.4(4.5) & 5.3(1.7) \\ 
  \midrule\multirow{4}[2]{*}{lsnr} & p=14 & 4.5(1)/4.5(1)/4.5(0.6) & 4.5(0.6) & 4.5(0.6) & 5.3(3.4)/5.4(3.8) & 4.8(1.3)/4.6(1) & 4.5(0.7) & 4.5(0.8) \\ 
   & p=30 & 4.1(0.3)/4.1(0.3)/4.3(0.6) & 4.3(0.6) & 4.3(0.6) & 5.1(7.2)/5.1(7.3) & 4.8(3.6)/4.3(1.2) & 4.3(1) & 4.3(0.9) \\ 
   & p=60 & 4.1(0.2)/4.1(0.2)/4.2(0.5) & - & 4.2(0.5) & 5(11.2)/4.9(10.1) & 4.8(8)/4.2(1.2) & 4.2(1.2) & 4.2(1) \\ 
   & p=180 & 4(0.1)/4(0.1)/4.1(0.4) & - & 4.1(0.4) & 4.7(18.7)/4.6(15.6) & 4.8(27)/4.1(1) & 4.1(1.4) & 4.1(0.9) \\ 
   \bottomrule 
\end{tabular}
}
\end{table}

% latex table generated in R 3.6.2 by xtable 1.8-4 package
% Sat Dec 21 23:36:18 2019
\begin{table}[ht]
\centering
\caption{The performance of BOSS compared to other methods, Sparse-Ex4, $\rho$=0.5, n=200} 
\scalebox{0.7}{
\begin{tabular}{|c|c|ccccccc|}
  \toprule 
 \multicolumn{1}{|c}{} &       & BOSS  & BS    & FS    & LASSO & Gamma LASSO & SparseNet & \multicolumn{1}{c|}{rLASSO} \\
 \multicolumn{1}{|c}{} &       & C$_p$-hdf/AICc-hdf/CV & CV    & CV    & AICc/CV & AICc/CV & CV    & \multicolumn{1}{c|}{CV}  \\
 \cmidrule{3-9}\multicolumn{1}{|c}{} &       & \multicolumn{7}{c|}{\% worse than the best possible BOSS}  \\
 \midrule\multirow{4}[2]{*}{hsnr} & p=14 & 34/33/25 & 22 & 26 & 29/28 & 17/25 & 25 & 33 \\ 
   & p=30 & 23/19/24 & 23 & 28 & 49/47 & 34/26 & 21 & 32 \\ 
   & p=60 & 18/17/21 & - & 24 & 56/54 & 52/21 & 17 & 28 \\ 
   & p=180 & 29/4/15 & - & 16 & 82/81 & 170/16 & 13 & 15 \\ 
  \midrule\multirow{4}[2]{*}{msnr} & p=14 & 27/25/29 & 20 & 21 & 38/37 & 25/20 & 17 & 24 \\ 
   & p=30 & 24/19/34 & 24 & 30 & 74/72 & 52/26 & 16 & 52 \\ 
   & p=60 & 17/13/33 & - & 28 & 90/89 & 78/26 & 14 & 62 \\ 
   & p=180 & 52/14/37 & - & 29 & 137/137 & 263/29 & 17 & 91 \\ 
  \midrule\multirow{4}[2]{*}{lsnr} & p=14 & 37/35/31 & 24 & 30 & 30/29 & 23/29 & 26 & 33 \\ 
   & p=30 & 37/35/32 & 23 & 34 & 35/35 & 30/31 & 23 & 38 \\ 
   & p=60 & 37/38/32 & - & 36 & 38/40 & 53/34 & 24 & 42 \\ 
   & p=180 & 49/31/26 & - & 34 & 33/35 & 120/31 & 20 & 38 \\ 
   \midrule 
 \multicolumn{1}{|c}{} &       & \multicolumn{7}{c|}{Relative efficiency} \\
\midrule\multirow{4}[2]{*}{hsnr} & p=14 & 0.88/0.88/0.94 & 0.97 & 0.93 & 0.91/0.92 & 1/0.94 & 0.94 & 0.89 \\ 
   & p=30 & 0.97/1/0.96 & 0.97 & 0.93 & 0.8/0.81 & 0.89/0.95 & 0.99 & 0.9 \\ 
   & p=60 & 0.99/1/0.96 & - & 0.94 & 0.75/0.76 & 0.77/0.96 & 1 & 0.91 \\ 
   & p=180 & 0.81/1/0.91 & - & 0.9 & 0.57/0.58 & 0.39/0.9 & 0.92 & 0.91 \\ 
  \midrule\multirow{4}[2]{*}{msnr} & p=14 & 0.92/0.94/0.91 & 0.97 & 0.96 & 0.84/0.85 & 0.93/0.97 & 1 & 0.94 \\ 
   & p=30 & 0.93/0.98/0.87 & 0.93 & 0.89 & 0.66/0.67 & 0.76/0.92 & 1 & 0.76 \\ 
   & p=60 & 0.97/1/0.85 & - & 0.88 & 0.6/0.6 & 0.64/0.9 & 1 & 0.7 \\ 
   & p=180 & 0.74/1/0.83 & - & 0.88 & 0.48/0.48 & 0.31/0.88 & 0.97 & 0.59 \\ 
  \midrule\multirow{4}[2]{*}{lsnr} & p=14 & 0.9/0.91/0.94 & 0.99 & 0.95 & 0.95/0.95 & 1/0.95 & 0.98 & 0.92 \\ 
   & p=30 & 0.9/0.91/0.93 & 1 & 0.92 & 0.91/0.91 & 0.94/0.93 & 1 & 0.89 \\ 
   & p=60 & 0.91/0.9/0.94 & - & 0.91 & 0.9/0.89 & 0.81/0.93 & 1 & 0.87 \\ 
   & p=180 & 0.81/0.92/0.96 & - & 0.9 & 0.91/0.89 & 0.55/0.92 & 1 & 0.87 \\ 
   \midrule 
 \multicolumn{1}{|c}{} &       & \multicolumn{7}{c|}{Sparsistency (number of extra variables)} \\
\midrule\multirow{4}[2]{*}{hsnr} & p=14 & 5.2(1.9)/5.1(1.4)/5.5(0.9) & 5.5(0.9) & 5.4(1) & 5.9(4.6)/6(5.4) & 5.7(1.3)/5.4(1.9) & 5.4(1.4) & 5.2(1.4) \\ 
   & p=30 & 4.5(0.9)/4.4(0.2)/5(1) & 5(1) & 4.8(1.1) & 5.7(10.4)/5.7(12) & 5.4(3.6)/4.8(2.3) & 4.8(2.1) & 4.5(1.8) \\ 
   & p=60 & 4.4(0.3)/4.2(0.1)/4.7(0.8) & - & 4.6(0.8) & 5.6(15)/5.7(17.8) & 5.5(7.4)/4.7(2.3) & 4.8(2.8) & 4.4(1.6) \\ 
   & p=180 & 4.3(11.4)/4(0)/4.2(0.5) & - & 4.1(0.5) & 5.1(20.2)/5.3(27.9) & 5.2(35.3)/4.2(2.1) & 4.2(3.5) & 4.1(0.8) \\ 
  \midrule\multirow{4}[2]{*}{msnr} & p=14 & 4.7(1.6)/4.6(1.3)/4.4(0.6) & 4.5(0.7) & 4.5(0.7) & 5.4(4.2)/5.5(5) & 4.7(1.1)/4.6(1.2) & 4.5(0.8) & 4.6(1.1) \\ 
   & p=30 & 4.2(1)/4.1(0.5)/4.2(0.7) & 4.2(0.7) & 4.2(1) & 5(9.3)/5.1(10.7) & 4.5(3.1)/4.2(1.4) & 4.2(1.2) & 4.3(2.9) \\ 
   & p=60 & 4.1(0.4)/4(0.2)/4.1(0.6) & - & 4.1(0.8) & 4.8(13.6)/4.9(15.9) & 4.5(6.8)/4.1(1.6) & 4.1(1.8) & 4.1(3.8) \\ 
   & p=180 & 4.1(10.7)/4(0.1)/4(0.4) & - & 4(0.6) & 4.5(19.3)/4.6(26.2) & 4.5(35.4)/4(1.7) & 4.1(2.6) & 4(5.2) \\ 
  \midrule\multirow{4}[2]{*}{lsnr} & p=14 & 4(2.7)/3.8(2.2)/3.3(0.8) & 3.3(0.8) & 3.3(1) & 4.5(3.6)/4.7(4.1) & 3.7(1.4)/3.5(1.7) & 3.4(1.4) & 3.7(2) \\ 
   & p=30 & 2.7(2.3)/2.3(1)/2.7(1.3) & 2.6(1) & 2.6(1.5) & 3.6(6.9)/3.6(7.5) & 3.4(3.7)/2.9(3.3) & 2.8(2.9) & 2.9(4) \\ 
   & p=60 & 1.9(1.1)/1.5(0.3)/2.3(1.2) & - & 2(1.1) & 2.9(8.5)/2.9(9.5) & 3.4(7.8)/2.4(4) & 2.5(3.9) & 2.4(5.2) \\ 
   & p=180 & 1.7(12.7)/1(0.2)/1.6(0.9) & - & 1.3(0.8) & 2.2(10.7)/2.1(13.3) & 3.5(32.4)/1.9(6.4) & 2.1(6.5) & 1.9(8) \\ 
   \bottomrule 
\end{tabular}
}
\end{table}

% latex table generated in R 3.6.1 by xtable 1.8-4 package
% Sat Nov  9 19:30:30 2019
\begin{table}[ht]
\centering
\caption{The performance of BOSS compared to other methods, Sparse-Ex4, $\rho$=0.5, n=2000} 

\scalebox{0.7}{
\begin{tabular}{|c|c|ccccccc|}
  \toprule 
 \multicolumn{1}{|c}{} &       & BOSS  & BS    & FS    & lasso & gamma lasso & SparseNet & \multicolumn{1}{c|}{rlasso} \\
 \multicolumn{1}{|c}{} &       & C$_p$-hdf/AICc-hdf/CV & CV    & CV    & AICc/CV & AICc/CV & CV    & \multicolumn{1}{c|}{CV}  \\
 \cmidrule{3-9}\multicolumn{1}{|c}{} &       & \multicolumn{7}{c|}{\% worse than the best possible BOSS}  \\
 \midrule\multirow{4}[2]{*}{hsnr} & p=14 & 12/11/25 & 19 & 18 & 48/47 & 19/16 & 13 & 18 \\ 
   & p=30 & 8/7/29 & 21 & 23 & 86/83 & 39/20 & 12 & 23 \\ 
   & p=60 & 7/7/28 & - & 23 & 125/120 & 73/22 & 14 & 24 \\ 
   & p=180 & 7/6/28 & - & 21 & 174/171 & 171/23 & 12 & 25 \\ 
  \midrule\multirow{4}[2]{*}{msnr} & p=14 & 33/33/26 & 18 & 20 & 44/42 & 20/23 & 18 & 32 \\ 
   & p=30 & 40/39/33 & 21 & 31 & 71/69 & 43/35 & 26 & 54 \\ 
   & p=60 & 42/43/35 & - & 42 & 93/90 & 76/44 & 31 & 71 \\ 
   & p=180 & 41/42/33 & - & 46 & 105/102 & 153/48 & 37 & 69 \\ 
  \midrule\multirow{4}[2]{*}{lsnr} & p=14 & 22/22/30 & 21 & 21 & 36/35 & 24/20 & 17 & 22 \\ 
   & p=30 & 14/13/30 & 22 & 23 & 61/59 & 55/20 & 15 & 24 \\ 
   & p=60 & 11/10/28 & - & 21 & 89/85 & 104/21 & 13 & 25 \\ 
   & p=180 & 8/8/27 & - & 20 & 125/121 & 242/19 & 10 & 29 \\ 
   \midrule 
 \multicolumn{1}{|c}{} &       & \multicolumn{7}{c|}{Relative efficiency} \\
\midrule\multirow{4}[2]{*}{hsnr} & p=14 & 1/1/0.89 & 0.94 & 0.95 & 0.75/0.76 & 0.94/0.96 & 0.99 & 0.94 \\ 
   & p=30 & 1/1/0.83 & 0.89 & 0.87 & 0.58/0.59 & 0.77/0.89 & 0.96 & 0.88 \\ 
   & p=60 & 1/1/0.83 & - & 0.87 & 0.47/0.48 & 0.62/0.87 & 0.93 & 0.86 \\ 
   & p=180 & 1/1/0.83 & - & 0.88 & 0.39/0.39 & 0.39/0.87 & 0.95 & 0.85 \\ 
  \midrule\multirow{4}[2]{*}{msnr} & p=14 & 0.88/0.88/0.93 & 1 & 0.98 & 0.82/0.83 & 0.99/0.96 & 1 & 0.89 \\ 
   & p=30 & 0.87/0.87/0.91 & 1 & 0.93 & 0.71/0.72 & 0.85/0.9 & 0.97 & 0.79 \\ 
   & p=60 & 0.92/0.92/0.98 & - & 0.92 & 0.68/0.69 & 0.75/0.91 & 1 & 0.77 \\ 
   & p=180 & 0.94/0.94/1 & - & 0.91 & 0.65/0.66 & 0.53/0.9 & 0.97 & 0.78 \\ 
  \midrule\multirow{4}[2]{*}{lsnr} & p=14 & 0.96/0.96/0.9 & 0.97 & 0.96 & 0.86/0.87 & 0.94/0.98 & 1 & 0.96 \\ 
   & p=30 & 0.99/1/0.87 & 0.93 & 0.92 & 0.7/0.71 & 0.73/0.95 & 0.99 & 0.91 \\ 
   & p=60 & 1/1/0.86 & - & 0.91 & 0.58/0.59 & 0.54/0.91 & 0.98 & 0.88 \\ 
   & p=180 & 0.99/1/0.85 & - & 0.9 & 0.48/0.49 & 0.32/0.91 & 0.98 & 0.83 \\ 
   \midrule 
 \multicolumn{1}{|c}{} &       & \multicolumn{7}{c|}{Sparsistency (number of extra variables)} \\
\midrule\multirow{4}[2]{*}{hsnr} & p=14 & 6(0.6)/6(0.5)/6(0.6) & 6(0.7) & 6(0.6) & 6(4.7)/6(5.5) & 6(1)/6(1) & 6(0.6) & 6(0.7) \\ 
   & p=30 & 6(0.2)/6(0.2)/6(0.6) & 6(0.6) & 6(0.6) & 6(10.9)/6(11.3) & 6(2.7)/6(1.4) & 6(0.8) & 6(0.7) \\ 
   & p=60 & 6(0.1)/6(0.1)/6(0.5) & - & 6(0.5) & 6(18)/6(17.7) & 6(6.4)/6(1.5) & 6(1.1) & 6(0.7) \\ 
   & p=180 & 6(0.1)/6(0.1)/6(0.3) & - & 6(0.3) & 6(32.2)/6(29.8) & 6(22.3)/6(1.3) & 6(1.3) & 6(0.5) \\ 
  \midrule\multirow{4}[2]{*}{msnr} & p=14 & 5.9(2.1)/5.9(2)/6(0.7) & 5.9(0.7) & 5.9(0.9) & 6(4.8)/6(5.5) & 6(1.2)/5.9(1.9) & 5.9(1.1) & 5.9(1.6) \\ 
   & p=30 & 5.6(1.1)/5.6(1)/5.9(0.8) & 5.9(0.7) & 5.8(1) & 6(10.9)/6(11.3) & 6(3.3)/5.8(2.8) & 5.8(2) & 5.7(2.3) \\ 
   & p=60 & 5.4(0.4)/5.3(0.3)/5.8(0.8) & - & 5.6(1.1) & 6(18)/6(17.6) & 6(7.7)/5.6(3.7) & 5.7(3.1) & 5.4(2.6) \\ 
   & p=180 & 4.8(0.2)/4.8(0.2)/5.5(1) & - & 5.1(0.8) & 5.9(31.7)/5.9(29) & 6(26.5)/5.1(4.5) & 5.2(4.7) & 4.8(2.1) \\ 
  \midrule\multirow{4}[2]{*}{lsnr} & p=14 & 4.6(1.4)/4.6(1.3)/4.6(0.7) & 4.6(0.7) & 4.6(0.8) & 5.5(4.4)/5.5(5) & 4.9(1.3)/4.6(1.2) & 4.5(0.8) & 4.5(0.9) \\ 
   & p=30 & 4.2(0.5)/4.2(0.4)/4.3(0.7) & 4.3(0.6) & 4.2(0.7) & 5.2(9.6)/5.3(9.9) & 4.8(3.6)/4.3(1.1) & 4.3(1) & 4.2(1) \\ 
   & p=60 & 4.1(0.2)/4.1(0.2)/4.2(0.5) & - & 4.1(0.5) & 5(15.5)/5(14.7) & 4.8(8.2)/4.1(1.2) & 4.1(1.2) & 4.1(1) \\ 
   & p=180 & 4(0.1)/4(0.1)/4(0.4) & - & 4(0.4) & 4.6(26.1)/4.6(23.6) & 4.8(27.9)/4(1) & 4.1(1.3) & 4(1) \\ 
   \bottomrule 
\end{tabular}
}
\end{table}

% latex table generated in R 3.6.2 by xtable 1.8-4 package
% Sat Dec 21 23:36:25 2019
\begin{table}[ht]
\centering
\caption{The performance of BOSS compared to other methods, Sparse-Ex4, $\rho$=0.9, n=200} 
\scalebox{0.7}{
\begin{tabular}{|c|c|ccccccc|}
  \toprule 
 \multicolumn{1}{|c}{} &       & BOSS  & BS    & FS    & LASSO & Gamma LASSO & SparseNet & \multicolumn{1}{c|}{rLASSO} \\
 \multicolumn{1}{|c}{} &       & C$_p$-hdf/AICc-hdf/CV & CV    & CV    & AICc/CV & AICc/CV & CV    & \multicolumn{1}{c|}{CV}  \\
 \cmidrule{3-9}\multicolumn{1}{|c}{} &       & \multicolumn{7}{c|}{\% worse than the best possible BOSS}  \\
 \midrule\multirow{4}[2]{*}{hsnr} & p=14 & 34/33/29 & 24 & 46 & 45/44 & 33/43 & 40 & 53 \\ 
   & p=30 & 25/21/33 & 21 & 56 & 73/71 & 40/36 & 28 & 58 \\ 
   & p=60 & 23/22/37 & - & 57 & 95/91 & 50/31 & 22 & 64 \\ 
   & p=180 & 27/7/34 & - & 68 & 123/110 & 91/2 & -10 & 62 \\ 
  \midrule\multirow{4}[2]{*}{msnr} & p=14 & 29/27/28 & 15 & 33 & 41/39 & 23/28 & 16 & 39 \\ 
   & p=30 & 25/19/30 & -4 & 52 & 61/60 & 29/23 & 3 & 56 \\ 
   & p=60 & 22/14/24 & - & 83 & 90/86 & 57/38 & 16 & 91 \\ 
   & p=180 & 31/17/26 & - & 66 & 78/66 & 78/26 & 6 & 70 \\ 
  \midrule\multirow{4}[2]{*}{lsnr} & p=14 & 41/39/31 & 27 & 54 & 51/50 & 46/47 & 38 & 55 \\ 
   & p=30 & 39/32/27 & 18 & 78 & 71/72 & 58/66 & 44 & 75 \\ 
   & p=60 & 29/28/23 & - & 81 & 80/80 & 79/79 & 57 & 84 \\ 
   & p=180 & 37/17/18 & - & 36 & 33/33 & 110/37 & 35 & 36 \\ 
   \midrule 
 \multicolumn{1}{|c}{} &       & \multicolumn{7}{c|}{Relative efficiency} \\
\midrule\multirow{4}[2]{*}{hsnr} & p=14 & 0.93/0.94/0.96 & 1 & 0.85 & 0.86/0.86 & 0.94/0.87 & 0.89 & 0.81 \\ 
   & p=30 & 0.97/1/0.91 & 1 & 0.78 & 0.7/0.71 & 0.87/0.9 & 0.95 & 0.77 \\ 
   & p=60 & 0.99/1/0.89 & - & 0.77 & 0.62/0.64 & 0.81/0.93 & 0.99 & 0.74 \\ 
   & p=180 & 0.71/0.84/0.68 & - & 0.54 & 0.41/0.43 & 0.47/0.88 & 1 & 0.56 \\ 
  \midrule\multirow{4}[2]{*}{msnr} & p=14 & 0.89/0.91/0.9 & 1 & 0.87 & 0.82/0.83 & 0.93/0.9 & 0.99 & 0.83 \\ 
   & p=30 & 0.77/0.81/0.74 & 1 & 0.63 & 0.6/0.6 & 0.75/0.78 & 0.93 & 0.62 \\ 
   & p=60 & 0.93/1/0.92 & - & 0.62 & 0.6/0.61 & 0.73/0.83 & 0.99 & 0.6 \\ 
   & p=180 & 0.81/0.9/0.84 & - & 0.64 & 0.59/0.64 & 0.59/0.84 & 1 & 0.62 \\ 
  \midrule\multirow{4}[2]{*}{lsnr} & p=14 & 0.9/0.91/0.97 & 1 & 0.83 & 0.84/0.84 & 0.87/0.86 & 0.92 & 0.82 \\ 
   & p=30 & 0.85/0.89/0.93 & 1 & 0.66 & 0.69/0.69 & 0.75/0.71 & 0.82 & 0.67 \\ 
   & p=60 & 0.95/0.97/1 & - & 0.68 & 0.69/0.68 & 0.69/0.69 & 0.78 & 0.67 \\ 
   & p=180 & 0.86/1/0.99 & - & 0.86 & 0.88/0.88 & 0.56/0.86 & 0.87 & 0.86 \\ 
   \midrule 
 \multicolumn{1}{|c}{} &       & \multicolumn{7}{c|}{Sparsistency (number of extra variables)} \\
\midrule\multirow{4}[2]{*}{hsnr} & p=14 & 5.8(2.9)/5.7(2.5)/5.6(1.6) & 5.6(0.8) & 5.3(2) & 6(6.5)/6(7.2) & 5.5(2.3)/5.4(3.4) & 5.3(2.4) & 5.4(3.2) \\ 
   & p=30 & 5.2(3.7)/5.1(2.8)/5.3(3.8) & 5(1) & 4.8(4.1) & 5.8(17.8)/5.8(19.5) & 4.9(4.5)/4.5(3.4) & 4.4(2.7) & 4.8(6.8) \\ 
   & p=60 & 5(4.2)/4.8(3)/4.9(4.7) & - & 4.5(4.3) & 5.6(30.2)/5.7(35.1) & 4.9(8.8)/4.3(3.6) & 4.3(3.3) & 4.3(8.7) \\ 
   & p=180 & 4.4(13.2)/4.2(2.4)/4.3(4.3) & - & 4.3(8) & 4.6(44.2)/4.8(64.9) & 4.4(29.2)/4.1(4) & 4.1(3.1) & 4.1(15.4) \\ 
  \midrule\multirow{4}[2]{*}{msnr} & p=14 & 5(2.5)/4.9(2.2)/4.7(1.4) & 4.6(0.7) & 4.9(2.4) & 5.6(6.3)/5.7(7) & 4.6(1.8)/4.9(3.3) & 4.6(2.4) & 5(3.9) \\ 
   & p=30 & 4.5(4.5)/4.4(3.6)/4.5(3.7) & 4.2(0.7) & 4.6(6.5) & 5.2(16.9)/5.3(18.5) & 4.4(4.7)/4.5(7) & 4.4(5.8) & 4.8(11.2) \\ 
   & p=60 & 4.3(5.9)/4.2(5.1)/4.3(5.7) & - & 3.7(6.3) & 4.8(28.3)/5(32.8) & 4.4(10.4)/4.2(11.8) & 4.3(11.7) & 4.1(17.8) \\ 
   & p=180 & 4.1(24.5)/3.6(7.6)/3.7(9.2) & - & 3(8.8) & 3.8(37.7)/4.3(59.6) & 4.3(37.5)/3.6(20.4) & 3.8(22.8) & 3.2(26) \\ 
  \midrule\multirow{4}[2]{*}{lsnr} & p=14 & 4.2(3)/4(2.7)/3.8(1.9) & 3.3(0.7) & 3.9(2.9) & 4.9(5.9)/5.1(6.5) & 3.8(2.6)/4.1(4.1) & 3.7(3.2) & 4.4(4.7) \\ 
   & p=30 & 3.1(5.5)/2.7(3.9)/3.2(5) & 2.7(0.9) & 2.2(4.4) & 3(10.1)/3(11) & 3.3(7.6)/2.8(8) & 2.9(8) & 2.6(8.5) \\ 
   & p=60 & 2.2(5.3)/1.9(3.9)/2.6(6.8) & - & 0.9(2) & 1.2(7.4)/1.2(8.4) & 2.8(12.8)/1.3(6.6) & 2(10) & 1.1(6.4) \\ 
   & p=180 & 1.4(18.2)/0.7(1.7)/1.1(5.5) & - & 0.2(0.6) & 0.3(4.8)/0.3(5.4) & 2.3(40.2)/0.3(4.4) & 0.6(9) & 0.3(4.4) \\ 
   \bottomrule 
\end{tabular}
}
\end{table}

% latex table generated in R 3.6.1 by xtable 1.8-4 package
% Sat Nov  9 19:30:39 2019
\begin{table}[ht]
\centering
\caption{The performance of BOSS compared to other methods, Sparse-Ex4, $\rho$=0.9, n=2000} 

\scalebox{0.7}{
\begin{tabular}{|c|c|ccccccc|}
  \toprule 
 \multicolumn{1}{|c}{} &       & BOSS  & BS    & FS    & LASSO & Gamma LASSO & SparseNet & \multicolumn{1}{c|}{rLASSO} \\
 \multicolumn{1}{|c}{} &       & C$_p$-hdf/AICc-hdf/CV & CV    & CV    & AICc/CV & AICc/CV & CV    & \multicolumn{1}{c|}{CV}  \\
 \cmidrule{3-9}\multicolumn{1}{|c}{} &       & \multicolumn{7}{c|}{\% worse than the best possible BOSS}  \\
 \midrule\multirow{4}[2]{*}{hsnr} & p=14 & 33/33/27 & 16 & 20 & 53/51 & 17/20 & 15 & 31 \\ 
   & p=30 & 33/32/33 & 16 & 33 & 108/106 & 26/22 & 12 & 55 \\ 
   & p=60 & 27/26/30 & - & 48 & 157/154 & 37/22 & 9 & 86 \\ 
   & p=180 & 15/15/25 & - & 90 & 226/223 & 68/33 & 10 & 154 \\ 
  \midrule\multirow{4}[2]{*}{msnr} & p=14 & 28/28/22 & 18 & 45 & 53/51 & 26/39 & 30 & 57 \\ 
   & p=30 & 22/22/26 & 23 & 89 & 108/105 & 56/75 & 55 & 112 \\ 
   & p=60 & 21/21/27 & - & 114 & 166/162 & 100/107 & 92 & 128 \\ 
   & p=180 & 25/25/30 & - & 107 & 253/250 & 190/110 & 105 & 110 \\ 
  \midrule\multirow{4}[2]{*}{lsnr} & p=14 & 33/33/32 & 20 & 29 & 45/44 & 24/28 & 19 & 40 \\ 
   & p=30 & 28/27/34 & 16 & 40 & 85/82 & 44/30 & 12 & 63 \\ 
   & p=60 & 19/19/30 & - & 59 & 122/119 & 75/32 & 9 & 95 \\ 
   & p=180 & 15/14/27 & - & 104 & 179/176 & 167/48 & 20 & 164 \\ 
   \midrule 
 \multicolumn{1}{|c}{} &       & \multicolumn{7}{c|}{Relative efficiency} \\
\midrule\multirow{4}[2]{*}{hsnr} & p=14 & 0.86/0.86/0.9 & 0.99 & 0.96 & 0.75/0.76 & 0.98/0.95 & 1 & 0.87 \\ 
   & p=30 & 0.84/0.84/0.84 & 0.96 & 0.84 & 0.54/0.54 & 0.89/0.92 & 1 & 0.72 \\ 
   & p=60 & 0.86/0.86/0.84 & - & 0.74 & 0.42/0.43 & 0.8/0.89 & 1 & 0.58 \\ 
   & p=180 & 0.95/0.96/0.88 & - & 0.58 & 0.34/0.34 & 0.65/0.82 & 1 & 0.43 \\ 
  \midrule\multirow{4}[2]{*}{msnr} & p=14 & 0.92/0.92/0.97 & 1 & 0.81 & 0.77/0.78 & 0.94/0.85 & 0.91 & 0.75 \\ 
   & p=30 & 0.99/1/0.96 & 0.99 & 0.64 & 0.58/0.59 & 0.78/0.7 & 0.78 & 0.58 \\ 
   & p=60 & 1/1/0.95 & - & 0.56 & 0.45/0.46 & 0.6/0.58 & 0.63 & 0.53 \\ 
   & p=180 & 1/1/0.96 & - & 0.6 & 0.35/0.36 & 0.43/0.6 & 0.61 & 0.6 \\ 
  \midrule\multirow{4}[2]{*}{lsnr} & p=14 & 0.9/0.9/0.9 & 1 & 0.92 & 0.82/0.83 & 0.96/0.93 & 1 & 0.85 \\ 
   & p=30 & 0.87/0.88/0.83 & 0.97 & 0.8 & 0.61/0.61 & 0.78/0.86 & 1 & 0.69 \\ 
   & p=60 & 0.92/0.92/0.84 & - & 0.69 & 0.49/0.5 & 0.62/0.83 & 1 & 0.56 \\ 
   & p=180 & 1/1/0.9 & - & 0.56 & 0.41/0.41 & 0.43/0.77 & 0.95 & 0.43 \\ 
   \midrule 
 \multicolumn{1}{|c}{} &       & \multicolumn{7}{c|}{Sparsistency (number of extra variables)} \\
\midrule\multirow{4}[2]{*}{hsnr} & p=14 & 6(1.4)/6(1.4)/6(0.6) & 6(0.7) & 6(0.9) & 6(6.6)/6(7.2) & 6(1.2)/6(1.8) & 6(0.9) & 6(1.8) \\ 
   & p=30 & 6(0.5)/6(0.4)/6(0.7) & 6(0.6) & 6(1.3) & 6(17.8)/6(18.9) & 6(2.6)/6(2.5) & 6(1.6) & 6(3) \\ 
   & p=60 & 6(0.4)/6(0.4)/6(0.8) & - & 6(2) & 6(34.2)/6(35.3) & 6(5.8)/6(3.9) & 6(3.1) & 6(5.1) \\ 
   & p=180 & 6(1.4)/6(1.4)/6(1.7) & - & 5.9(3.8) & 6(72.7)/6(73.8) & 6(17.2)/6(9.4) & 6(8.7) & 5.9(11.5) \\ 
  \midrule\multirow{4}[2]{*}{msnr} & p=14 & 6(2)/6(2)/6(1.3) & 6(0.7) & 5.9(2.4) & 6(6.6)/6(7.2) & 6(2.4)/5.9(4.1) & 5.9(2.9) & 5.8(4.2) \\ 
   & p=30 & 5.9(2.4)/5.9(2.4)/5.9(2.6) & 5.9(0.7) & 5.4(3.4) & 6(17.8)/6(18.8) & 5.9(5.9)/5.6(7.2) & 5.6(6.2) & 5.2(5.9) \\ 
   & p=60 & 5.8(4.8)/5.8(4.8)/5.9(5.6) & - & 4.7(1.9) & 6(34)/6(35.1) & 5.8(12.4)/4.9(7.7) & 5(8.3) & 4.4(3.1) \\ 
   & p=180 & 5.5(12.9)/5.5(12.7)/5.7(16.2) & - & 4.2(0.6) & 5.7(68.6)/5.7(69.3) & 5.4(31.9)/4.2(3.4) & 4.3(5) & 4.1(0.6) \\ 
  \midrule\multirow{4}[2]{*}{lsnr} & p=14 & 5(2.5)/5(2.4)/4.7(1.1) & 4.6(0.7) & 4.6(1.5) & 5.7(6.4)/5.7(7) & 4.6(1.5)/4.7(2.4) & 4.5(1.5) & 4.7(2.8) \\ 
   & p=30 & 4.5(2.1)/4.4(1.9)/4.4(1.7) & 4.3(0.6) & 4.3(2.2) & 5.4(16.6)/5.4(17.6) & 4.5(3.7)/4.3(3.5) & 4.2(2.5) & 4.5(5.3) \\ 
   & p=60 & 4.2(1.9)/4.2(1.8)/4.3(2.3) & - & 4.2(3.2) & 5.1(30.6)/5.1(31.3) & 4.3(8.1)/4.2(5.8) & 4.2(5) & 4.4(9.8) \\ 
   & p=180 & 4.1(3.7)/4.1(3.6)/4.1(4.4) & - & 3.7(3.7) & 4.6(60.3)/4.6(60.5) & 4.3(26.8)/4.1(14.8) & 4.2(14.2) & 3.7(15.1) \\ 
   \bottomrule 
\end{tabular}
}
\end{table}

% latex table generated in R 3.6.1 by xtable 1.8-4 package
% Sat Nov  9 19:30:39 2019
\begin{table}[ht]
\centering
\caption{The performance of BOSS compared to other methods, Dense, $\rho$=0, n=200} 

\scalebox{0.7}{
\begin{tabular}{|c|c|ccccccc|}
  \toprule 
 \multicolumn{1}{|c}{} &       & BOSS  & BS    & FS    & lasso & gamma lasso & SparseNet & \multicolumn{1}{c|}{rlasso} \\
 \multicolumn{1}{|c}{} &       & C$_p$-hdf/AICc-hdf/CV & CV    & CV    & AICc/CV & AICc/CV & CV    & \multicolumn{1}{c|}{CV}  \\
 \cmidrule{3-9}\multicolumn{1}{|c}{} &       & \multicolumn{7}{c|}{\% worse than the best possible BOSS}  \\
 \midrule\multirow{4}[2]{*}{hsnr} & p=14 & 0/0/0 & 0 & 0 & 0/0 & 0/1 & 0 & 75 \\ 
   & p=30 & 2/2/7 & 8 & 7 & 1/1 & 8/2 & 4 & 5 \\ 
   & p=60 & 9/11/11 & - & 11 & 0/-2 & -1/-1 & 1 & 3 \\ 
   & p=180 & 18/13/11 & - & 12 & 18/12 & 5/5 & 3 & 13 \\ 
  \midrule\multirow{4}[2]{*}{msnr} & p=14 & 0/0/6 & 6 & 6 & 1/1 & 7/1 & 3 & 21 \\ 
   & p=30 & 4/5/10 & 10 & 10 & 0/-1 & 8/2 & 4 & 4 \\ 
   & p=60 & 11/12/12 & - & 12 & -3/-5 & -1/-2 & -1 & 0 \\ 
   & p=180 & 19/14/13 & - & 12 & 3/0 & 13/1 & 2 & 6 \\ 
  \midrule\multirow{4}[2]{*}{lsnr} & p=14 & 7/9/20 & 19 & 20 & 3/3 & 17/8 & 10 & 13 \\ 
   & p=30 & 15/19/16 & 16 & 16 & -8/-8 & 5/-1 & -3 & -1 \\ 
   & p=60 & 16/14/14 & - & 14 & -9/-8 & 11/-4 & -4 & -3 \\ 
   & p=180 & 22/7/8 & - & 9 & -5/-3 & 65/0 & -1 & 2 \\ 
   \midrule 
 \multicolumn{1}{|c}{} &       & \multicolumn{7}{c|}{Relative efficiency} \\
\midrule\multirow{4}[2]{*}{hsnr} & p=14 & 1/1/1 & 1 & 1 & 1/1 & 1/0.99 & 1 & 0.57 \\ 
   & p=30 & 0.99/0.99/0.95 & 0.94 & 0.94 & 1/1 & 0.93/0.99 & 0.97 & 0.97 \\ 
   & p=60 & 0.9/0.88/0.89 & - & 0.89 & 0.98/1 & 1/0.99 & 0.98 & 0.96 \\ 
   & p=180 & 0.88/0.91/0.93 & - & 0.93 & 0.88/0.93 & 0.99/0.99 & 1 & 0.92 \\ 
  \midrule\multirow{4}[2]{*}{msnr} & p=14 & 1/1/0.95 & 0.95 & 0.95 & 1/1 & 0.94/0.99 & 0.97 & 0.83 \\ 
   & p=30 & 0.96/0.95/0.91 & 0.9 & 0.91 & 0.99/1 & 0.92/0.98 & 0.96 & 0.95 \\ 
   & p=60 & 0.86/0.85/0.85 & - & 0.85 & 0.98/1 & 0.96/0.97 & 0.96 & 0.96 \\ 
   & p=180 & 0.84/0.88/0.89 & - & 0.89 & 0.97/1 & 0.88/0.98 & 0.98 & 0.94 \\ 
  \midrule\multirow{4}[2]{*}{lsnr} & p=14 & 0.96/0.94/0.86 & 0.86 & 0.85 & 0.99/1 & 0.88/0.95 & 0.93 & 0.91 \\ 
   & p=30 & 0.8/0.77/0.79 & 0.79 & 0.79 & 1/1 & 0.88/0.93 & 0.94 & 0.93 \\ 
   & p=60 & 0.79/0.8/0.8 & - & 0.8 & 1/1 & 0.83/0.95 & 0.96 & 0.94 \\ 
   & p=180 & 0.78/0.89/0.88 & - & 0.88 & 1/0.98 & 0.58/0.95 & 0.96 & 0.94 \\ 
   \midrule 
 \multicolumn{1}{|c}{} &       & \multicolumn{7}{c|}{Sparsistency (number of extra variables)} \\
\midrule\multirow{4}[2]{*}{hsnr} & p=14 & 14/14/14 & 14 & 14 & 14/14 & 14/14 & 14 & 13 \\ 
   & p=30 & 29.2/29/26 & 25.7 & 25.9 & 28.6/29.2 & 25.3/28.6 & 27.2 & 26.3 \\ 
   & p=60 & 35.8/24.3/28 & - & 27.6 & 40.5/44.9 & 29.8/38.4 & 36.8 & 32.4 \\ 
   & p=180 & 36.9/17/19.3 & - & 19.2 & 47.5/67.6 & 38.7/36.4 & 32.4 & 36.3 \\ 
  \midrule\multirow{4}[2]{*}{msnr} & p=14 & 14/14/13.4 & 13.4 & 13.4 & 13.9/13.9 & 13.4/13.9 & 13.6 & 12.7 \\ 
   & p=30 & 26.9/26/21.1 & 20.7 & 20.9 & 25.1/26.3 & 18.8/24.9 & 23.7 & 22.1 \\ 
   & p=60 & 26/14.9/18.4 & - & 18.3 & 32.2/36.4 & 22.7/29.2 & 29.6 & 24.2 \\ 
   & p=180 & 30.7/7.9/9.9 & - & 9.9 & 33.8/49.1 & 40.6/30.1 & 27.9 & 24.3 \\ 
  \midrule\multirow{4}[2]{*}{lsnr} & p=14 & 12.2/11.8/8.3 & 8.4 & 8.2 & 10.6/11.2 & 7.6/10.4 & 10 & 8.6 \\ 
   & p=30 & 13/7.8/8 & 7.5 & 7.8 & 14/15.1 & 10.3/12.3 & 12.7 & 10.3 \\ 
   & p=60 & 5.4/1/4.2 & - & 4.1 & 15.2/16.6 & 15.3/13.7 & 14.6 & 11.1 \\ 
   & p=180 & 16.6/0.3/1.4 & - & 1.5 & 12.4/15.8 & 41.6/12.2 & 13.5 & 10.7 \\ 
   \bottomrule 
\end{tabular}
}
\end{table}

% latex table generated in R 3.6.1 by xtable 1.8-4 package
% Sat Nov  9 19:30:47 2019
\begin{table}[ht]
\centering
\caption{The performance of BOSS compared to other methods, Dense, $\rho$=0, n=2000} 

\scalebox{0.7}{
\begin{tabular}{|c|c|ccccccc|}
  \toprule 
 \multicolumn{1}{|c}{} &       & BOSS  & BS    & FS    & LASSO & Gamma LASSO & SparseNet & \multicolumn{1}{c|}{rLASSO} \\
 \multicolumn{1}{|c}{} &       & C$_p$-hdf/AICc-hdf/CV & CV    & CV    & AICc/CV & AICc/CV & CV    & \multicolumn{1}{c|}{CV}  \\
 \cmidrule{3-9}\multicolumn{1}{|c}{} &       & \multicolumn{7}{c|}{\% worse than the best possible BOSS}  \\
 \midrule\multirow{4}[2]{*}{hsnr} & p=14 & 0/0/0 & 0 & 0 & 2/2 & 1/4 & 0 & 295 \\ 
   & p=30 & 0/0/1 & 1 & 1 & 1/1 & 3/6 & 1 & 16 \\ 
   & p=60 & 5/5/7 & - & 7 & 1/0 & 0/0 & 3 & 5 \\ 
   & p=180 & 7/7/9 & - & 9 & 15/14 & 3/5 & 2 & 9 \\ 
  \midrule\multirow{4}[2]{*}{msnr} & p=14 & 0/0/0 & 0 & 0 & 0/0 & 0/2 & 0 & 105 \\ 
   & p=30 & 1/1/5 & 5 & 5 & 1/1 & 4/2 & 3 & 7 \\ 
   & p=60 & 8/8/9 & - & 9 & 1/0 & -1/0 & 3 & 4 \\ 
   & p=180 & 8/8/10 & - & 10 & 13/12 & 3/4 & 3 & 7 \\ 
  \midrule\multirow{4}[2]{*}{lsnr} & p=14 & 0/0/4 & 4 & 4 & 0/0 & 3/1 & 3 & 27 \\ 
   & p=30 & 3/3/10 & 10 & 9 & 1/0 & 7/2 & 5 & 7 \\ 
   & p=60 & 13/13/12 & - & 12 & 0/-1 & 0/1 & 3 & 4 \\ 
   & p=180 & 9/9/12 & - & 13 & 9/8 & 14/5 & 4 & 8 \\ 
   \midrule 
 \multicolumn{1}{|c}{} &       & \multicolumn{7}{c|}{Relative efficiency} \\
\midrule\multirow{4}[2]{*}{hsnr} & p=14 & 1/1/1 & 1 & 1 & 0.98/0.98 & 0.99/0.96 & 1 & 0.25 \\ 
   & p=30 & 1/1/0.99 & 0.99 & 0.99 & 0.99/0.99 & 0.97/0.94 & 0.99 & 0.86 \\ 
   & p=60 & 0.95/0.95/0.93 & - & 0.93 & 0.99/0.99 & 1/1 & 0.97 & 0.95 \\ 
   & p=180 & 0.95/0.96/0.94 & - & 0.94 & 0.89/0.9 & 0.99/0.98 & 1 & 0.94 \\ 
  \midrule\multirow{4}[2]{*}{msnr} & p=14 & 1/1/1 & 1 & 1 & 1/1 & 1/0.99 & 1 & 0.49 \\ 
   & p=30 & 1/1/0.96 & 0.96 & 0.96 & 1/1 & 0.97/0.99 & 0.97 & 0.94 \\ 
   & p=60 & 0.92/0.92/0.91 & - & 0.91 & 0.98/0.99 & 1/0.99 & 0.96 & 0.95 \\ 
   & p=180 & 0.95/0.95/0.93 & - & 0.93 & 0.91/0.92 & 1/0.99 & 1 & 0.96 \\ 
  \midrule\multirow{4}[2]{*}{lsnr} & p=14 & 1/1/0.96 & 0.96 & 0.96 & 1/1 & 0.98/0.99 & 0.98 & 0.79 \\ 
   & p=30 & 0.97/0.97/0.91 & 0.91 & 0.92 & 1/1 & 0.94/0.98 & 0.96 & 0.94 \\ 
   & p=60 & 0.88/0.88/0.88 & - & 0.88 & 0.99/1 & 0.99/0.99 & 0.96 & 0.95 \\ 
   & p=180 & 0.95/0.96/0.93 & - & 0.93 & 0.96/0.97 & 0.91/1 & 1 & 0.97 \\ 
   \midrule 
 \multicolumn{1}{|c}{} &       & \multicolumn{7}{c|}{Sparsistency (number of extra variables)} \\
\midrule\multirow{4}[2]{*}{hsnr} & p=14 & 14/14/14 & 14 & 14 & 14/14 & 14/14 & 14 & 13 \\ 
   & p=30 & 30/30/29.8 & 29.8 & 29.8 & 30/30 & 30/30 & 29.9 & 28.8 \\ 
   & p=60 & 50.1/49.6/39.8 & - & 39.7 & 53.1/54.3 & 46.4/50.2 & 44.8 & 41.5 \\ 
   & p=180 & 32.5/31.6/32.3 & - & 32.1 & 88.1/88.8 & 62.2/60.9 & 46.4 & 37.9 \\ 
  \midrule\multirow{4}[2]{*}{msnr} & p=14 & 14/14/14 & 14 & 14 & 14/14 & 14/14 & 14 & 13 \\ 
   & p=30 & 29.8/29.8/28.1 & 28.2 & 28.1 & 29.6/29.8 & 28.9/29.7 & 28.7 & 27.8 \\ 
   & p=60 & 42/41.1/31.3 & - & 31.1 & 48.1/49.1 & 37.1/43.5 & 37.5 & 33.4 \\ 
   & p=180 & 23.9/23.2/24.1 & - & 24.1 & 75.1/74.9 & 51.3/44.6 & 38.9 & 27.9 \\ 
  \midrule\multirow{4}[2]{*}{lsnr} & p=14 & 14/14/13.7 & 13.7 & 13.7 & 14/14 & 13.8/14 & 13.8 & 12.7 \\ 
   & p=30 & 27.8/27.8/22.1 & 21.9 & 22.1 & 26.8/27.5 & 21/26.2 & 24.6 & 22.4 \\ 
   & p=60 & 30/28.5/19.9 & - & 19.5 & 38.5/38.9 & 25.4/31.4 & 29.1 & 24.2 \\ 
   & p=180 & 14.1/13.5/14.1 & - & 14 & 55.3/53.9 & 42/31.8 & 29.7 & 22.2 \\ 
   \bottomrule 
\end{tabular}
}
\end{table}

% latex table generated in R 3.6.2 by xtable 1.8-4 package
% Sat Dec 21 23:36:38 2019
\begin{table}[ht]
\centering
\caption{The performance of BOSS compared to other methods, Dense, $\rho$=0.5, n=200} 
\scalebox{0.7}{
\begin{tabular}{|c|c|ccccccc|}
  \toprule 
 \multicolumn{1}{|c}{} &       & BOSS  & BS    & FS    & LASSO & Gamma LASSO & SparseNet & \multicolumn{1}{c|}{rLASSO} \\
 \multicolumn{1}{|c}{} &       & C$_p$-hdf/AICc-hdf/CV & CV    & CV    & AICc/CV & AICc/CV & CV    & \multicolumn{1}{c|}{CV}  \\
 \cmidrule{3-9}\multicolumn{1}{|c}{} &       & \multicolumn{7}{c|}{\% worse than the best possible BOSS}  \\
 \midrule\multirow{4}[2]{*}{hsnr} & p=14 & 0/0/4 & 0 & 0 & 0/0 & 1/2 & 0 & -1 \\ 
   & p=30 & 1/1/8 & 9 & 8 & 2/1 & 6/2 & 5 & 1 \\ 
   & p=60 & 13/13/12 & - & 12 & 10/7 & 5/5 & 7 & 4 \\ 
   & p=180 & 37/14/13 & - & 16 & 47/27 & 10/13 & 8 & 7 \\ 
  \midrule\multirow{4}[2]{*}{msnr} & p=14 & 0/0/5 & 5 & 4 & 0/0 & 2/1 & 2 & 1 \\ 
   & p=30 & 4/5/10 & 11 & 10 & 4/3 & 8/4 & 8 & 3 \\ 
   & p=60 & 16/15/14 & - & 14 & 11/6 & 4/6 & 7 & 4 \\ 
   & p=180 & 43/14/14 & - & 17 & 37/21 & 18/18 & 11 & 14 \\ 
  \midrule\multirow{4}[2]{*}{lsnr} & p=14 & 5/7/17 & 19 & 17 & 9/7 & 16/8 & 14 & 12 \\ 
   & p=30 & 18/18/17 & 16 & 18 & 10/7 & 10/10 & 11 & 7 \\ 
   & p=60 & 17/13/13 & - & 14 & 6/5 & 19/8 & 8 & 5 \\ 
   & p=180 & 47/4/8 & - & 9 & 2/4 & 79/6 & 6 & 4 \\ 
   \midrule 
 \multicolumn{1}{|c}{} &       & \multicolumn{7}{c|}{Relative efficiency} \\
\midrule\multirow{4}[2]{*}{hsnr} & p=14 & 0.99/0.99/0.95 & 0.99 & 0.99 & 0.99/0.99 & 0.99/0.97 & 0.99 & 1 \\ 
   & p=30 & 0.98/0.98/0.93 & 0.91 & 0.93 & 0.98/0.98 & 0.94/0.98 & 0.94 & 0.98 \\ 
   & p=60 & 0.92/0.92/0.93 & - & 0.93 & 0.95/0.98 & 0.99/0.99 & 0.97 & 1 \\ 
   & p=180 & 0.78/0.94/0.95 & - & 0.92 & 0.73/0.84 & 0.97/0.95 & 0.99 & 1 \\ 
  \midrule\multirow{4}[2]{*}{msnr} & p=14 & 1/1/0.95 & 0.95 & 0.96 & 1/1 & 0.98/0.99 & 0.98 & 0.99 \\ 
   & p=30 & 0.98/0.98/0.93 & 0.92 & 0.93 & 0.99/1 & 0.95/0.98 & 0.95 & 0.99 \\ 
   & p=60 & 0.89/0.9/0.91 & - & 0.91 & 0.93/0.98 & 1/0.98 & 0.97 & 1 \\ 
   & p=180 & 0.77/0.97/0.97 & - & 0.95 & 0.81/0.91 & 0.94/0.94 & 1 & 0.97 \\ 
  \midrule\multirow{4}[2]{*}{lsnr} & p=14 & 0.97/0.96/0.87 & 0.86 & 0.87 & 0.93/0.96 & 0.88/0.94 & 0.9 & 0.91 \\ 
   & p=30 & 0.9/0.9/0.92 & 0.92 & 0.91 & 0.97/1 & 0.98/0.97 & 0.96 & 1 \\ 
   & p=60 & 0.9/0.93/0.93 & - & 0.93 & 0.99/1 & 0.88/0.98 & 0.97 & 1 \\ 
   & p=180 & 0.69/0.97/0.94 & - & 0.94 & 1/0.98 & 0.57/0.96 & 0.96 & 0.98 \\ 
   \midrule 
 \multicolumn{1}{|c}{} &       & \multicolumn{7}{c|}{Sparsistency (number of extra variables)} \\
\midrule\multirow{4}[2]{*}{hsnr} & p=14 & 14/14/14 & 14 & 14 & 14/14 & 14/14 & 14 & 14 \\ 
   & p=30 & 29.7/29.6/26.1 & 25.1 & 26 & 29.1/29.7 & 27/29.2 & 27 & 28 \\ 
   & p=60 & 42/29.1/29.4 & - & 28.6 & 44.7/50.7 & 33.7/43.1 & 35 & 42 \\ 
   & p=180 & 43.6/17/20.2 & - & 19.6 & 52.2/87.1 & 40.3/42.5 & 32.4 & 63.5 \\ 
  \midrule\multirow{4}[2]{*}{msnr} & p=14 & 14/14/13.7 & 13.6 & 13.7 & 14/14 & 13.8/14 & 13.8 & 13.9 \\ 
   & p=30 & 28.3/27.8/21.1 & 18.8 & 20.9 & 26.7/28.2 & 20.3/26.2 & 22.6 & 25.1 \\ 
   & p=60 & 34.4/17.9/19.9 & - & 19.6 & 34.7/43.5 & 24.7/33.7 & 27.1 & 35.8 \\ 
   & p=180 & 36.7/8.3/11 & - & 9.7 & 24.5/61.4 & 39.6/28.9 & 22.1 & 47.6 \\ 
  \midrule\multirow{4}[2]{*}{lsnr} & p=14 & 13.1/12.7/9.4 & 8.7 & 9.3 & 10.8/12 & 8.7/11.4 & 10.3 & 10.6 \\ 
   & p=30 & 13.4/5.7/7.5 & 6.6 & 7.1 & 5.3/12.9 & 11.8/11.8 & 10.3 & 11.4 \\ 
   & p=60 & 4.8/0.6/3.5 & - & 3 & 3.7/11.2 & 15.7/9.4 & 8.9 & 8.7 \\ 
   & p=180 & 19/0.2/1.1 & - & 0.9 & 2.7/8.6 & 38.9/7.2 & 6.9 & 5.9 \\ 
   \bottomrule 
\end{tabular}
}
\end{table}

% latex table generated in R 3.6.1 by xtable 1.8-4 package
% Sat Nov  9 19:30:56 2019
\begin{table}[ht]
\centering
\caption{The performance of BOSS compared to other methods, Dense, $\rho$=0.5, n=2000} 

\scalebox{0.7}{
\begin{tabular}{|c|c|ccccccc|}
  \toprule 
 \multicolumn{1}{|c}{} &       & BOSS  & BS    & FS    & LASSO & Gamma LASSO & SparseNet & \multicolumn{1}{c|}{rLASSO} \\
 \multicolumn{1}{|c}{} &       & C$_p$-hdf/AICc-hdf/CV & CV    & CV    & AICc/CV & AICc/CV & CV    & \multicolumn{1}{c|}{CV}  \\
 \cmidrule{3-9}\multicolumn{1}{|c}{} &       & \multicolumn{7}{c|}{\% worse than the best possible BOSS}  \\
 \midrule\multirow{4}[2]{*}{hsnr} & p=14 & 0/0/0 & 0 & 0 & 2/2 & 2/5 & 0 & 452 \\ 
   & p=30 & 0/0/3 & 0 & 0 & 1/1 & 9/10 & 0 & 26 \\ 
   & p=60 & 6/6/7 & - & 7 & 4/3 & 4/5 & 5 & 7 \\ 
   & p=180 & 8/8/9 & - & 12 & 37/36 & 20/20 & 8 & 38 \\ 
  \midrule\multirow{4}[2]{*}{msnr} & p=14 & 0/0/2 & 0 & 0 & 0/0 & 1/4 & 0 & 179 \\ 
   & p=30 & 0/0/5 & 6 & 5 & 1/0 & 3/5 & 3 & 9 \\ 
   & p=60 & 10/10/10 & - & 10 & 7/6 & 4/5 & 7 & 10 \\ 
   & p=180 & 9/9/11 & - & 14 & 36/34 & 16/18 & 9 & 37 \\ 
  \midrule\multirow{4}[2]{*}{lsnr} & p=14 & 0/0/5 & 2 & 2 & 0/0 & 1/1 & 1 & 45 \\ 
   & p=30 & 2/2/9 & 10 & 10 & 2/1 & 7/2 & 7 & 6 \\ 
   & p=60 & 17/16/13 & - & 14 & 11/10 & 6/8 & 8 & 13 \\ 
   & p=180 & 11/11/13 & - & 16 & 32/30 & 19/18 & 10 & 34 \\ 
   \midrule 
 \multicolumn{1}{|c}{} &       & \multicolumn{7}{c|}{Relative efficiency} \\
\midrule\multirow{4}[2]{*}{hsnr} & p=14 & 1/1/1 & 1 & 1 & 0.98/0.98 & 0.98/0.95 & 1 & 0.18 \\ 
   & p=30 & 0.98/0.98/0.96 & 0.98 & 0.98 & 0.97/0.97 & 0.9/0.89 & 0.98 & 0.78 \\ 
   & p=60 & 0.98/0.98/0.96 & - & 0.96 & 1/1 & 0.99/0.98 & 0.98 & 0.97 \\ 
   & p=180 & 1/1/0.99 & - & 0.96 & 0.79/0.79 & 0.9/0.89 & 1 & 0.78 \\ 
  \midrule\multirow{4}[2]{*}{msnr} & p=14 & 1/1/0.98 & 1 & 1 & 1/1 & 0.99/0.96 & 1 & 0.36 \\ 
   & p=30 & 0.98/0.98/0.94 & 0.93 & 0.94 & 0.98/0.98 & 0.95/0.94 & 0.96 & 0.91 \\ 
   & p=60 & 0.95/0.95/0.95 & - & 0.95 & 0.98/0.98 & 1/1 & 0.98 & 0.95 \\ 
   & p=180 & 0.99/1/0.98 & - & 0.95 & 0.8/0.81 & 0.93/0.92 & 1 & 0.79 \\ 
  \midrule\multirow{4}[2]{*}{lsnr} & p=14 & 1/1/0.95 & 0.98 & 0.98 & 1/1 & 0.99/0.99 & 0.99 & 0.69 \\ 
   & p=30 & 0.98/0.98/0.91 & 0.9 & 0.91 & 0.98/0.98 & 0.93/0.97 & 0.93 & 0.94 \\ 
   & p=60 & 0.91/0.91/0.93 & - & 0.93 & 0.95/0.96 & 1/0.98 & 0.98 & 0.93 \\ 
   & p=180 & 0.99/1/0.97 & - & 0.95 & 0.83/0.85 & 0.93/0.93 & 1 & 0.83 \\ 
   \midrule 
 \multicolumn{1}{|c}{} &       & \multicolumn{7}{c|}{Sparsistency (number of extra variables)} \\
\midrule\multirow{4}[2]{*}{hsnr} & p=14 & 14/14/14 & 14 & 14 & 14/14 & 14/14 & 14 & 13 \\ 
   & p=30 & 30/30/30 & 30 & 30 & 30/30 & 30/30 & 30 & 28.8 \\ 
   & p=60 & 53.6/53.3/40.3 & - & 40 & 55.4/56.9 & 48.4/49.4 & 43.8 & 48.4 \\ 
   & p=180 & 36/34.5/35.1 & - & 32.6 & 106.5/113.5 & 76.5/77.1 & 43 & 63.7 \\ 
  \midrule\multirow{4}[2]{*}{msnr} & p=14 & 14/14/14 & 14 & 14 & 14/14 & 14/14 & 14 & 13 \\ 
   & p=30 & 30/30/28.6 & 28.3 & 28.5 & 29.8/29.9 & 29.4/29.7 & 29.1 & 28.2 \\ 
   & p=60 & 47.5/46.6/31.6 & - & 31.2 & 51.5/53.4 & 42.1/48.1 & 36.3 & 43 \\ 
   & p=180 & 27.3/26.2/27.1 & - & 24.1 & 90.5/98.9 & 60/54.4 & 35.2 & 53.6 \\ 
  \midrule\multirow{4}[2]{*}{lsnr} & p=14 & 14/14/13.9 & 13.9 & 13.9 & 14/14 & 14/14 & 13.9 & 12.8 \\ 
   & p=30 & 29.1/29/22.7 & 21.2 & 22.1 & 28/28.9 & 23/27.6 & 24.1 & 25.5 \\ 
   & p=60 & 36.1/34.7/21.1 & - & 19.8 & 42.2/45.5 & 28.3/34.9 & 26.5 & 34.9 \\ 
   & p=180 & 17/16/17 & - & 14.3 & 61.8/72 & 43.7/33.1 & 25.3 & 43.6 \\ 
   \bottomrule 
\end{tabular}
}
\end{table}

% latex table generated in R 3.6.1 by xtable 1.8-4 package
% Sat Nov  9 19:30:57 2019
\begin{table}[ht]
\centering
\caption{The performance of BOSS compared to other methods, Dense, $\rho$=0.9, n=200} 

\scalebox{0.7}{
\begin{tabular}{|c|c|ccccccc|}
  \toprule 
 \multicolumn{1}{|c}{} &       & BOSS  & BS    & FS    & lasso & gamma lasso & SparseNet & \multicolumn{1}{c|}{rlasso} \\
 \multicolumn{1}{|c}{} &       & C$_p$-hdf/AICc-hdf/CV & CV    & CV    & AICc/CV & AICc/CV & CV    & \multicolumn{1}{c|}{CV}  \\
 \cmidrule{3-9}\multicolumn{1}{|c}{} &       & \multicolumn{7}{c|}{\% worse than the best possible BOSS}  \\
 \midrule\multirow{4}[2]{*}{hsnr} & p=14 & 0/0/5 & 0 & 0 & 0/0 & 6/7 & 1 & 64 \\ 
   & p=30 & 2/2/9 & 10 & 8 & 2/2 & 14/15 & 8 & 9 \\ 
   & p=60 & 15/19/14 & - & 12 & 12/11 & 12/13 & 14 & 14 \\ 
   & p=180 & 46/15/12 & - & 12 & 71/43 & 23/24 & 20 & 46 \\ 
  \midrule\multirow{4}[2]{*}{msnr} & p=14 & 1/1/7 & 8 & 6 & 2/1 & 8/8 & 5 & 24 \\ 
   & p=30 & 4/5/12 & 12 & 10 & 4/3 & 8/5 & 9 & 7 \\ 
   & p=60 & 20/23/16 & - & 13 & 30/17 & 12/15 & 15 & 19 \\ 
   & p=180 & 58/26/15 & - & 17 & 61/46 & 31/37 & 26 & 49 \\ 
  \midrule\multirow{4}[2]{*}{lsnr} & p=14 & 11/13/18 & 19 & 18 & 11/10 & 13/12 & 18 & 16 \\ 
   & p=30 & 19/15/14 & 15 & 13 & 10/11 & 19/12 & 12 & 11 \\ 
   & p=60 & 14/11/12 & - & 12 & 9/12 & 34/13 & 12 & 12 \\ 
   & p=180 & 59/12/10 & - & 10 & 15/17 & 106/12 & 9 & 11 \\ 
   \midrule 
 \multicolumn{1}{|c}{} &       & \multicolumn{7}{c|}{Relative efficiency} \\
\midrule\multirow{4}[2]{*}{hsnr} & p=14 & 0.99/0.99/0.95 & 0.99 & 0.99 & 0.99/0.99 & 0.94/0.93 & 0.99 & 0.61 \\ 
   & p=30 & 0.98/0.98/0.91 & 0.9 & 0.92 & 0.98/0.98 & 0.87/0.87 & 0.93 & 0.92 \\ 
   & p=60 & 0.87/0.84/0.87 & - & 0.89 & 0.89/0.9 & 0.89/0.88 & 0.88 & 0.88 \\ 
   & p=180 & 0.77/0.97/1 & - & 1 & 0.65/0.78 & 0.91/0.9 & 0.93 & 0.77 \\ 
  \midrule\multirow{4}[2]{*}{msnr} & p=14 & 0.99/0.99/0.94 & 0.93 & 0.94 & 0.98/0.99 & 0.93/0.93 & 0.95 & 0.8 \\ 
   & p=30 & 0.99/0.98/0.92 & 0.92 & 0.93 & 0.99/1 & 0.95/0.98 & 0.95 & 0.97 \\ 
   & p=60 & 0.93/0.92/0.97 & - & 0.99 & 0.86/0.96 & 1/0.97 & 0.98 & 0.94 \\ 
   & p=180 & 0.73/0.91/1 & - & 0.98 & 0.71/0.79 & 0.88/0.84 & 0.91 & 0.77 \\ 
  \midrule\multirow{4}[2]{*}{lsnr} & p=14 & 0.93/0.91/0.87 & 0.87 & 0.88 & 0.94/0.94 & 0.91/0.93 & 0.88 & 0.9 \\ 
   & p=30 & 0.92/0.96/0.97 & 0.96 & 0.97 & 1/1 & 0.92/0.98 & 0.98 & 0.99 \\ 
   & p=60 & 0.95/0.98/0.97 & - & 0.97 & 1/0.97 & 0.81/0.97 & 0.97 & 0.97 \\ 
   & p=180 & 0.69/0.97/0.99 & - & 1 & 0.95/0.93 & 0.53/0.98 & 1 & 0.98 \\ 
   \midrule 
 \multicolumn{1}{|c}{} &       & \multicolumn{7}{c|}{Sparsistency (number of extra variables)} \\
\midrule\multirow{4}[2]{*}{hsnr} & p=14 & 14/14/14 & 14 & 14 & 14/14 & 14/14 & 14 & 12.7 \\ 
   & p=30 & 29.5/29.3/25.2 & 23 & 24.6 & 28.9/29.5 & 24.4/24.8 & 26.2 & 26.7 \\ 
   & p=60 & 44.3/26.1/27.4 & - & 23.6 & 46.5/49.8 & 31.6/34.4 & 32.9 & 42.1 \\ 
   & p=180 & 46.7/15.6/21.3 & - & 17.2 & 54.4/93.4 & 44.8/46.9 & 37.7 & 71.4 \\ 
  \midrule\multirow{4}[2]{*}{msnr} & p=14 & 14/14/13.5 & 13.2 & 13.4 & 13.9/14 & 13.4/13.7 & 13.5 & 12.5 \\ 
   & p=30 & 28.6/28/20.2 & 16.6 & 19 & 26.4/27.9 & 19.3/24.4 & 22.3 & 25 \\ 
   & p=60 & 33/13/18.4 & - & 14.9 & 30.3/42.2 & 24.7/32.3 & 25.8 & 36.5 \\ 
   & p=180 & 41.6/5.8/14.1 & - & 9.3 & 4.8/42.5 & 46.7/30.9 & 26 & 30.2 \\ 
  \midrule\multirow{4}[2]{*}{lsnr} & p=14 & 10.6/9.8/7.1 & 6.4 & 6.6 & 8.9/9.3 & 7.5/8.6 & 7 & 7.5 \\ 
   & p=30 & 8.6/4.1/5.2 & 4.3 & 4.5 & 8.4/9.2 & 10.3/7.6 & 5.9 & 6.7 \\ 
   & p=60 & 3.7/1.4/4 & - & 2.7 & 2.7/8.1 & 14.5/6.2 & 5 & 5 \\ 
   & p=180 & 18/1/2.1 & - & 1.6 & 3.9/6.8 & 37.7/4 & 3.3 & 3 \\ 
   \bottomrule 
\end{tabular}
}
\end{table}

% latex table generated in R 3.6.2 by xtable 1.8-4 package
% Sat Dec 21 23:36:50 2019
\begin{table}[ht]
\centering
\caption{The performance of BOSS compared to other methods, Dense, $\rho$=0.9, n=2000} 
\scalebox{0.7}{
\begin{tabular}{|c|c|ccccccc|}
  \toprule 
 \multicolumn{1}{|c}{} &       & BOSS  & BS    & FS    & LASSO & Gamma LASSO & SparseNet & \multicolumn{1}{c|}{rLASSO} \\
 \multicolumn{1}{|c}{} &       & C$_p$-hdf/AICc-hdf/CV & CV    & CV    & AICc/CV & AICc/CV & CV    & \multicolumn{1}{c|}{CV}  \\
 \cmidrule{3-9}\multicolumn{1}{|c}{} &       & \multicolumn{7}{c|}{\% worse than the best possible BOSS}  \\
 \midrule\multirow{4}[2]{*}{hsnr} & p=14 & 0/0/0 & 0 & 0 & 3/4 & 27/27 & 2 & 0 \\ 
   & p=30 & 0/0/3 & 1 & 1 & 2/2 & 106/106 & 1 & 3 \\ 
   & p=60 & 7/8/9 & - & 6 & 6/6 & 74/74 & 8 & 7 \\ 
   & p=180 & 10/10/10 & - & 9 & 39/38 & 56/56 & 17 & 35 \\ 
  \midrule\multirow{4}[2]{*}{msnr} & p=14 & 0/0/3 & 0 & 0 & 1/1 & 8/8 & 1 & 0 \\ 
   & p=30 & 0/0/5 & 7 & 5 & 1/1 & 25/25 & 3 & 6 \\ 
   & p=60 & 12/12/10 & - & 8 & 9/8 & 11/11 & 9 & 10 \\ 
   & p=180 & 10/11/12 & - & 9 & 41/39 & 22/23 & 16 & 38 \\ 
  \midrule\multirow{4}[2]{*}{lsnr} & p=14 & 1/1/7 & 5 & 4 & 1/1 & 5/8 & 3 & 3 \\ 
   & p=30 & 3/3/11 & 11 & 9 & 3/2 & 7/6 & 8 & 8 \\ 
   & p=60 & 19/18/14 & - & 10 & 14/13 & 8/11 & 12 & 15 \\ 
   & p=180 & 12/12/13 & - & 11 & 40/38 & 27/26 & 18 & 37 \\ 
   \midrule 
 \multicolumn{1}{|c}{} &       & \multicolumn{7}{c|}{Relative efficiency} \\
\midrule\multirow{4}[2]{*}{hsnr} & p=14 & 1/1/1 & 1 & 1 & 0.97/0.97 & 0.79/0.79 & 0.98 & 1 \\ 
   & p=30 & 1/1/0.97 & 0.99 & 0.99 & 0.98/0.98 & 0.49/0.49 & 0.99 & 0.98 \\ 
   & p=60 & 0.98/0.98/0.97 & - & 1 & 1/1 & 0.61/0.61 & 0.97 & 0.99 \\ 
   & p=180 & 0.99/0.99/0.99 & - & 1 & 0.78/0.79 & 0.7/0.7 & 0.93 & 0.8 \\ 
  \midrule\multirow{4}[2]{*}{msnr} & p=14 & 1/1/0.97 & 1 & 1 & 0.99/0.99 & 0.92/0.92 & 0.99 & 1 \\ 
   & p=30 & 1/1/0.96 & 0.94 & 0.95 & 1/1 & 0.8/0.8 & 0.97 & 0.95 \\ 
   & p=60 & 0.96/0.96/0.98 & - & 1 & 0.99/0.99 & 0.97/0.97 & 0.99 & 0.98 \\ 
   & p=180 & 0.98/0.98/0.98 & - & 1 & 0.77/0.78 & 0.89/0.88 & 0.94 & 0.79 \\ 
  \midrule\multirow{4}[2]{*}{lsnr} & p=14 & 1/1/0.94 & 0.95 & 0.97 & 1/1 & 0.96/0.93 & 0.98 & 0.98 \\ 
   & p=30 & 0.99/0.99/0.92 & 0.92 & 0.94 & 1/1 & 0.96/0.96 & 0.94 & 0.95 \\ 
   & p=60 & 0.91/0.92/0.95 & - & 0.99 & 0.95/0.96 & 1/0.98 & 0.97 & 0.95 \\ 
   & p=180 & 1/0.99/0.99 & - & 1 & 0.8/0.81 & 0.88/0.88 & 0.94 & 0.81 \\ 
   \midrule 
 \multicolumn{1}{|c}{} &       & \multicolumn{7}{c|}{Sparsistency (number of extra variables)} \\
\midrule\multirow{4}[2]{*}{hsnr} & p=14 & 14/14/14 & 14 & 14 & 14/14 & 14/14 & 14 & 14 \\ 
   & p=30 & 30/30/29.9 & 29.9 & 29.9 & 30/30 & 26.1/26.1 & 30 & 29.9 \\ 
   & p=60 & 53.2/52.7/39.2 & - & 36.3 & 55.5/57.2 & 29.4/29.4 & 46 & 49.5 \\ 
   & p=180 & 37/35/38.6 & - & 30.2 & 109.6/118.6 & 43.1/43.1 & 52.4 & 74.3 \\ 
  \midrule\multirow{4}[2]{*}{msnr} & p=14 & 14/14/14 & 14 & 14 & 14/14 & 14/14 & 14 & 14 \\ 
   & p=30 & 29.9/29.9/28.2 & 27.3 & 27.9 & 29.7/29.9 & 26.3/26.3 & 28.8 & 28.7 \\ 
   & p=60 & 47.5/46.6/31.3 & - & 27.4 & 51.1/53.4 & 35.5/35.6 & 37.2 & 44 \\ 
   & p=180 & 30.8/28.4/32 & - & 23 & 95.1/104.1 & 62.7/59 & 44.6 & 71.3 \\ 
  \midrule\multirow{4}[2]{*}{lsnr} & p=14 & 14/14/13.8 & 13.6 & 13.7 & 14/14 & 13.8/13.9 & 13.8 & 13.8 \\ 
   & p=30 & 28.8/28.8/21.2 & 18.5 & 20.2 & 27.6/28.4 & 21.7/24.8 & 23.5 & 24.8 \\ 
   & p=60 & 35.3/33.6/20.6 & - & 16.3 & 41/43.9 & 28.5/33.1 & 26.4 & 35.4 \\ 
   & p=180 & 18.5/16.6/21.4 & - & 11.8 & 65.3/76.1 & 49.6/41.2 & 32.3 & 60.8 \\ 
   \bottomrule 
\end{tabular}
}
\end{table}




\clearpage
\bibliographystyleonline{chicago}
\bibliographyonline{reference.bib}